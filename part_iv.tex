\addcontentsline{toc}{subsection}{Quantity and quality of a vowel; vowel as consonant; division of a word, notes 1-3}
1. (§ 1) In the pronunciation of a vowel quantity and quality are to be distinguished. The quantity is the time required for the pronunciation of the vowel. It may be long, short, or medium. The quality depends on the shape which the mouth must assume for the pronunciation of the vowel (more about quantity and quality of a vowel and about what constitutes a real diphthong is contained in Practical and Theoretical Esperanto, pp. 7‒8).

The quantity of the vowels in Ido is medium. As to quality the vowels \textbf{e} and \textbf{o} may have either the closed pronunciation, as in the French word né and the English word dome, or the open pronunciation, as in debt, dog. It is however, recommendable to employ the closed pronunciation in open syllables, i.e., those ending with a vowel, and the open pronunciation in closed syllables, i.e., those ending with a consonant, for example, closed \textbf{e}: \textbf{de}, \textbf{me}, \textbf{ne}, \textbf{bela}, \textbf{lego}, \textbf{vera}, \textbf{elevar}; open \textbf{e}: \textbf{ed}, \textbf{el}, \textbf{mem}, \textbf{nek}, \textbf{per}, \textbf{gento}, \textbf{tempo}, \textbf{direktar}, \textbf{diversa}; closed \textbf{o}: \textbf{lo}, \textbf{no}, \textbf{po}, \textbf{bona}, \textbf{mola}, \textbf{nomo}, \textbf{memoro}; open \textbf{o}: \textbf{od}, \textbf{ol}, \textbf{dop}, \textbf{por}, \textbf{forta}, \textbf{kolda}, \textbf{pronta}, \textbf{exportar}. The vowel \textbf{a} should be pronounced neither too closed nor too open (VII, 400).

2. (§ 2, 3, 4) It is rather strange to say a vowel becomes a consonant (Gramm. Compl. p. 8, 2). This is the reason why the words ``sort of'' in parentheses have been added in the text and why the vowel \textbf{u} has been called a quasi-consonant in the instances where it has the character of a consonant. The quasi-consonant \textbf{u} sounds like English w. It can be easily pronounced between a preceding consonant (except some consonants, as \textbf{l}, \textbf{r}, etc.) and a succeeding vowel. In these instances the vowel \textbf{u} may therefore be rightly designated as a consonant, for its sound is not the same as when it is a pure vowel. After \textbf{a} and \textbf{e}, however, the \textbf{u} retains its pure vowel sound and cannot properly be called a consonant.

3. (§ 5) When a word has to be divided at the end of a line, each part must contain a vowel, i.e., be a syllable for itself, and the components of the digraphs and diphthongs must not be separated (IV, 433, dec. 485).

4. (§ 7) Names of peoples and of languages without a definite country are to be considered as proper names just as those with a country; therefore \textbf{Judo}, \textbf{Malayo}, \textbf{Eskimoo}, \textbf{Hebrea}, \textbf{Latina}. Proper names and their derivations used in a metaphorical sense require capital letters, as \textbf{Drakonala}, Draconic; \textbf{Platonala}, Platonic; \textbf{Bizancano}, one who wastes time in discussing useless subtleties.

On the other hand words like albino, creole, mulatto are no proper names, but only common names or names of species and must therefore have small initial: \textbf{albino}, \textbf{kreolo}, \textbf{mulato} (VI, 606).

For the derivations from the proper name Mefistofeles the special root \textbf{mefistofel} has been adopted, and therefore they have small initials: \textbf{mefistofelala}, relating to Mephistopheles; \textbf{mefistofelatra}, in the manner of Mephistopheles, satanic (V, 52).

\addcontentsline{toc}{subsection}{As good, as much, as possible, note 5}
5. (§ 15) Phrases such as: as good as possible, are best translated by \textbf{maxim bona posible}. For if \textbf{tam-kam} were used, the question would arise whether \textbf{posibla} or \textbf{posible} should be taken, \textbf{tam bona kam posibla} or \textbf{tam bona kam posible}. The first construction is more logical but rather ambiguous, implying a comparison between goodness and possibility. To the second construction the objection may be raised that \textbf{kam posible} may not be admissible because it is an abbreviation of a complete sentence: \textbf{kam esas posibla}.

The adverb as much as possible is \textbf{maxim multe posible}. Instead of this one may say shorter: \textbf{maxime posible} (§ 34). \textbf{Ni evintis maxime posible la radiki finanta per vokali}, we have avoided as much has possible the roots ending with vowels (VII, 90, line 3 from below).

\addcontentsline{toc}{subsection}{Pronoun \textbf{co}, \textbf{to}, \textbf{quo}; \textbf{omno quo}; \textbf{altru}, \textbf{irgu}; etc., notes 6-8}
6. (§ 25, 26, 27) It causes confusion to say that the pronouns co, to, quo refer to undetermined things (``on n'emploie quo que lorsqu'il s'agit d'une chose indéterminée'', Gramm. Compl. § 21, p. 15; II, 148‒150). These pronouns never refer to things, determined or undetermined (\textbf{nedeterminita kozi}; II, 3), but only to facts, and to call a fact an undetermined thing (chose indéterminée, \textbf{nedeterminita kozo}) is confusing.

7. (§ 27) It would be more proper to say \textbf{ulo qua}, \textbf{omno qua}, \textbf{co qua}, etc., instead of the customary \textbf{ulo quo}, \textbf{omno quo}, \textbf{co quo}, etc. The form \textbf{quo} may be justified after the pronouns \textbf{co} and \textbf{to} because they refer to facts so that the succeeding \textbf{quo} would also refer to a fact indirectly. But with pronouns like \textbf{nulo}, \textbf{ulo}, \textbf{omno}, etc. the custom of using \textbf{quo} instead of \textbf{qua} is rather open to criticism. For \textbf{nulo}, \textbf{ulo}, \textbf{omno}, etc. do not express facts so that the succeeding quo does not even indirectly refer to a fact, and yet the author's view that the pronouns \textbf{co}, \textbf{to}, \textbf{quo}, \textbf{lo} refer to facts only has been generally accepted (II, 149, 150, 721; III, 147).

8. (§ 30, 3) The forms \textbf{altru}, \textbf{irgu}, \textbf{omnu}, etc. refer to undetermined or indefinite persons. The postulate has, therefore, been made that relative and demonstrative pronouns should have similar forms, \textbf{cu}, \textbf{tu}, \textbf{quu}, to refer to undetermined or indefinite persons (II, 1‒4).

9. (§ 41) It had been proposed to the academy to connect the multiplicator with the multiplicand by a hyphen: \textbf{quara-dek}, 40; \textbf{oka-cent}, 800; \textbf{nona-mil}, 9000; etc. For a word like \textbf{sepadek}, one may put the accent on the \textbf{a}, thus \textbf{sepâdek}. This proposition has not been accepted, yet some manuals of the best authors did use a hyphen in such numbers (VII, 41).

10. (§ 41) It is permissible to say \textbf{dek ed un}, \textbf{dek ed ok} instead of \textbf{dek e un}, \textbf{dek e ok}. Some may even prefer the first forms because they do not contain a hiatus. The assertion has been made by a prominent authority (privately to the writer) that \textbf{dek e un}, \textbf{dek e ok} are clearer than \textbf{dek ed un}, \textbf{dek ed ok}. No reason was given why this should be the case. Probably the reason is that \textbf{dek e un}, \textbf{dek e ok} are uniform with the other additive numbers \textbf{dek e du}, \textbf{dek e tri} as to the connecting particle. This assertion is open to question (§ 50).

11. (to page \pageref{parts}) The verbality of the roots of some words may be questioned because the words denote concrete things as \textbf{grelo}, hail; \textbf{lumo}, light; \textbf{nivo}, snow; \textbf{pluvo}, rain; \textbf{rabato}, discount; etc. Yet closer consideration shows that the roots are verbal. Those things are the result of certain actions. The latter must first take place before the former come into existence. The verbs formed from those roots are, therefore, primary and the nouns secondary (see also note 13).

\addcontentsline{toc}{subsection}{Fundamental principle of unambiguity and its consequences upon word deriviation, note 12}
12. (to page \pageref{reversibility}) The principle of reversibility follows from the more general principle of unambiguity formulated by W. Ostwald: every element of a word (root, affix) must have but one invariable meaning which it must retain in any combination it may enter into (one sign—one invariable sense). One consequence of this general principle is that every derivation must be reversible. For form and meaning of a derivative are obtained by adding an element to a word, and when this element is omitted, the original word is restored in form and therefore the original meaning must be reverted to since the same form must always have the same meaning. If the original meaning is not obtained, the derivation is wrong. Suppose, for instance, from the word \textbf{krono}, the crown, were derived the verb \textbf{kronar}, meaning to crown. According to a general rule the noun derived immediately from a verb denotes an action. The noun \textbf{krono}, therefore, formed from the verb \textbf{kronar} would mean the act of crowning, while before the retransformation the same word \textbf{krono} meant the crown which is not identical with the act of crowning. The derivation of the verb \textbf{kronar}, to crown, from the noun \textbf{krono}, the crown, is, thus, not reversible and therefore inadmissible. The same inconsistency would be arrived at if any other meaning than to crown were given to the derived verb \textbf{kronar}. The general rule can be deducted therefrom that a verb cannot be derived immediately from a noun which does not denote an action, state, or relation.

Another consequence or requirement of the principle of unambiguity is that one element of a word should not change the sense of any other element of the word and therefore every element of idea in the meaning of a composed word must be indicated by an element of form. Since the affixes have meanings by themselves, it follows that the sense of a word built by affixes is at once given. It comprises the meanings of all the elements of the word.

The grammatical endings have no meanings by themselves, but merely serve to impart to a word its grammatical rôle. The meanings of words built with these endings are, therefore, contained in the roots alone. They are not arbitrary, but determined by the above principles. Yet they have to be defined. It is not sufficient to say, for instance, a noun is obtained by a verb by changing the ending \textbf{-ar} into \textbf{-o}, but the meaning of the noun thus formed must be stated precisely. The rules of derivation expose these definitions. The test for the correctness of a particular derivation is the reversibility of the latter.

13. (§ 53) Sometimes the verbal noun denotes both the action and its result, the idea of which is often inseparable from that of the action itself (dont l'idée est souvent inséparable de celle de l'action elle-même, Etude sur la Dériv., p. 9): \textbf{dekreto} is the act of decreeing and the decree itself; \textbf{lumo} the act of giving light and the light; \textbf{nebulo} the act of producing fog and the fog; \textbf{noto} the act of making notes and the note; \textbf{odoro} the act of giving forth an odor and the odor, etc. (See also note 22.)

\addcontentsline{toc}{subsection}{Substantivizing of the adjective, note 14}
14. (§ 54) The importance of more detailed rules regarding the immediate substantivizing of the adjective is proved by the vast discussion of the subject contained in the official organ Progreso, and justifies the rather extensive treatment of § 54. The latter contains nothing that would disagree with any official rule. There is only one official rule for the immediate transformation of an adjective into a substantive (Grammaire Complète par Ido, § 111, p. 43): ``On peut et doit substantifier un adjectif par simple changement de finale grammaticale.'' If this rule had added explicitly that the substantive thus obtained usually denotes a person, this additional statement could be proved to be arbitrary and not logical. On the other hand since this rule does not define closely the meaning of such a substantive, and the two adjectives it gives as examples (\textbf{blinda}, \textbf{virga}) express qualities essentially human (animal), the custom has become established in practice to consider the immediately substantivized adjective as usually denoting a person. This is proper with adjectives expression qualities essentially human, but no reason whatsoever can be adduced why substantives like \textbf{alto}, \textbf{belo}, \textbf{palo}, \textbf{perfekto}, \textbf{natanto}, \textbf{sendito}, etc., should mean a high, beautiful, pale, perfect, swimming, sent, etc., person rather than thing. Moreover it is difficult to reconcile this custom in practice with the principle of reversibility (II, 147). For if the substantives \textbf{alto}, \textbf{belo}, \textbf{palo}, etc., signify a high, beautiful, pale, etc., person, we should, in passing back to the adjective, receive adjectives applicable only to persons but not to things. Now this does not hold good; adjectives like \textbf{alta}, \textbf{bela}, \textbf{pala}, \textbf{perfekta}, \textbf{natanta}, \textbf{sendita}, and hundreds of others express qualities applicable to things no less than to persons.

At any rate the rule, given in § 54, that with adjectives like those just mentioned the context determines the meaning of the substantitives more closely, not only violates no official rule, but even does not disagree with the custom in practice.\footnotemark[1]
\footnotetext[1]{The decreeing of a period of stability in the spring of 1914 (see Preface, p. 5) prevented the publication of an essay by the writer proving conclusively that the adjective substantivized immediately can logically have no other but the Aristotelian (I, 563; II, 146, 475) sense and that a special suffix is therefore needed for transforming the adjective into a substantive denoting a person. This essay will probably be published later.}

15. (Prefix \textbf{ge-}, p. 41) Since the adoption of the rule that substantives in general denote undetermined sex (dec. 1089, 1090, VI, 212) the prefix \textbf{ge-} is used only when emphasis is to be put upon the union of both sexes. Parents had to be translated formerly by \textbf{gepatri} because \textbf{patri} alone signified only fathers. But now \textbf{patri} means parents. Ladies and gentlemen had to be translated by \textbf{gesiori} or better yet and more ceremoniously (Gramm. Compl., p. 52) by \textbf{siorini e siori}, now \textbf{siori} alone suffices.

16. (Prefix \textbf{des-}, p. 41) The use of the prefix \textbf{des-} has rightly been restricted to a great extent, original roots having been adopted for a good many words formerly built with it: \textbf{rara}, rare, instead of \textbf{desofta}; \textbf{mala}, bad, instead of \textbf{desbona}; \textbf{mikra}, small, instead of \textbf{desgranda}; \textbf{stulta}, foolish, instead of \textbf{dessaja}; \textbf{malada}, sick, instead of \textbf{dessana}; \textbf{trista}, sad, instead of \textbf{desgaya}, etc., etc.

\addcontentsline{toc}{subsection}{Suffixes \textbf{-ier}, \textbf{-il}, \textbf{-ist} notes 17-19}
17. (Suffix \textbf{-ier}, p. 44) Dr. Couturat's contention that \textbf{gibiero} is not admissible (VI, 595), is not supported by decisions 592, 593 (IV, 562) through which \textbf{-ier} has received the definition: ``characterized by.'' A marginal note states expressly that the suffix \textbf{-ier} replaces the suffix \textbf{-ul} of the Grammaire Complète, and here (p. 54, § 125) \textbf{gibulo} (\textbf{gibiero}) is cited with the meaning hunch back.

18. (Suffix \textbf{-il}, p. 44) It is difficult to decide whether an independent verb should be created and the instrument for the action should be derived from it by the suffix \textbf{-il}, or an independent word for the instrument be selected and the verb derived from it by the suffix \textbf{-ag}. It would appear that when the idea of the instrument is primitive, the verb should be the derived word, and when the idea of the action is primitive, the instrument should be the derived word. This again is difficult to decide as shown by an attempt to do so which seems to involve itself in contradictions (III, 212, 292).

19. (Suffix \textbf{-ist}, p. 45) The definition of the suffix \textbf{-ist} is not arbitrarily invented, but determined by the international sense of this international suffix. In the greatest majority of cases this sense is that of professional occupation with a certain thing. Accordingly the suffix would appear to be applicable only to substantive roots and by the intermedium of the verbal substantive also to verbal roots, but not to adjective roots. For the root of a word is needed that designates the thing one is occupied with professionally. Words such as \textbf{socialisto}, \textbf{materialisto}, etc., if they are to be derivations in Ido and not integral root words, cannot be regarded as composed of the adjectives \textbf{sociala}, \textbf{materiala}, etc., and the suffix \textbf{-isto}, but must be considered as consisting of the nouns social, etc., and the suffix. If these nouns had the Aristotelian sense postulated by the writer (I, 563; II, 146, 475) the words \textbf{socialisto}, etc., would properly be derivations in Ido. But since they have not the Aristotelian sense, an extension had to be added to the definition of the suffix \textbf{-ist} (Gramm. Compl., § 124, p. 54) to justify as derivations words such as \textbf{idealisto}, \textbf{materialisto}, \textbf{socialisto}, etc. Those to whom this extension might appear rather arbitrary because it does not readily follow from the definition of the suffix may use the words anyhow, regarding them as integral root words. For all these words are very international as such.

20. (Suffix \textbf{-al}, p. 45) The suffix \textbf{-al} is useless in many adverbs. They are preferably derived directly from the nouns: \textbf{praktiko}, practice; \textbf{praktike}, practically; \textbf{principo}, principle; \textbf{principe}, in principle; \textbf{persono}, person; \textbf{persone}, in person; etc. The suffix, however, is necessary in the adjectives: \textbf{praktikala}, practical; \textbf{personala}, personal; \textbf{principala}, relating to principle (VI, 27). % originally 19.

21. (Suffix \textbf{-if}, p. 48) Although the verbs formed with the suffix \textbf{-if} are essentially intransitive because they contain in themselves the object, they may, in rare instances, assume a direct object: \textbf{urinifar sango}, to discharge blood with the urine; \textbf{sudorifar sango}, to sweat blood. It would also be permissible to say: \textbf{urinifar kun sango} (V, 350; see also end of § 72 and note 49). % originally 20.

\addcontentsline{toc}{subsection}{Suffix \textbf{-ig} with mixed verbs and with transitive verbs used intransitively, note 22}
22. (Suffix \textbf{-ig}, p. 49) The suffix \textbf{-ig} imparts to a transitive verb a passive sense: \textbf{trovigar} = \textbf{trovatigar} = \textbf{igar trovata}, to cause to be found. This justifies also the use of the preposition \textbf{da} with such a verb. With intransitive verbs the suffix \textbf{-ig} has the sense of \textbf{-antigar} or \textbf{igar-anta}: \textbf{dormigar} = \textbf{dormantigar} = \textbf{igar dormanta} (Etude sur la Dériv., p. 49). % originally 21.

Mixed verbs used intransitively (see mixed verbs, § 112) have the construction of intransitive verbs:\textbf{ me cesigas la ludo} (\textbf{la ludo cesas}, mixed verb used intransitively), I make the play stop. Mixed verbs used transitively have the construction of the transitive verbs: \textbf{me cesigas la ludo da la pueri} (\textbf{la pueri cesas la ludo}, mixed verb used transitively), I make the children stop the play. Transitive verbs without object (for inst., \textbf{dicar}, \textbf{parolar}, \textbf{manjar}, etc., see note 48) have the construction of instransitive verbs: \textbf{la poeto paroligas sua protaganista} (\textbf{la protagonisto parolas}, transitive verb without object), the poet makes his hero speak.

Mixed verbs and transitive verbs without object, however may lead to ambiguities, especially when the subject of the non-transformed verb is not mentioned. \textbf{Lu glutigas la animalo}, may mean, he makes the animal swallow, or, he causes the animal to be swallowed (by the serpent, \textbf{da la serpento}), according to whether \textbf{glutar} is intended to be used intransitively or transitively. \textbf{El manjigis la hanyuno}, may mean, she fed the chicken, and also, she gave the chicken as food (to her child, \textbf{da sua filio}). \textbf{Lu manjigis sua kavalo}, may mean, he fed his horse, and also, he gave his horse as food (to the wild beasts, \textbf{da la sovaja bestii}). Where such ambiguities may arise, composed mixed verbs are better dissolved into their component parts and an appropriate construction employed. I made my horse eat or I fed my horse is best translated by, \textbf{me igis, ke mea kavalo majez}; and, I gave my horse as food, is best translated by, \textbf{me igis, ke mea kavalo esez manjata}. With an infinitive construction a proper order of the parts of the sentence will help to remove ambiguities. The construction, \textbf{me igis manjar mea kavalo}, is rather doubtful, for \textbf{kavalo} may be taken as the subject or the object of \textbf{manjar}; but in the order, \textbf{me igis mea kavalo manjar}, \textbf{kavalo} can only be the subject of \textbf{manjar}.

\addcontentsline{toc}{subsection}{Suffix \textbf{-iz}, note 23}
23. (Suffix \textbf{-iz}, p. 49) The definition that the suffix \textbf{-iz} signifies to provide with would appear to render it applicable only to a substantive root. For the root of a word is necessary which denotes an object, the object one provides with. In general this is indeed the case so that in by far the greatest majority of instances the suffix is met with joined to a nominal root. But many a verbal noun denotes, besides the action expressed by the verb, also its result the idea of which is often inseparable from that of the action itself (dont l'idée est souvent inséparable de celle de l'action elle-même, Etude sur la Dériv., p. 9). \textbf{Kopio} means the act of copying and the copy; \textbf{lumo} the act of giving light and the light; \textbf{nivo} the act of snowing and the snow; \textbf{noto} the act of making notes and the note (the words written as a note); \textbf{pluvo} the act of raining and the rain (the falling water); \textbf{respondo} the act of answering and the answer (the words used as an answer); \textbf{skiso} the act of sketching and the sketch; \textbf{vundo} the act of wounding and the wound; etc. Such verbal nouns, therefore, denote objects, with which something may be provided, and to the roots of such nouns representing verbal roots the suffix \textbf{-iz} may properly be applied in conformity with the above definition. % originally 22.

Useful transitive verbs may thus be formed from a verbal root and the suffix \textbf{-iz}, for instance \textbf{injektizar ulu} or \textbf{ulo}, to inject somebody or something, i.e., to charge somebody or something with a fluid by injection; \textbf{nomizar ulu} or \textbf{ulo}, to name somebody or something, i.e., to give a name to, to provide with a name, while \textbf{nomar} alone means to say somebody's name (see note 51; III, 38, 296).

The above seems to be a simple and plausible way to explain the propriety of using the suffix \textbf{-iz} with verbal roots. It sounds rather strange to justify it by saying (VI, 296, 596) a verb formed from a verbal root and the suffix \textbf{-iz} signifies to provide with the act expressed by the root, for instance: \textbf{respondizar letro}, to provide a letter with the act of answering; \textbf{notizar libro}, to provide a book with the act of making notes (``\textbf{garnisar libro per l'ago ipsa notar}''); etc.

24. (Suffix \textbf{-um}, p. 50) From \textbf{nazo}, nose, the word \textbf{nazumo}, pince-nez, was formed. This word has been made superfluous. \textbf{Binoklo} had the meaning: opera-glass, but through decision 758 (V, I, 659) it received the meaning: eye glasses instead of opera-glass. The three forms of eye glasses are: \textbf{naz-binoklo} (\textbf{nazo-binoklo}), pince-nez; \textbf{manu-binoklo}, lorgnette; \textbf{orel-binoklo}, spectacles, barnacles. Opera-glass is \textbf{bilorneto} (dec. 1243, VII, 70). % originally 23.

25. (§ 71) Prepositions or adjectives are used to form words built in natural languages with affixes: \textbf{kontrenatura}, unnatural; \textbf{kontrelega}, illegal; \textbf{segunlega}, lawful; \textbf{senskopa}, aimless; \textbf{legokonforma}, lawful; \textbf{skopokonforma}, serviceable (VI, 28).  % originally 24.

\addcontentsline{toc}{subsection}{Intransitive verbs composed with prepositions treated transitively, note 26}
26. (§ 72) In some instances one may not agree to treating transitively a verb composed of an intransitive verb and a preposition, for instance: \textbf{ekirar }(\textbf{enirar})\textbf{ la chambro}, to walk out of (into) the room. For a transitive verb always admits of a passive voice. This would render justifiable the passive construction: \textbf{la chambro ekiresas} which would appear strange indeed. In such instances it is therefore preferable to dissolve the composition. The repetition of the preposition must at any rate be avoided. Irar ek (en) la chambro is prefrable to ekirar (enirar) la chambro, but the construction ekirar (enirar) ek (en) la chambro is objectionable as containing a tautology (see note 60 to preposition ad; Gramm. Compl., § 164; II, 411, note 1).  % originally 25.

27. (§ 73) The propriety of forming the adjectives \textbf{undiala}, \textbf{omnadiala}, of one day, of every day (daily), may be called into question. If they be considered as consisting of the adjectives \textbf{undia}, \textbf{omnadia} and the suffix \textbf{-al}, they are superfluous. For the adjectives \textbf{undiala}, \textbf{omnadiala} do not differ in meaning from the shorter ones \textbf{undia}, \textbf{omnadia}. They must, therefore, be considered as composed of the suffix \textbf{-al} and the nouns \textbf{undio}, \textbf{omnadio}. Such nouns hardly ever occur although they are correctly constructed (see § 54 and not e14; VI, 597, 598; Gramm. Compl., p. 73, § 162). % originally 26.

\addcontentsline{toc}{subsection}{Logic of the Ido system of derivation, note 28}
28. (p. 37) Conclusion of Part II Dealing with Derivatives and Word Formation. % originally 27.

The question whether or not there can be more than one logical system of derivation applicable to the international language has been most excellently treated by Dr. L. Couturat (I, 391-394). His answer is cited here almost unchanged, for there could hardly be a clearer answer to the question.

``There can exist no other system (than that of Ido) entirely regular and logical. For the whole system depends not upon the choice of some affixes (which are more or less arbitrary), but upon the immediate derivation, i.e. upon the relation between substantive, adjective, verb, and adverb formed form the same root by the grammatical endings \textbf{-o}, \textbf{-a}, \textbf{-ar}, \textbf{-e}. These relations cannot be invented arbitrarily nor defined in several ways. For the whole language is founded upon the principle that every root, even every element, of a word has only one invariable meaning which it retains in all compositions it enters into. On the other hand and because of the same principle (which is the great principle of unambiguity formulated by W. Ostwald: one sign—one invariable sense; and which is more fundamental than any book), the grammatical endings have no other sense than to show the grammatical species of a word or to indicate the different rôle which the idea of a word plays in a sentence. \textbf{Parolo} is the idea of `\textbf{parol}' as a substantive, \textbf{parolar} is the same idea as verb. Now if \textbf{parolo} has the same idea as \textbf{parolar}, only under substantive form, it can have no other sense but `\textbf{la ago parolar},' the action of speaking. For then, and only then, it will actually have the same sense as the verb. Likewise if the adjective \textbf{parola} has the same idea as the substantive, only under adjective form, it can signify only: ``which is \ldots'': \textbf{parola} = which is a word; for example, \textbf{parola kontrato}, a verbal contract, a contract which is a word. Finally if the adverb parole expresses the same idea as the adjective (substantive, verb) under adverbial form, it can signify only: ``\textbf{per maniero parola},'' in speaking manner, or ``\textbf{per parolo},'' by word. To summarize, the rules of immediate derivation are not at all arbitrary, nor even free, but determined absolutely by the logical principles of the language \ldots, if only these are accepted.

``Every derivation which is not immediate must be mediate, i.e., accomplished by some affix. For this means that the word desired contains another element than the idea of the root alone, that this root is, therefore, not sufficient to express the desired idea, that another ideal element is to be added to the idea of the root, another word element to the root. We have no right to decree that `\textbf{parola}' should have the sense of `\textbf{parolanta}' or of `\textbf{parolebla}' or of any other idea embracing that of `\textbf{parol}.' We are at liberty to select the necessary affix, but one must be selected and employed, and then it will merely be a synonym of \textbf{-ant}, \textbf{-ebl}. The selection can be discussed from a linguistic point of view, but not from one of logic. And since every single affix, according to Ostwald's fundamental principle of unambiguity (one word element—one invariable sense), must have only one sense, two affixes of Ido cannot be formally identified. The affixes of Ido can only be increased in number, but they cannot be diminished, for then two ideas distinct and to be distinguished would be fused. In short, one can invent a system similar to that of Ido or more complete, but none essentially different.''

\addcontentsline{toc}{subsection}{Position of the negativing particle \textbf{ne}, note 29}
29. (§ 75) Natural languages almost invariably negative the finite verb even when the negation would logically belong to the infinitive. This international idiom is easily explained and also justified. In the majority of cases it is very difficult to decide where the negation belongs, to the finite verb or to the infinitive. This would mean that it is irrelevant to make the distinction. The sentence: \textbf{me ne volas departar}, I do not want to depart, does not differ materially from: \textbf{me volas ne departar}, I have the will not to depart; both mean: I want to stay. Rightly, therefore, the best stylists of Ido employ this international idiom of negativing the finite verb even where the negation would belong to the infinitive. \textbf{Se on ne volas variar l'atak-angulo, on mustas variar la rapideso di la flugo}, if one does not wish to vary the angle of attack, he must vary the rapidity of the flight (L. Couturat, VI, 124). By following this international custom one is spared the racking of his brain in deciding where, according to strict logic, the negation would belong. This international idiom may be employed the more as no ambiguity can arise, especially when the sentence is not removed from its context. When, however, it is obvious where the negation belongs, as in the sentences: \textbf{vu devas ne mentiar}, and \textbf{vu ne devas pagar la taxo}, you must not (have the duty not to) lie, and you must not (have not the duty to) pay the tax, the negation is to be placed properly. % originally 28.

\addcontentsline{toc}{subsection}{Order of the parts of a sentence; of the genitive, note 30}
30. (§ 77) Richness in flections and agreement of the words in case, number, and gender are needed for great freedom in the order of the parts of a sentence. Modern languages lack these two qualities of the ancient ones and therefore require a certain normal order of the sentence for the sake of clearness. The international language destined for the expression of modern thought must follow this example (IV, 276). % originally 29.

The following sentence is a good example of a faulty order of the parts. \textbf{Quo esas verajo decidar, nur povas intelekto}. \textbf{Decidar} must precede, not succeed its complement \textbf{quo esas verajo}, and \textbf{nur} must not be separated from \textbf{intelekto}. The following three orders are proper: \textbf{decidar, quo esas verajo, nur intelekto povas}; \textbf{quo esas verajo, nur intelekto povas decidar}; \textbf{nur intelekto povas decidar, quo esas verajo}, only the intellect can decide what is truth. The last order seems to be the clearest of the three (VII, 161).

The genitive normally succeeds the noun it belongs to. When, however, the latter is accompanied by long complements, inversion of this order may be employed with advantage although no official rule expressly permits this inversion (III, 54, 695). \textbf{Di mea vicino la gardeno, quan vu vidis, esas bela}, my neighbor's garden, which you saw, is beautiful. In the order: \textbf{la gardeno di mea vicino, quan \ldots}, the relative pronoun would refer to \textbf{vicino}. \textbf{Di la domo la fenestri ornita per multa flori}, the house's windows ornamented with many flowers. In this order \textbf{ornita} belongs to \textbf{fenestri} while in the normal order it would belong to \textbf{domo}. \textbf{Di mea frato l'amiko preske tote griza malgre sua yuneso}, my brother's friend almost entirely gray in spite of this youth. \textbf{Griza} belongs to \textbf{amiko} while in the normal order it would belong to \textbf{frato}. (The order: \textbf{di mea frato l'amiko, malgre sua yuneso preske tote griza}, III, 147, is not deserving of imitation.) Inversion of the genitive may be used with advantage for emphasis. \textbf{Di la filio la vivo salvesez, Deo, mean prenez vice lua}, the child's life be saved, oh God, take mine instead of his.

\addcontentsline{toc}{subsection}{Review of the French method of using comma, note 31}
31. (§ 78) Review of the French Method of Using Comma. The principle ``separate all clauses by commas,'' if applied unexceptionally would indeed render the use of the comma easy even for the average scholar. The French method, however, of using comma with relative clauses is not as simple as might appear from the rule, formulated by the writer, that comma is to be used or omitted according to whether the relative clause is an accessory or a necessary part of the sentence. For it is often very difficult to decide whether one or the other is the case. % originally 30.

Another explanation of the French method has been given (VI, 380), which may be very useful at times and is therefore described here although it sometimes offers even greater difficulty. A comma is used with a relative clause when it is qualifying or explanatory, but not when it is determinative. In the sentence: \textbf{me ne amas la infanti, qui facas bruiso}, I do not love the children, who make a noise, the relative clause is qualifying or explanatory, i.e. it means all children or it is equal to: because children make a noise. In the sentence, however: \textbf{Me ne amas la infanti qui facas bruiso}, the relative clause is determinative, it means only those of the children who make a noise. Comma is therefore always necessary before a relative pronoun or adverb whenever the word is referred to is otherwise determined, for then the relative clause is qualifying, but not determinative. This occurs (a) when the word referred to is a proper name, (b) when the word referred to is accompanied by a determinative, as a demonstrative or possessive attribute. \textbf{La amiko qua ne audacas dicar la verajo ne esas vera amiko}, the friend who does not dare to tell the truth is no true friend; but \textbf{mea olda amiko, qua kustumis dicar la verajo, ne audicis cakaze dicar ol a me}, my old friend, who used to tell the truth, did not dare to tell it to me in this case. \textbf{Regardez ta homo, qua lekta jurnalo: lu laudas tamen omno quon ni dicas}, look at that person, who reads a journal: he nevertheless praises everything we say. \textbf{Napoleon, qua vivids l'enemiki avancar, sendis kontre li sua olda guardo}, Napoleon, who saw the enemies advance, sent his old guard against them.

The difficulty of the preceding explanation is twofold. It is sometimes less easy to decide whether a relative clause is qualifying or determinative than to decide whether it is an ancillary or a necessary part of the sentence. Second one usually forgets whether it is with the qualifying relative clauses that a comma must be used or with the determinative ones. The writer for one always experienced this difficulty and often had to take recourse to Progreso, VI, p. 380, until he chanced upon his own explanation which is more easily rememberable.

The difficulty of the preceding explanation is proved by the fact that its author, to satisfy the repeated demands of explanations (``\textbf{on demandis expliki pri la decido 1062},'' VI, 380), deemed it desirable to supplement it by another explanation which is very good indeed and much simpler (VI, 593): ``If the sentence loses every sense through the suppression of the relative clause, the latter is determinative (no comma is to be used); if the sentence conserves a sense, though less complete and less precise, after the suppression of the relative clause, the latter is explanatory (a comma must be used). \textbf{La homi qui sempre laboras esas plu felica kam la homi qui sempre ocias}, the people who always work are happier than the people who are always idle. This sentence has no sense without the relative clauses. \textbf{La animali, qui ne havas raciono, esas plu felica kam la homi, qui havas raciono}, the animals, which have no reason, are happier than men, who have reason. After suppression of the relative clauses this sentence still has a fairly good sense.

This explanation coincides with the shorter one of the writer, but has the same difficulty. Not all sentences with relative clauses are as plain as the last two so skillfully selected for the purpose, but frequently a relative clause is a necessary part of the sentence and therefore inseparable by comma, and yet the omission of the clause will not leave the sentence without a fairly good sense and therefore a comma must be used.

The following is a good illustration of the difficulty of the French method. Consider the following French sentence: le maître de mon ami que vous avez vu hier partira demain, the teacher of my friend whom you saw yesterday will depart to-morrow. A very capable linguist expressed the opinion that the French language can distinguish (``\textbf{La Franca pavas facar distingo}'') whether the relative clause refers to maître or to ami by using or omitting a comma and thus it can render this sentence clear. And yet a most competent French scholar (in a private letter to the writer) states that such a sentence remains ambiguous, comma or no comma (``\textbf{la frazo esas bisenca kun o sen komo}'').

The object of this rather too lengthy discussion is to show that the French method is quite complicated and that it would be much simpler to apply unexceptionally the method expressed in the principle: ``separate all clauses by commas.'' It is true, in some instances a sentence may be ambiguous with this method. But ambiguity will exist only when the sentence is removed from its context, not otherwise. In the sentence: \textbf{me ne ama la infanti, qui facas bruiso}, the context will show whether all children are meant or only some children. In the sentence: \textbf{la maestro di mea amiko, quan vu vidis hiere, departos morge}, the teacher of my friend whom you saw yesterday will depart to-morrow, the context will show who was seen, the teacher or the friend. Comma could, therefore, be used in both these sentence without fear of ambiguity. The slight disadvantage of having, once in a great while, to take recourse to the context is by far outweighed by the great advantage accruing from the simplicity of the method. The difficulty of the French method is proved by the fact that only expert linguists are able to employ it correctly and by the above illustration showing that two competent linguists may be at variance as to whether the use or omission of a comma can render clear an ambiguous sentence.

Fortunately for the student of Ido the principle: separate all clauses by commas, as been adopted for the greater part, thus sparing him to a great extent the difficulty of the French method which has been most expressively characterized as too full of caprice or nuance (``\textbf{tro kapricoza e nuancoza},'' II, 171). Regarding the relative clauses, however, for which the French method has been made obligatory, it may be safely predicted that unless the student be an expert grammarian he will frequently commit errors in spite of the excellent explanations and directions given by Dr. L. Couturat (decision 1062, VI, 21, must read: ``\textbf{on adoptas la Franca uzo di la komo avan la relative pronomi}''; \textbf{pos} is a misprint).

\addcontentsline{toc}{subsection}{Some punctuation marks, note 32}
32. (§ 78) The \textbf{apostrofo}, apostrophe (') is used to indicate the omission of a letter. It occurs in Ido almost only with the definite article the \textbf{a} of which is sometimes elided (§ 9). The elision of the \textbf{a} of the adjective is preferably not indicated by an apostrophe (§ 14). % originally 31.

While the \textbf{streketo} (-), hyphen, denotes connection, the \textbf{streko} (—), dash, longer than the \textbf{streketo}, is a sign of separation. The \textbf{streko} is used when the speaker or the subject is changed. In the latter case it is preferable to make a new paragraph (\textbf{alineo}). Differently from the English dash the \textbf{streko} does not indicate an interruption (see \textbf{puntaro}).

The \textbf{parentezi}, parentheses, include a word, phrase, or clause not belonging to the context. They must not be used too frequently, the understanding of the latter may be impeded thereby. Since they serve essentially for the purpose of including something, they must always be used as a pair, never singly as is often done with numbers. Instead of 1) either (1) or still better 1\textsuperscript{e} or 1\textsuperscript{·e} (abbreviation of \textbf{unesme}, firstly) is to be used.

Less frequent punctuation marks are the \textbf{puntaro} (\ldots), successive dots, \textbf{kramponi} [ ], brackets, \textbf{embracili} \{ \}, braces, and \textbf{noto-referi}, reference marks.

The \textbf{puntaro}, several succeeding points, indicate an interruption in general. In a sentence it may be produced either by another one or by the speaker himself in order to change the expression of this thoughts or to leave his reader guessing what his unfinished sentence means. In English a dash is used for this purpose. \textbf{Scrooge dicis, ke il video lu \ldots yes, il dicis ya}, Scrooge said that he would see him—yes, indeed he did (Dickens).

The \textbf{kramponi} and \textbf{embracili} are used in mathematics as parentheses of the second and third degree: \{ \ldots [ \ldots ( \ldots ) \ldots ] \ldots \}. The \textbf{kramponi} may be used in prose in similar cases and as special parentheses.

One \textbf{embracilo} is used to bring several lines on one side of it into relation to one line on the other side. The point faces this line, and the \textbf{concativity} faces the other lines: \ldots \hspace{-1em} \begin{tabular}{r l} \ldelim\{{3}{*}[] & \hspace{-1em} \ldots \\ & \hspace{-1em} \ldots \\ & \hspace{-1em} \ldots \end{tabular} Except in \textbf{synoptical} tables where it is really useful, the \textbf{embracilo} is to be employed but rarely. It disturbs the composition. It may be put also horizontally as in genealogy:\vspace{-1em}
\begin{center}Jacob \\ $\overbrace{\mathrm{Reuben, Simeon, Levi, Jehuda}}$\end{center}

The most convenient \textbf{noto-referi}, reference marks, are numbers. When many are needed on one page, they offer a sufficient supply. Furthermore other signs frequently used for reference, as the \textbf{steleto} (*), star, asterisk, and the \textbf{kruco} (†), cross, dagger-sign, obelisk, are also employed for other purposes. The \textbf{steleto}, for instance, denotes in linguistics a conjectural and in Ido a non-official form. The \textbf{kruco} generally indicates a deceased person or a date of death, and in philology an archaic form or obsolete word.

The quotation is marked differently in different languages and even in the same language not always in the same way. In English one or two inverted commas (`, ``) above the line mark the beginning, and one or two apostrophes mark the close of the quotation (',''). In Ido only the French signs are used, two pairs of small parallel arcs on the line (« \ldots »). Each pair must have its concavity towards the quotation.

33. (§ 85) An attribute of the object may be made recognizable by the accusative form. But to employ the accusative also for the purpose of distinguishing an attribute of the subject from the latter would be an unjustified solecism (see note 49). % originally 32.

34. (§ 89) The distinction between nationals and inhabitants is not so important or necessary that there should be different names for both (I, 652). The same name may therefore be used for nationals and inhabitants even in those instances in which the names of the peoples are derived from original adjectives: \textbf{Anglo} may mean both an Englishman and an inhabitant of England, etc. % originally 33.

The forms \textbf{Angliano}, \textbf{Franciano}, \textbf{Germaniano}, etc., are regularly constructed and therefore legitimate. They do not differ in meaning from the shorter forms \textbf{Anglo}, \textbf{Franco}, \textbf{Germano}, etc., which are preferable. Such shorter forms ought to exist in all instances, or at least in all those instances in which a country belongs to a people, distinct and independent and especially having a language of its own, as \textbf{Austria}, \textbf{Hungaria}, \textbf{Italia}, etc.

35. (§ 90) The adjectives German, Roman, are \textbf{Germana}, \textbf{Romana} (original adjectives), but the adjectives Germanic, Romanic, are \textbf{Germanala}, \textbf{Romanala} (derived adjectives). A member of the Germanic or Romanic race is \textbf{Germanalo}, \textbf{Romanalo}, while a German, a Roman is \textbf{Germano}, \textbf{Romano} (I, 452). % originally 34.

The dictionary is to be consulted to ascertain the name of the people of a country and the adjectives relating to both of them.

\addcontentsline{toc}{subsection}{Formulas of politeness in letters; personal names; address, note 36}
36. (§ 9) Formulas of politeness, especially in beginning and concluding a letter (P. de Janko, II, 679; \textbf{formuli di politeso}, IV, 470). In addressing a person, especially at the beginning of a letter, the simple ``\textbf{Sioro}'' (or \textbf{Siorulo}, \textbf{Siorino} when the gender of the addressee is known and it is desirable to express it) is sufficient. Title, function, or profession may be added: \textbf{Siroro Kolego}, \textbf{Prezidanto}, \textbf{Profesoro}, \textbf{Doktoro}, \textbf{Advokato}, etc. Qualifying adjectives, as \textbf{estimata}, \textbf{honorizata}, \textbf{altestimata}, etc., are superfluous. Persons of particularly high rank are addressed with \textbf{Sinioro}: \textbf{Sinioro Episkopo}, \textbf{Ministro}, \textbf{Generalo}, etc. % originally 35.

Parents, friends, etc., are addressed in a more intimate manner: \textbf{Kara amiko}, \textbf{Kara Maria}, \textbf{Mea kara filio}, \textbf{Kara filio}, or even \textbf{Mea filio}.

A letter is best concluded by: ``\textbf{kun sincera saluto}'' followed by the name of the writer. To be less formal \textbf{sincera} may be substituted by other epithets, as \textbf{amikala}, \textbf{simpatioza}, \textbf{devota}, etc.

In enunciating personal names, as in official documents, lists, etc., the family name is put first and the forename or forenames in second place (IV, 472; VI, 52).

In writing an address the English way is recommended, beginning with the name of the addressee followed by the street, city, country (IV, 470; VI, 52).

\addcontentsline{toc}{subsection}{Preposition \textbf{de} with terms of quantity; \textbf{di} with \textbf{propra}; \textbf{ego}; \textbf{sua}, notes 37-40}
37. (§ 92, c) The preposition de is required after terms of quantity only when quantity proper is indicated, but \textbf{di} is preferable to \textbf{de} when there exists a real genitive relation. \textbf{La nombro di lua adversi esas tante granda, ke il mustas sukombar}, the number of his adversaries is so great that he must succumb. Between \textbf{nombro} and \textbf{adversi} there is a genitive relation just as there would be one in \textbf{la potenteso di lua adversi}, the power of his adversaries. In general the practical rule holds good that a genitive relation exists whenever the noun connected with the term of quantity is accompanied by the article or a possessive adjective. \textbf{La granda multeso di la moyeni efektigis la vinko}, the great multitude of the resources effected the victory (III, 624). % originally 36.

38. (§ 95) It is bad style to say: \textbf{ica chambro havas dek e kin pedi de alteso} (or \textbf{alte}). It is clearer and more logical to say: \textbf{ica chambro esas alta de dek e kin pedi}, this room is 15 feet high (see preposition \textbf{de}; V, 342). % originally 37.

The adjective \textbf{propra}, own, peculiar, belonging to, is connected with its complement by the preposition \textbf{di}, which serves to indicate possession. \textbf{Esas propra dil granda anmi praktikar la pardono}, it is peculiar to great souls to practice forgiveness. \textbf{La raciono esas propra di la homo}, reason is peculiar to man (L. de Beaufront, VII, 353, 354).

39. (§ 96) The English self, ego, is translated by \textbf{ego} (plural \textbf{egoi}): \textbf{mea altra ego}, my other self; \textbf{mea propra ego}, my own self; \textbf{lua anciena ego}, his former self; \textbf{la kara ego}, one's own dear self; \textbf{lu tro amas sua ego}, he is too fond of himself (III, 146; dec. 693, 694, IV, 692). \textbf{Ego} is to be regarded as a root, it does not lose the \textbf{-o} with suffixes: \textbf{egoismo}, egoism; \textbf{egoisto}, egoist. % originally 38.

40. (§ 98 c) It is permissible to say: \textbf{il venis kun sua amiko}, he came with his friend (V, 628: ``\textbf{on darfas dicar: il venis kun sua amiko}''). The use of the reflexive pronoun in such an instance may be justified by the consideration that the sentence has only one subject which is the possessor, and the possessed object appears as a complement of the verb. When however the scene has the construction: he and his friend came only \textbf{lua} is permissible, \textbf{lu e lua amiko venis}. % originally 39.

\textbf{Sua} is used in the following sentence. \textbf{Penigiva esis la voyo, quan la soldati prenis en sua marcho}, wearisome was the way which the soldiers took in their march. For the possessed object (\textbf{marcho}) is complement to the verb the subject of which is the possessor (\textbf{soldati}). But \textbf{sua} is not to be used in the following construction: p\textbf{enigiva esis la voyo prenita da la soldati en lia marcho}, wearisome was the way taken by the soldiers in their march. For the possessor is not the subject of the sentence (V, 627, 628).

\addcontentsline{toc}{subsection}{Adverb \textbf{jus}, note 41}
41. (§ 105) \textbf{Jus} refers to the moment passed a little while ago. \textbf{Tu venas tro tarde, lu jus mortis}, you come too late, he just died. With other meanings of the English just, as when it denotes identity of time, place, and circumstances, other adverbs are to be used, as \textbf{precize}, \textbf{exakte}, \textbf{akurate}, \textbf{apene} (III, 486, 487). \textbf{E ta esas exakte la speco di vivo por qua il esas formacita}, and this is just the sort of life he is formed for (Bulwer). \textbf{Fore! Tu ne plus salvas l'amiko, la morton lu subisas precize nun}, away! thou savest the friend no more, death he is just undergoing (Schiller). \textbf{Tre plata ol }(\textbf{la tero})\textbf{ aspektis \ldots apene exter la aquo e ne pluse}, very flat it (the land) looked—just out of water and no more (Jackson). % originally 40.

42. (§ 105) \textbf{Tam \ldots kam posible}, as much as possible. The expression as well \ldots as, or both \ldots and, may be translated by \textbf{tam \ldots kam}: \textbf{tam la patro kam la filio esas stulti}, both the father and the son are fools (II, 410; see also § 128). Sometimes \textbf{kune} is clearer than \textbf{tam \ldots kam}. \textbf{El esis kune bela e bona}, she was both beautiful and good. \textbf{Tam \ldots kam} would be rather equivocal in such an instance (III, 289). % originally 41.

\addcontentsline{toc}{subsection}{\textbf{Ankore}, \textbf{ja}, \textbf{ne plus}, note 43}
43. (§ 106) \textbf{Ankore} and \textbf{ja} must be carefully distinguished. The difference given in Gramm. Compl. (§ 38: \textbf{ankore} = until now, until then; \textbf{ja} from now on, from then on) is not sufficiently clear. Simpler and clearer is the following definition (II, 141). \textbf{Ankore} refers to a fact or act existing already, \textbf{ja} to the beginning of a fact or act earlier than expected. With the negation there must be the same difference. \textbf{Ne ankore} corresponds, therefore, to no longer, \textbf{ne ja} often to not yet. \textbf{Kad vu ankore hungras}, or \textbf{hungras ankore?} \textbf{No, me ne ankore hungras, ne hungras ankore, ankore ne hungras}, are you still hungry? No, I am no longer (no more) hungry. \textbf{Kad vu ja hungras?} \textbf{No, me ne ja hungras}, are you hungry already? No, I am not yet hungry. % originally 42.

With inchoative verbs such as \textbf{arivar}, to arrive, \textbf{komencar}, to begin, \textbf{dormeskar}, to fall asleep, etc., even this definition is not sufficient and therefore the proper application of \textbf{ankore} and \textbf{ja} quite difficult and one may remain in doubt which of the two particles to use, especially when the verb is negative. \textbf{Lu ankore arivis}, he arrived yet; \textbf{lu ne ankore arivis}, \textbf{ne arivis ankore}, a\textbf{nkore ne arivis}, he did not arrive yet. \textbf{Lu ne ja arivis} may also mean he did not arrive yet. \textbf{Lu ja dormeskis}, he fell asleep already; \textbf{lu ne ja dormeskis}, he had not fallen asleep already.

To make a difference between \textbf{ne ankore} and \textbf{ankore ne} (III, 335) is far-fetched. \textbf{Ja ne} is sometimes as much as not even. \textbf{Il ja ne esas kontenta, quon il dicos, kande \ldots}, he is not even satisfied, what will he say when \ldots (Gramm. Compl., p. 25). It would appear that \textbf{ne ja} could just as well be used in this sentence.

\textbf{Ne plus} is almost synonymous to \textbf{ne ankore}. \textbf{Me ne plus hungras}, I am no more hungry. In the future tense only one is applicable in a given instance. \textbf{Me ne plus pagos}, I will pay no more; but \textbf{me ne pagos ankore}, I will not pay yet. \textbf{Ne plus lu audez tua voco, ne plus vekez per tua voko}, no more shall he hear thy voice, no more awake at thy call (Macpherson).

\textbf{No, nule}, is not at all; \textbf{me anke ne}, neither I, I also not; \textbf{ne nur \ldots ma anke}, not only \ldots but also; \textbf{ne min}(\textbf{e}), not less.

\addcontentsline{toc}{subsection}{So much the better; how much; together with; rather, notes 44-47}
44. (§ 115, III) The adverbial phrases: so much the better, so much the worse (French: tante mieux, tante pis; German: desto besser, desto schlimmer) which are international idioms, are best translated by: \textbf{to esas }(\textbf{esos})\textbf{ bona}, \textbf{me }(\textbf{ni})\textbf{ joyas }(joyos\textbf{}); \textbf{to esas }(\textbf{esos})\textbf{ nebona}, \textbf{mala}, \textbf{me }(\textbf{ni})\textbf{ regretas }(\textbf{regretos}). For instance: If he is rich, so much the better for him, \textbf{se il esas richa, to esas bona por il}, or \textbf{me joyas por il}. If he does not come, so much the worse, \textbf{se il ne venos, to esos nebona} or \textbf{me regretas to} (L. de Beaufront, IV, 415, 416). % original

45. (§ 107, 111) Much, how much, so much, little, a little with succeeding substantive is better translated with an adjective than with an adverbial form: \textbf{poka karno}, little meat; \textbf{kelka pano}, little bread; \textbf{multa vino}, much wine; \textbf{quanta pekunio}, how much money (better than \textbf{poke karno}, \textbf{kelke pano}, etc.). When a substantive does not follow, a substantive form of these determinatives is preferable to an adjective form. \textbf{L'unu recevas kelko ed esis kontenta, a l'altru on donis poko e lu plendis}, one received a little and was satisfied, to the other they gave little and he complained. An adverbial form is to be used when these words are real adverbs. \textbf{Me esas kelke fatigita}, I am a little tired. \textbf{Lu parolis poke, pro ke lu esis kelke rauka}, he spoke little because he was somewhat (a little) hoarse (III, 23, 593). % originally 44.

46. (§ 107, IV) The combination ``together with'' is not to be expressed by \textbf{kune kun}, which is as little admissible as \textbf{pose pos}. Usually the word together is merely a pleonasm in the combination. \textbf{Kun} alone is, therefore, sufficient to translate the latter. Where this would not appear to be the case, \textbf{samtempe kam}, simultaneously with, \textbf{unione kun}, in union with, may be used, but never \textbf{kune kun}. \textbf{La infanti dormas kune en un lito}, the children sleep together in one bed. \textbf{La matro e la filiino arivis kune}, the mother and the daughter arrived together. \textbf{La matro arivis kun la filiino}, the mother arrived together with the daughter. \textbf{La matro arivis e la filiino pose}, the mother arrived and the daughter afterwards. \textbf{La filipino arivis pos la matro}, the daughter arrived after the mother (\textbf{pose pos} would be impossible). \textbf{La tirano mortigis la filii unione kun lia patri}, the tyrant killed the children together with their parents. \textbf{La mastro desaparis samtempe kam lua }(\textbf{sua})\textbf{ servisto}, the master disappeared together with his servant (II, 410). % originally 45.

47. (§ 107, IV) The English rather is to be translated by \textbf{prefere} only when there is a real preference, otherwise by \textbf{plu bone}, \textbf{plu juste}, \textbf{plu vere}, \textbf{kelke}, etc. Venez prefere morge kam cadie vespere, come rather to-morrow morning than this evening. \textbf{Ni nomizas nia afliktesi Destino, ma devus plu juste nomizar nia granda sucesi tale}, we call our sorrows Destiny, but ought rather to name our high successes so (Lowell). \textbf{Lu dicis adio kelke haste}, he said good-bye rather hurriedly (Molesworth) (II, 669). % originally 46.

\addcontentsline{toc}{subsection}{Transitive verbs without object; intransitive verb with direct object; invariability of the direct object, notes 48-51}
48. (§ 172) Some transitive verbs appear frequently to be intransitive and thus resemble the mixed verbs merely because the object is often not expressed. Such verbs are: \textbf{drinkar}, to drink; \textbf{glutar}, to swallow; \textbf{lektar}, to read; \textbf{manjar}, to eat; \textbf{respondar}, to answer; \textbf{skribar}, to write, etc. These verbs (they may be called transitive verbs without object or transitive verbs used intransitively) differ from the mixed verbs in that an object can always be added while with mixed verbs used intransitively there is no object to be added. In \textbf{manjar rapide ne esas bona por le digesto}, to eat hastily is not good for the digestion, the object \textbf{repasto}, meal, may be added, but in \textbf{la mala vetero balde cesos}, the bad weather will soon cease, there is no object to be added to the verb (see note 22). % originally 47.

49. (§ 112) In rare instances an intransitive verb may assume a direct object and even admit of the formation of the passive voice. This is the case when the action expressed by the verb is taken as object. \textbf{Li iris penigiva voyo}, they went a wearisome way. \textbf{Lu kombatis senespera kombato}, he fought a hopeless fight. \textbf{El vivas mizeroza vivo}, she lives a life full of misery. \textbf{La infanto dormas dolca dormo}, the child sleeps a sweet sleep. \textbf{Parkombatita, parluktita esas la longa grava lukto}, fought out, finished is the long, hard fight (Schiller, das Siegesfest). % originally 48.

50. (§ 112) It has been proposed (II, 726; VI, 490, 607; VII, 102, 297, 351) to regard sometimes an intransitive verb as transitive and the attribute as the direct object of the verb in order to be able to distinguish the attribute from the the subject by using for the former the accusative form, especially in questions. \textbf{Quon divenas ulo}, what becomes something? The writer regards this construction as a real solecism for which there is no necessity. He maintains that ordinarily the attribute is sufficiently distinguished from the subject by the context, and where it is not, it can easily be made so by simple means without resorting to a strange solecism. These contentions are proved conclusively in an essay of his which will appear at some future time. % originally 49.

51. (§ 113) Because of the principle that a verb cannot have two objects, the vehicle which is loaded with goods and the goods which are loaded on the vehicle cannot be direct object to the same verb. Two different verbs have therefore been adopted for the two cases, \textbf{charjar} for the first and \textbf{kargar} for the second one: \textbf{charjar veturo per vari}, to load a vehicle with goods; \textbf{kargar vari sur veturo}, to load goods on a vehicle (dec. 723, V. 66; III, 38, IV, 106). % originally 50.

In the instances just cited there is a different verb for each of the two objects. In other instances the verb for one object differs from that for the other object only through the suffix \textbf{-iz}. For example, the verb \textbf{injektar} has as object the fluid injected into, and the verb \textbf{injectizer} has as object the person or thing injected with a fluid: \textbf{injektar fluido koloroza en la veini}, to inject a colored fluid into the veins; \textbf{injektizar la veini per fluido koloroza}, to inject the veins with a colored fluid. Another example is: \textbf{notar remarki en libro}, to note remarks in a book; \textbf{notizar libro per remarki}, to note (annotate) a book with remarks. Again another example is: \textbf{respondar kurajiganta paroli}, to answer encouraging words (to give encouraging words as an answer); \textbf{respondizar kurajiganta paroli} (= \textbf{respondar a kurajiganta paroli}), to answer encouraging words, i.e., to give an answer to encouraging words (see note 23).

The syntactic formula for \textbf{pagar} is: \textbf{pagar ulo ad ulu po ulo}, to pay something to somebody for something: \textbf{me pagas pekunio a ko mereisto po varo}, I pay money to a merchant for ware. Some freedom of construction, however, is permissible with this verb; one may say \textbf{pagar ulo} and \textbf{pagar ulu}, to pay something and to pay somebody (dec. 717, V, 66; dec. 679, IV, 691).

\addcontentsline{toc}{subsection}{Want of the verbal character in verbal and participial nouns, notes 52-54}
52. (§ 119) A verbal noun (a substantive derived from a verb) loses its verbal character as far as the object is concerned, it cannot have a direct object as complement (see note 54). The clearest way to connect the verbal noun with the object and the subject of the verb at the same time is by using for the former the proposition \textbf{di} and for the latter the preposition \textbf{da}: \textbf{guvernado di la populo da la populo}, government of the people by the people: \textbf{la ocido di la elefanto da la leono}, the killing of the elephant by the lion. When there is only one complement, \textbf{da} will always mark the subject while \textbf{di} may indicate both the subject and the object. \textbf{La odio da la tirano}, the hate of the tyrant, can mean only: the tyrant hates, but \textbf{la odio di la tirano}, the hate of the tyrant, can mean both: the tyrant hates and the tyrant is hated. To be precise the verbal noun of the passive may be used. \textbf{L'odieso di la tirano} can mean only: the tyrant is hated. With verbs expressing sentiment the preposition a marks the object of the sentiment, and the subject of the sentiment may be joined to the verbal noun by the preposition \textbf{di}: \textbf{la amo di la filii a la patri} (= \textbf{la amo di la patri da la filii}), the love of the children for the parents. In any case the most accurate construction (which, however, is generally not necessary) is the verbal noun of the passive connected with the object by \textbf{di} and with the subject by \textbf{da}: \textbf{la envidieso di la richi da la povri}, the envy of the poor towards the rich (II, 401, 480; Gramm. Compl., p. 29). % originally 51.

53. (§ 120) It is advisable not to replace classes by participles accompanied by long compliments. For thereby long sentences would be created which are less conducive to clearness than short sentences. In every language the clearest style is the best one, and the first rule to obtain it is: do not make long sentences (``\textbf{ne facez longa frazi},'' IV, 718). % originally 52.

54. (§ 122) A participial substantive the same as any other substantive formed from a verb (for instance by the suffix \textbf{-er}, \textbf{-ist}, etc.) loses its verbal character, i.e., can have no direct object (see note 52), but only a complement with a preposition: \textbf{la konstruktinto di la ponto}, the builder of (the one who has built) the bridge (\textbf{la konstruktinto di la ponto} is not permissible); \textbf{la skribero di la letro}, the writer of the letter (VII, 159). % originally 53.

The loss of the verbal character may not be apparent with participial (and other verbal) substantives derived from intransitive (and also mixed verbs, §112, note 48) verbs which have a prepositional expression as complement. For the same preposition which is used with the verb will, as rule, be admissible also with the verbal substantive: \textbf{la migrero tra la dezerto}, the wanderer through the desert; \textbf{la sufreri pro tuberkloso}, the sufferer from tuberculosis; \textbf{la kombatinti por la patrio}, those who have fought for the fatherland, \textbf{la drinkinti ek ta fonto}, those who drank from that fountain. In an example such as the last one it my perhaps be more advisable not to employ a verbal substantive, but to use another construction: \textbf{ti qui drankas ek ta fonto}.

(§§ 37, 38, 121) The participle of the future in the active as well as in the passive combined with the verb \textbf{esar} translates appropriately the English phrases `to be about',  `to be going to', `to be on the point of'. \textbf{Ni esas departonta}, we are on the point of leaving. \textbf{Me esas promocota}, I am going to just graduated. \textbf{Il esis cedonta}, he was about to yield. \textbf{Li esis kaptota}, they were at the point of being captured.

The passive participle of the present in combination with the verb \textbf{esar} serves to form the passive voice pure and simple and can therefore not be used in this combination to form secondary tenses (temps secondaires, Gramm. Compl., end of  §30). By analogy the active participle of the present, too, would appear to be inappropriate for this purpose. The active participle of the present will rarely be required in combination with \textbf{esar}. The only secondary tense it could thus form would be one expressing continuation or prolongation of an action and for this purpose we have the more expressive suffix \textbf{-ad}. \textbf{Mea kanario esas kantanta, dum ke me skribas ico} = \textbf{mea kanario kantadas, dum ke me skribas ico}, my canary is singing while I write this. The second form is preferable to the first one. It follows that the English form `I am reading' is not to be translated word for word in Ido by `\textbf{me esas lektanta}.' Ordinarily `\textbf{me lektas}' is sufficient. When, however, continuation or prolongation of the action is to be indicated, `\textbf{me lektadas}' is more expressive than `\textbf{me esas lektanta}.' The latter form may be used with advantage to imitate the English mode of expression the more so as it is perfectly intelligible even for one unfamiliar with the English language (see principle of translation, p. 17).

\addcontentsline{toc}{subsection}{Reciprocity, note 55}
55. (§ 124) Good writers express reciprocity by combining the preposition inter with transitive verbs (in the same manner as with intransitive verbs) without adding an object: \textbf{li interkomprenas}, they understand each other. Some objection may be raised to such a construction. A sentence with a transitive verb and without an object gives the impression of being incomplete. The view that \textbf{inter} replaces or includes an object may be questioned. The official grammar (§ 34, 16) states ``in certain cases'' a verb may be combined with inter to express reciprocity, but gives no example of a transitive verb being thus employed. It would, therefore, be preferable to express reciprocity with transitive verbs in the usual manner by \textbf{l'unu l'altru}: \textbf{li komprenas l'unu l'altru}, they understand each other. This construction is correct at any rate and certainly clearer than \textbf{li interkomprenas}. % originally 54.

\addcontentsline{toc}{subsection}{Hours of the day; some numeral series; enunciation of large numbers; international measures of length, notes 56-59}
56. (§ 126) According to the custom of several countries (Belgium, France, Italy) the hours of the afternoon are counted from 12 till 24: \textbf{pos duadek kloki}, after 8 o'clock in the evening. % originally 55.

\textbf{Qua kloko esas?} is not correct to translate `what time is it?' (VII, 399).

57. (§ 128) The series last (ultimate), last but one (penultimate), last but two (antepenultimate), last but three, last but four, etc., is \textbf{lasta}, \textbf{duesma lasta}, \textbf{triestma lasta}, \textbf{quaresma lasta}, etc. The second and third terms of the series may also be \textbf{prelasta}, \textbf{anteprelasta} respectively. (III, 217). % originally 56.

\textbf{Omna duesma dio}, every other day, means the first day of each pair of several succeeding pairs of days, i.e., the 1st, 3rd, 5th, 7th, etc.; \textbf{omna triesma dio}, every third day, i.e., the first day of every three days, the 1st, 4th, 7th, 10th, etc. (IV, 235; V, 680).

58. (§ 138) Large numbers may be enunciated in the same manner as decimal fractions, i.e., by naming each digit for itself and omitting million, thousand, hundred, ten: 3276894501, \textbf{triamil e duacent e sepadek e sisa milion e }(\textbf{,})\textbf{ okacent e nonadek e quaramil e }(\textbf{,})\textbf{ kinacent e un}, may be enunciated thus: \textbf{tri, du, sep, ok, non, quar, kin, zero, un} (VII, 39). % originally 57.

59. (§ 132) The international measures of length and their signs are: m = \textbf{metro}, meter; km = \textbf{kilometro}, kilometer; cm = \textbf{centimetro}, centimeter; mm = \textbf{milimetro}, millimeter. Their squares and cubes are designated and named as follows: m\textsuperscript{2} = \textbf{metro-quadrato}, square meter; m\textsuperscript{3} = \textbf{metro-kubo}, cubic meter; km\textsuperscript{2} = \textbf{kilometro-quadrato}, square kilometer; km\textsuperscript{3} = \textbf{kilometro-kubo}, cubic kilometer; cm\textsuperscript{2} = \textbf{centimetro-quadrato}, square centimeter; etc. % originally 58.

The term: an area of 2 square meters is to be distinguished from: an area of 2 meters square, i.e., of 4 square meters. The first term is in Ido: \textbf{areo de du metro-quadrati}. The second term could be translated in Ido by: \textbf{areo de du metri quadrate}, but clearer is \textbf{areo de quar metro-quadrati} and this construction is therefore preferable, though the first one corresponds with the English manner of expression (IV, 608; VI, 52, 141).

\addcontentsline{toc}{subsection}{Preposition \textbf{ad} with verbs expression motion; \textbf{de}; \textbf{lor}, notes 60-62}
60. (§ 133) \textbf{Ad} is prefixed to other prepositions to indicate motion into: \textbf{aden}, \textbf{addop}, \textbf{adsub}, \textbf{adsur}. \textbf{La mikra ucelo flugis aden la granda kajo}, the little bird flew into the big cage. \textbf{En la kajo} may mean the bird was flying about in the cage. \textbf{La navo natas adsub la ponto}, the ship swimps to the place under the bridge. \textbf{Sub la ponto} would mean the ship is under the bridge and swims there. % originally 59.

This use of \textbf{ad} is, however, to be restricted to cases where it is really necessary to avoid ambiguities. Whenever motion into is sufficiently indicated by other words, notably the verb, \textbf{ad} is not to be added to the preposition. For the principle of simplicity (economy) requires that the same idea should not be expressed twice. The sentence, \textbf{Polikrates jetis sua juvelo maxim kara en la maro}, P. threw his most cherished jewel into the sea, is entirely clear. It is, therefore, superfluous to join \textbf{ad} to \textbf{en} in such a sentence (III, 226; VI, 341).

61. (§ 133) \textbf{De} is prefixed to other prepositions to indicate motion from. \textbf{Adportez la stuleto desub la sofao e metez ol apud mea skribotablo}, bring the little stool from under the sofa and put it at my desk. \textbf{La bruiso semblas venar dedop ta muro}, the noise seems to come from behind that wall. % originally 60.

62. (§ 133) Since the expression ``at the time of the time'' is not admissible, the preposition \textbf{lor} cannot indicate a definite hour, day, month, year (VI, 213, dec. 1115). In such instances the prepositions \textbf{ye}, \textbf{en}, and \textbf{dum} are used, the first one for a precise time, the other two with reference to an interval of time: \textbf{ye quar kloki e duimo}, at half past four o'clock; \textbf{ye la nonesma }(\textbf{di})\textbf{ julio}, on the 9th of July; \textbf{en la vintro}, in the winter; \textbf{dum la unesma yaro di lua administrado}, during the first year of his administration. % originally 61.

\textbf{Lor} is applicable only to definite events: \textbf{lor la su-liberigo di la kolonii di Nordamerika}, at the time of the liberation of the colonies of North America. \textbf{Lor la morto di mea patrulo me esis danjeroze malada}, at the time of the death of my father I was dangerously ill. \textbf{Dum} brings out distinctly the duration of an event. \textbf{Dum la Franca revoluciono la vivo di nulu esis sekura en Paris}, during the French revolution nobody's life was safe in Paris (V, 28). "

``At the time of Perikles'' is \textbf{ye la tempo di Pericles}, for a person is not an event.

\addcontentsline{toc}{subsection}{Adverbs and adjectives \textbf{infre}, \textbf{infra}, \textbf{supre}, \textbf{supra}, etc., note 63}
63. (§ 136) The following adjectives and adverbs require a closer definition: \textbf{alta}, \textbf{basa}, \textbf{infra}, \textbf{infre}, \textbf{suba}, \textbf{sube}, \textbf{supera}, \textbf{supere}, \textbf{supra}, \textbf{supre}, \textbf{sura}, \textbf{sure}, \textbf{inferiora}, \textbf{superiora}. \textbf{Alta}, high, and its opposite \textbf{basa}, low, express dimension. \textbf{Suba}, lower, and its opposite \textbf{supera}, upper, indicate a situation or direction relative to a certain point. \textbf{Infra}, lowermost, and its opposite \textbf{supra}, uppermost, refer to a fixed point which is the bottom or the top of something. Considering the third story of a building of six stories, \textbf{la supera etaji}, the upper stories, are the 4th 5th, and 6th stories. while \textbf{la supra etajo}, the uppermost story, is the sixth only. \textbf{La suba etaji}, the lower stories are the 2nd and 1st ones, but \textbf{la infra etajo}, the lowermost story is only the first one. The adverbs \textbf{supre} and \textbf{infre} mean at the top and on the bottom of something, i.e., they indicate a fixed point, while the adverbs \textbf{supere}, above, and \textbf{sube}, below, express a relation to a certain point. When contact with the latter is to be brought out, the adjective \textbf{sura}, upper, and the adverb \textbf{sure}, above, are preferable to \textbf{supera} and \textbf{supere}, i.e., \textbf{sura} (\textbf{sure}) differs from \textbf{supera} (\textbf{supere}) in the same way as the preposition \textbf{sur} from \textbf{super} (see these prepositions). In the above example \textbf{la sura etajo}, the story lying above, the upper story, is the fourth one only. % originally 62.

The expressions: see above, see below in citations are to be translated by \textbf{videz supere}, \textbf{videz sube}. \textbf{Videz supre}, \textbf{videz infre} would mean: see at the top or at the bottom of the page.

The top and the bottom are \textbf{la suprajo}, \textbf{la infrajo}, when a material part of something is to be indicated, while \textbf{la supro} and \textbf{la infro} denote geometrical extremities. \textbf{La suprajo di la komodo esas marmora}, the top of the bureau is of marble. \textbf{La supro di la armoro esas horizontala}, the top of the case is horizontal.

The following sentence brings out clearly the difference between \textbf{alta}, \textbf{basa}, \textbf{supra}, \textbf{infra}. \textbf{La supra etajo esas basa e la infra etajo esas alta}, the uppermost story is low and the lowermost story is high.

\textbf{Superiora} and \textbf{inferiora} mean superior and inferior in sociological and economic sense (being above, super, and below, sub, somebody). The adjectives \textbf{supera} and \textbf{suba} are more general and are used in relation to space and figuratively (II, 163, 664; III, 228, 411).

\addcontentsline{toc}{subsection}{Conjunctions \textbf{or}, \textbf{yen}, \textbf{ya}, notes 64-66}
64. (§ 137) The conjunction or is used only in reasoning to present new argument. In logic it introduces the minor premise of a syllogism. It differs from \textbf{do} and \textbf{ma} in that the former introduces conclusion and latter brings opposition. \textbf{Omno foligas esas veneno; or l'alkoholo foligas, esas veneno. Ma l'alkoholo fortigas, nutras \ldots}, everything that befools us is poison. Now the alcohol befools, therefore it is a poison. But the alcohol strengthens, nourishes \ldots In everyday use the order is reversed, the minor premise being placed after the conclusion and the major premise being usually omitted as self-understood. \textbf{Nam} is then used instead of \textbf{or}. \textbf{L'alkoholo esas veneno, nam ol foligas}, the alcohol is poison, for it befools (VI, 131). % originally 63.

65. (§ 137) \textbf{Yen} is properly an interjection, but it is also cited the among coordinate conjunctions (Gramm. Compl., § 82). Some would even regard it as a preposition because it is ordinarily followed by a noun or pronoun: \textbf{yen la volfo!}, here is the wolf!, look, the wolf!, voilà le loup!; \textbf{yen me, yen lu}, here am, here he is, me voilà, le voilà. And it cannot be denied \textbf{yen} has a prepositional character. For this reason some have postulated that \textbf{yen ke} should used instead of \textbf{yen} alone before a sentence: \textbf{yen ke la treno arivas}, here comes the train! (VI, 604; VII, 289). Others regard this usage as a French idiom: voilà que le train arrive! They prefer \textbf{yen} alone before a sentence because \textbf{yen} is properly an interjection: \textbf{yen la treno arivas} (VII, 207) It would appear the latter view is more justified. For the sentence introduced by a preposition with \textbf{ke} is a dependent clause while the introduced by \textbf{yen} is independent. % originally 64.

66. (§ 137) \textbf{Ya} may be used for expressing emphasis both of a whole sentence and of a part of it, especially of the subject (§ 75). \textbf{Vu ne povas ekirar, pluvas ya}, you cannot go out, it (certainly) rains (c'est qu'il pleut, es regnet ja!) \textbf{Esis ya mea kulpo}, it was my (accent on my) fault. \textbf{Me ya facis to}, it was I who did it (or simply I with accent on it) (II, 675; III, 152; VII, 195, dec. 377). % originally 65.
