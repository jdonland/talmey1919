\subsection*{INTRODUCTION.}
\addcontentsline{toc}{subsection}{Introduction: Origin of the Ido vocabulary; principles of reversibility}
Ido is a language strictly a posteriori, containing no word whatsoever of a root arbitrarily selected. Its great superiority over other artificial languages constructed after the a posteriori principle lies in that it never violates this principle while those languages transgress it often, some of them, as Esperanto, in the most frequently occurring words, as pronouns, conjunctions, adverbs of time, place, and manner, etc., so that the a posteriori character of those languages is lost to a considerable extent. 

The foundation of the Ido vocabulary is furnished by those words which have long been in international use, as all technical and scientific terms and many other words. The rest of the vocabulary has been selected from ancient and modern languages according to the principle of maximum of internationality. It follows therefore that upon the whole Ido has the aspect of a Romanic language. But it contains also a large percentage of Anglo-German roots which have been found to be more international than equivalent roots of a Romanic origin. 

From the roots many words are constructed by means of affixes and by combination. Root words or original words are only those belonging to these parts of speech: article, pronoun (personal), adverb (original), numeral, preposition, conjunction, and interjection. They represent only a small part of the vocabulary while by far the largest part consists of words formed from roots. 

The words of the four principal parts of speech, verb, substantive, adjective, and adverb, are obtained by adding the endings \textbf{-ar}, \textbf{-o}, \textbf{-a}, \textbf{-e} to roots. Inversely by these characteristic grammatical endings the grammatical role of a word of these four parts of speech is at once recognized, the ending \textbf{-ar} characterizing the verb, \textbf{-o} the substantive, \textbf{-a} the adjective, and \textbf{-e} usually the adverb. 

The sum of words derived from one root by means of grammatical endings and affixes is called a word family (L. Couturat, Étude sur la Dérivation, p. 5). Only by a regular system of derivation is it possible to construct complete word families or to obtain from one root all words that may be needed. No natural language possesses regular complete word families, but some useful and convenient derivative is usually wanting. 

The roots themselves have no grammatical role as far as the four principal parts of speech are concerned. The grammatical role of a word is obtained only after the respective grammatical ending has been added to the root, making of it a verb, noun, adjective, or adverb. Yet there are certain roots which are essentially verbal, as \textbf{am}, love; \textbf{esper}, hope; \textbf{labor}, work; \textbf{dorm}, sleep; \textbf{ces}, cease. But as verbal are to be regarded also the roots of words denoting a state, as \textbf{enoyo}, boredom; \textbf{hungro}, hunger; \textbf{joyo}, joy; \textbf{paco}, peace; etc.; or of words expressing a relation, as \textbf{debo}, debt; \textbf{difero}, difference; \textbf{manko}, lack; \textbf{pezo}, weight; \textbf{valoro}, value; etc. Other roots are essentially substantive, as \textbf{dom}, house; \textbf{tabl}, table; \textbf{aqu}, water. Again others are essentially adjective, as \textbf{bon}, good; \textbf{bel}, beautiful; \textbf{mikr}, small; \textbf{grand}, big. Finally some roots are essentially adverbial, as \textbf{bald}, soon; \textbf{nun}, now; \textbf{nur}, only; \textbf{sat}, enough (11). \label{parts}

The derivation of the form of a word must go hand in hand with the derivation of the meaning of the word, otherwise the word is useless and no little confusion arises, as shown by the example of Esperanto which possesses an extensive system of derivation of word forms (Pract. \& Theor. Esper., p. 34; Étude sur la Dériv.., pp. 5, 6). There must be an unequivocal correspondence between form and meaning of a derivative. This can only be obtained by strictly observing the principle of reversibility which has been propounded in a most lucid manner by Dr. L. Couturat in his excellent monograph on the derivation (Étude sur la Dérivation en Esperanto, pp. 6, 7): “Every derivation must be reversible, i.e., if one passes from one word to another of the same family by means of a certain rule, he ought to be able to pass back from the second word to the first one by means of a rule which is the exact reverse of the first rule. For instance, if the suffix \textbf{-ist} designates a person who is occupied (professionally) with the thing expressed by the root as shown by the derivatives \textbf{artisto}, artist; \textbf{muzikisto}, musician, the substantive obtained by suppressing this suffix must designate the thing (\textbf{arto}, art; \textbf{muziko}, music) with which is occupied the person indicated by the derived substantive (\textbf{artisto}, \textbf{muzikisto}). This requirement of common sense, which is an indispensable condition of the regularity of derivation, we call the “principle of reversibility” (12). \label{reversibility}

There are two modes of derivation. The immediate derivation is accomplished by adding a grammatical ending to a root or by changing one grammatical ending into another one. A noun is thus obtained from a verb by substituting \textbf{-o} for \textbf{-ar}, etc. The mediate derivation builds new words by means of affixes.

\subsection*{IMMEDIATE DERIVATION.}
\addcontentsline{toc}{subsection}{Immediate Derivation—§§ 53-58}
53. Verb and Substantive. A verbal root furnishes a verb immediately. The substantive derived immediately from this verb denotes the action, state, or relation expressed by the verb: \textbf{pagar}, to pay; \textbf{pago}, payment; \textbf{drinkar}, to drink; \textbf{drinko}, the action of drinking; \textbf{esperar}, to hope; \textbf{espero}, hope; \textbf{iracar}, to be angry; \textbf{iraco}, wrath; \textbf{valorar}, to be worth; \textbf{valoro}, value (13). 

Inversely a verb can be derived from a noun immediately only when the latter denotes an action, state, or relation, i.e., when the root is verbal. Nouns expressing an action, as \textbf{elekto}, election; \textbf{movo}, movement; etc., or a state as \textbf{hungro}, hunger; \textbf{joyo}, joy; \textbf{paco}, peace, etc., or a relation as \textbf{debo}, debt; \textbf{difero}, difference; \textbf{pezo}, weight; etc., furnish verbs immediately: \textbf{elektar}, to elect; \textbf{movar}, to move; \textbf{hungrar}, to hunger; \textbf{joyar}, to rejoice; \textbf{pacar}, to be in peace; \textbf{debar}, to owe; \textbf{diferar}, to differ; \textbf{pezar}, to weigh. But from nouns such as \textbf{domo}, house; \textbf{tablo}, table; etc., verbs cannot be formed immediately. They would not have any sense. 

Some nouns denoting the result of an action also furnish verbs immediately: \textbf{parodio}, parody; \textbf{parodiar}, to parody; \textbf{profito}, profit; \textbf{profitar}, to profit; \textbf{pruino}, hoar-frost; \textbf{pruinar}, to form hoar-frost; \textbf{rabato}, discount; \textbf{rabatar}, to give a discount; \textbf{respondo}, answer; \textbf{respondar}, to answer; \textbf{testamento}, will; \textbf{testamentar}, to make a will (13). 

\small Remark. From substantives not expressing an action, state, or relation verbs cannot be derived immediately, but special affixes (\textbf{-ag}, \textbf{-es}, \textbf{-if}, \textbf{-ig}, \textbf{-iz}, see these affixes) are required for that end. One must be careful not to form verbs immediately from nouns such as the following (this error is very prevalent in Esperanto): \textbf{oro}, gold; \textbf{salo}, salt; \textbf{sapono}, soap; \textbf{armo}, weapon; \textbf{brido}, bridle; \textbf{krono}, crown; \textbf{kruco}, cross; \textbf{martelo}, hammer; \textbf{afisho}, placard; \textbf{kolonio}, colony; \textbf{floro}, flower; \textbf{jermo}, germ; \textbf{chefo}, chief; \textbf{gasto}, guest; etc., etc. \normalsize

54. Substantive and Adjective. In forming by immediate derivation an adjective from a substantive or a substantive from an adjective it is useful to keep in mind which is the primary word. If a root be essentially substantive, the adjective is the derived or secondary word and the substantive the primary one. For instance, \textbf{ora}, golden, is secondary or derived, \textbf{oro}, gold, is primary; \textbf{festa}, festive, is secondary, \textbf{festo}, festival, is primary. If the root be essentially adjective, the substantive is the secondary word and the adjective the primary one. For instance, \textbf{sajo}, a wise person, is secondary, \textbf{saja}, wise, is primary; \textbf{blindo}, a blind person (animal), is secondary, \textbf{blinda}, blind, is primary; \textbf{extremo}, extremity, is secondary, \textbf{extrema}, extreme, is primary; \textbf{rekto}, a straight line, is secondary, \textbf{rekta}, straight, is primary; \textbf{kavo}, a hollow, is secondary, \textbf{kava}, hollow, is primary. 

Many roots, however, from which adjectives and substantives may be formed immediately are neither essentially substantive nor essentially adjective, as \textbf{dezert-}, desert, deserted; \textbf{invalid-}, invalid; \textbf{Angl-}, English; \textbf{parent-}, related, relative; \textbf{laik-}, lay, layman. In such instances nothing indicates which is the primitive word, the substantive or the adjective. 

An adjective formed by immediate derivation from a substantive signifies being so and so, being that and that. a) The root is essentially substantive: \textbf{festo}, a festival; \textbf{festa dio}, a festive day, a day that is a festival; \textbf{oro}, gold; \textbf{ora vazo}, a golden vase, a vase that is (consists of) gold; \textbf{gardeno}, garden; \textbf{gardena urbo}, a gardenlike city, a city that is a garden; \textbf{proverbo}, proverb; \textbf{proverba expresuro}, a proverbial expression, an expression that is a proverb. b) The root is essentially adjective. This case needs hardly to be illustrated, for the adjective is the primitive word and its meaning, therefore, evident: \textbf{sajo}, a wise person, \textbf{saja judiciisto}, a wise judge, a judge who is a wise person; \textbf{blindo}, a blind person (animal), \textbf{blinda mendikisto}, a blind beggar; \textbf{extremo}, extremity, extreme end, \textbf{extrema chambro}, extreme room, a room that is the extreme end; \textbf{rekto}, a straight line, \textbf{rekta strado}, a straight street, a street that is (forms) a straight line; \textbf{kavo}, a hollow, \textbf{kava sfero}, a hollow sphere, a sphere that is (contains) a hollow. c) The root is neither essentially substantive nor essentially adjective: \textbf{dezerto}, a desert, \textbf{dezerta lando}, a desolate country, a country that is a desert; \textbf{invalido}, an invalid, \textbf{invalida soldato}, an invalid soldier, a soldier who is an invalid. In all three cases the definition being that and that, so and so holds good. 

A substantive formed by immediate derivation from an adjective has a similar meaning, being that and that, being so and so, being the carrier of the quality expressed by the adjective. 

a) If the adjective expresses a quality essentially human (animal), the substantive obtained denotes a person (animal): \textbf{saja}, wise, \textbf{sajo}, a sage, a wise person; \textbf{vidva}, widowed, \textbf{vidvo}, a widowed person; \textbf{katolika}, Catholic, \textbf{katoliko}, a catholic; \textbf{suverena}, sovereign, \textbf{suvereno}, a sovereign; \textbf{blinda}, blind, \textbf{blindo}, a blind person (animal); \textbf{parazita}, parasitic, \textbf{parazito}, a parasite. 

b) If the adjective expresses a quality essentially non-human, the substantive obtained denotes a thing: \textbf{acida}, sour, \textbf{acido}, an acid; \textbf{dezerta}, desolate, \textbf{dezerto}, a desert; \textbf{rekta}, straight, \textbf{rekto}, a straight line; \textbf{kava}, hollow, \textbf{kavo}, a hollow. 

c) If the adjective expresses a quality that is neither essentially human (animal) nor essentially non-human, the meaning of the substantive obtained is actually determined by the context (14). 

It is always possible to transform a given adjective into a substantive by simply changing the grammatical ending, for the condition that the substantive obtained should denote the carrier of the quality expressed by the adjective, can always be fulfilled. On the other hand, a substantive cannot always be changed into an adjective immediately. \textbf{Doma laboro}, a work that is a house, has as little sense as \textbf{gardena pordo}, a gate that is a garden (see suffix \textbf{-al}). 

55. Verb and Adjective. In general a verb cannot be derived from an adjective immediately. Only when the root is essentially verbal can an adjective be transformed into a verb by immediate derivation: \textbf{parola}, oral, verbal, \textbf{parolar}, to speak; \textbf{espera}, hopelike, \textbf{esperar}, to hope. Otherwise special suffixes are required to construct verbs from adjectives, as the suffixes \textbf{-igar}, to make, to render so and so; \textbf{esar}, to be so and so; etc. 

An adjective may be formed immediately from a verb and its meaning is obtained by passing from the verb to the immediately derived substantive and thence to the adjective: \textbf{skribar}, to write; \textbf{skribo}, writing; \textbf{skriba}, being writing, being in writing; \textbf{parolar}, to speak; \textbf{parolo}, speaking, word; \textbf{parola}, being a word, expressed in speech. Such verbal adjectives are not identical with the active or passive participle. \textbf{Skriba} differs from \textbf{skribanta} as well as from \textbf{skribata} or \textbf{skribita}. Verbal adjectives should rarely be employed. 

56. Adverb and Adjective, Substantive, and Verb. All derived adverbs correspond to adjectives (original adjectives and those derived immediately from substantives and verbs) except the adverbs formed from prepositions. They signify: in a manner expressed by the adjective: \textbf{bone}, well; \textbf{male}, badly; \textbf{blinde}, blindly. With roots essentially substantive the meaning is similar, in the manner of: \textbf{memore}, by heart; \textbf{fine}, finally; \textbf{pede}, on foot; \textbf{nokte}, at night. The meaning of an adverb derived from a verbal root is obtained by the medium of the verbal adjective: \textbf{parole}, verbally; \textbf{skribe}, in writing. 

It is simple to pass inversely from a given adverb to other parts of speech by immediate derivation when the root is adjective, substantive, verbal, or prepositional, i.e., when the adverb is a derived one. In transforming a derived adverb it is necessary to go back to the primitive word and use the latter as a basis: \textbf{bone}, well; \textbf{cadie}, to-day; \textbf{irge}, in any manner; \textbf{jorne}, in day time; fine, finally; \textbf{konseque}, consequently; \textbf{prefere}, preferably; \textbf{volunte}, willingly; \textbf{cise}, on this side; \textbf{transe}, on the other side; lore, then, etc., are to be restored to: \textbf{bona}, good; \textbf{cadia}, of to-day; \textbf{irga}, any; \textbf{jorno}, day; \textbf{fino}, end; \textbf{konsequar}, to follow; \textbf{preferar}, to prefer; \textbf{voluntar}, to be willing; \textbf{cis}, this side of; \textbf{trans}, the other side of; \textbf{lor}, at the time of; etc. Any part of speech derivable from these bases may then be formed. 

In transforming an original adverb it is almost the adjective alone that comes into consideration, and only then an adjective is derivable, if it can signify of this and this manner; \textbf{balde}, soon; \textbf{balda respondo}, an early reply; \textbf{quaze}, so to speak; \textbf{quaza diftongo}, a sort of diphthong, a quasi-diphthong; \textbf{nun}, now; \textbf{nuna evento}, a present event. Two cases are to be considered in the way of transforming original adverbs into adjectives. 

a) The adverbs ending in \textbf{-e} change this into \textbf{-a}; \textbf{balde}, soon; \textbf{balda}, early; \textbf{hiere}, yesterday; \textbf{hiera}, of yesterday; \textbf{infre}, below; \textbf{infra}, lowermost; \textbf{pasable}, passably; \textbf{pasabla}, passable; \textbf{quaze}, so to speak; \textbf{quaza}, quasi-; \textbf{sempre}, always; \textbf{sempra}, perpetual; \textbf{supre}, on top; supra, uppermost; etc. (V, 555, 556). 

\small Remark. The adverbs \textbf{hike}, here, and \textbf{ibe}, there, form the adjectives \textbf{hikala} and \textbf{ibala} as synonyms of \textbf{hika} and \textbf{iba} (see suffix -al; IV, 562, dec. 598). \normalsize

b) The original adverbs not ending in -e form adjectives by simple addition of \textbf{-a}: \textbf{forsan}, perhaps; \textbf{forsana}, possible; \textbf{jus}, just; \textbf{jusa}, of a moment ago; \textbf{nun}, now; \textbf{nuna}, present; \textbf{nur}, only; \textbf{nura}, only; \textbf{olim}, once upon a time; \textbf{olima}, former, of yore; \textbf{plu}, more; \textbf{plua}, of greater amount or degree (not to be confounded with \textbf{plusa}, additional, from the preposition \textbf{plus}, see § 105); \textbf{quik}, at once; \textbf{quika}, immediate, prompt; \textbf{sat}, enough; \textbf{sata} (\textbf{sat multa}), sufficient. With some of this class of adverbs it is preferable to form the corresponding adjective by means of \textbf{multa}: \textbf{tro multa}, too much, is preferable to \textbf{troa}; \textbf{tre multa}, very much, preferable to \textbf{trea}; \textbf{maxim multa}, most, to \textbf{maxima}; \textbf{minim multa}, least, to \textbf{minima}. The adverbs \textbf{ja}, already; \textbf{mem}, even; \textbf{ne}, not; \textbf{no}, no; \textbf{ya}, indeed; \textbf{yes}, yes, do not lend themselves to the formation of adjectives. 

A verb may be formed from a given adverb only when the root is verbal. Otherwise adverbs, the same as other particles, do not furnish verbs immediately. For as shown before verbs can be derived immediately only from verbal roots and roots of particles are evidently not verbal. 

57. Preposition and Adverb and Adjective. Many prepositions furnish adverbs and adjectives, the transformation being accomplished simply by adding \textbf{-e} or \textbf{-a} respectively even in those prepositions that end in \textbf{-e} (the meanings of the prepositions are to be found in § 133, those of the adverbs and adjectives follow easily): \textbf{ante}, \textbf{antee}, \textbf{antea}; \textbf{apud}, \textbf{apude}, \textbf{apuda}; \textbf{cirkum}, \textbf{cirkume}, \textbf{cirkuma}; \textbf{cis}, \textbf{cise}, \textbf{cisa}; \textbf{dop}, \textbf{dope}, \textbf{dopa}; \textbf{dum}, \textbf{dume}, \textbf{duma}; \textbf{for}, \textbf{fore}, \textbf{fora}; \textbf{kontre}, \textbf{kontree}, \textbf{kontrea}; \textbf{lor}, \textbf{lore}, \textbf{lora}; \textbf{pos}, \textbf{pose}, \textbf{posa}; \textbf{proxim}, \textbf{proxime}, \textbf{proxima}; \textbf{trans}, \textbf{transe}, \textbf{transa}; etc. 

Not all prepositions lend themselves to the formation of adverbs and adjectives, as \textbf{da}, \textbf{di}, \textbf{pro}, \textbf{til}, etc. The adverb corresponding to \textbf{en} is \textbf{interne}, to \textbf{ek} \textbf{extere}, to \textbf{per} \textbf{mediate}, to \textbf{malgre} \textbf{tamen}, to \textbf{ultre} \textbf{pluse} (adjective: \textbf{plusa}). 

58. Examples of immediately formed word families of verb, substantive, adjective, and adverb are: \textbf{parolar}, to speak; \textbf{parolo}, speaking, word; \textbf{parola}, verbal, oral, in speech; \textbf{parole}, orally; \textbf{promisar}, to promise; \textbf{promiso}, promise; \textbf{promisa}, which is a promise; promise, by way of promise; \textbf{respondar}, to answer; \textbf{respondo}, answer; \textbf{responda}, which is an answer; \textbf{responde}, by way of answer; \textbf{skribar}, to write; \textbf{skribo}, writing; \textbf{skriba}, in writing; \textbf{skribe}, in writing. Such complete word families are rare. From the preceding paragraphs it may be gathered that they can be obtained only from verbal roots. 
\subsection*{MEDIATE DERIVATION.}
\addcontentsline{toc}{subsection}{Mediate Derivation—§§ 59-68}
The mediate derivation is accomplished by means of affixes which are either prefixes or suffixes. The former precede the root, the latter succeed it being inserted between the root and the grammatical ending.
\subsection*{PREFIXES.}
59. Prefixes to derive especially substantives are: 

\textbf{Anti-}, against, opposite to, is a prefix used only in technical and scientific terms. In ordinary language the preposition \textbf{kontre} is prefixed instead of, and in the same sense as, \textbf{anti-}: \textbf{antipolo}, \textbf{antipole}; \textbf{antiprismo}, \textbf{antiprism}. 

\textbf{Arki-} denotes an eminent degree, arch-: \textbf{arkianjelo}, archangel; \textbf{arkiduko}, archduke; \textbf{arkifripono}, arch villain; \textbf{arkifurtisto}, incorrigible thief. 

\textbf{Bo-} denotes relationship by marriage: \textbf{bopatro}, parent in-law; \textbf{bofrato}, sister-in-law, brother-in-law. 

\textbf{Ex-} signifies out of office, former, late, ex-. It is prefixed to words denoting office or condition: \textbf{exprezidanto}, ex-president; \textbf{exoficiro}, former officer. 

\textbf{Gala-} is a prefix qualifying holidays, dinners, representations, etc.: \textbf{galadio}, galaday; \textbf{galavesto}, galadress. 

\textbf{Ge-} denotes persons of different sex taken together: \textbf{gepatri}, parents (father and mother); \textbf{gespozi}, husband and wife; \textbf{gesiori} N., Mr. and Mrs. N. (15). 

\textbf{Para-} denotes protection against, warding off: \textbf{parafulmino}, lightning arrester; \textbf{paralumo}, lamp shade; \textbf{parafairo}, fire guard; \textbf{parapluvo}, umbrella; \textbf{paravento}, wind shield; \textbf{parafalo}, parachute. 

60. Prefixes to derive especially verbs are the following:

\textbf{Dis-} denotes separation, dispersion: \textbf{disdonar}, to distribute; \textbf{dissekar}; to dissect; \textbf{dissendar}, to distribute by sending. 

\textbf{Par-} denotes perfection or thorough completion of an action; \textbf{parlernar}, to learn thoroughly; \textbf{parlektar}, to read through; \textbf{parlaborar}, to work thoroughly; \textbf{parplugar}, to plough thoroughly. 

\textbf{Retro-} denotes the inverse action, back; \textbf{retrosendar}, to send back, to return; \textbf{retrovenar}, to come back; \textbf{retrocedar}, to recede. \textbf{Retro} is also an independent adverb meaning backwards. 

\textbf{Ri-} denotes repetition of an action, again: \textbf{rivenar}, to come again; \textbf{risendar}, to send again; \textbf{ridicar}, to say again; \textbf{rividar}, to see again. The independent adverb corresponding to the prefix \textbf{ri-} is \textbf{itere}, again, once more. 

61. Prefixes to form words of every kind (verb, substantive, adjective, adverb) are the following: 

\textbf{Des-} denotes the contrary, opposite: \textbf{deshonoro}, dishonor; \textbf{desdensa}, diffuse, loose; \textbf{deskonsilar}, to dissuade (16). 

The contrary is to be distinguished from the simple negation which is formed with \textbf{ne-}: \textbf{utila}, useful; \textbf{neutila}, useless; \textbf{desutila}, hurtful. 

\textbf{Mi-} means partly, half: \textbf{mihoro}, half an hour; \textbf{miapertita}, half open, partly open; \textbf{okuli miklozita}, eyes half closed; \textbf{mimortinta}, half dead; \textbf{mifacita}, half done. \textbf{Mi-} denotes also relationship resulting from a second marriage: \textbf{mifratulo}, half-brother; \textbf{mifratino}, half-sister (having one parent in common). 

\textbf{Stifa-} also denotes relationship resulting from a second marriage, but there is no parent in common; it corresponds to the English step- (German stief-): \textbf{stifa-fratulo}, step brother; \textbf{stifa-fratino}, stepsister; \textbf{stifa-filio}, stepchild; \textbf{stifa-patrulo}, stepfather; \textbf{stifa-patrino}, stepmother. 

\textbf{Mis-} denotes erroneous, wrong action: \textbf{miskomprenar}, to misunderstand; \textbf{misuzo}, abuse (different from \textbf{trouzo}, excessive use); \textbf{misaudar}, to hear wrongly. 

\textbf{Pre-} generally replaces the prepositions \textbf{ante} and \textbf{avan} in composed words; \textbf{prematura}, premature; \textbf{previdar}, to foresee; \textbf{predicar}, to foretell; \textbf{preavo}, great grandparent (\textbf{posnepoto}, great grandchild). 

\textbf{Mono-}, \textbf{bi-}, \textbf{quadri-}, \textbf{quinqua-}, \textbf{sexa-}, \textbf{septua-}, \textbf{okta-}, \textbf{nona-}, are prefixes used principally in technical terms. 

\subsection*{SUFFIXES.}
62. Suffixes to form especially substantives are: 

\textbf{-aj} with a non-verbal root denotes something made of, consisting of or possessing a certain quality; \textbf{lignajo}, woodwork; \textbf{lanajo}, woolen stuff; \textbf{molajo}, something soft; \textbf{stupidajo}, nonsense. 

Sometimes it means an act of: \textbf{amikajo}, an act of friendship; \textbf{infantajo}, a childish act. 

With verbal roots a difference is to be made between transitive and intransitive verbs. With the latter it refers to the subject so that \textbf{-aj} equals \textbf{-antaj}: \textbf{reptajo}, something that creeps, reptile (\textbf{reptero} is preferable to \textbf{reptajo}); \textbf{existajo}, something that exists. With transitive and mixed verbs it has a passive sense, referring to the object and being identical with \textbf{-ataj} or \textbf{-itaj}: \textbf{chanjajo}, something changed; \textbf{plantajo}, something planted, a plant; \textbf{manjajo}, something eaten, a meal; \textbf{drinkajo}, a drink. 

\textbf{-an} denotes an individual pertaining to a class (city, community, country, etc.), member of: \textbf{senatano}, a senator; \textbf{societano}, member of a society; \textbf{Parisano}, a Parisian; \textbf{Kanadano}, a Canadian. 

\textbf{-ar} denotes collection, sum: \textbf{navaro}, a fleet; \textbf{vortaro}, vocabulary; \textbf{homaro}, humanity. 

Ambiguities should be guarded against in using this suffix. In general it denotes the most extensive collection. \textbf{Homaro} means humanity, not a society of men; \textbf{vortaro} is the sum of the words of a language, not a group of words; \textbf{navaro} is the fleet of a country, not a squadron of ships; \textbf{militistaro} is the army of a country, not an army, still less a detachment of soldiers. In all such instances a proper term is to be used. 

In some instances the meaning of the word formed remains rather vague. Thus \textbf{arboraro} may vary from a grove, a group of trees, to a forest. It is therefore preferable to use special terms in such instances: \textbf{foresto}, a forest; \textbf{bosko}, a wood; \textbf{bosketo}, a spinney. 

\textbf{-ari} denotes a person who passively takes part in some act or contract, especially legal: \textbf{depozario}, depositary, trustee; \textbf{vendario}, vendee; \textbf{pagario}, payee; \textbf{testamentario}, legatee; \textbf{sendario}, addressee; \textbf{donacario}, donee; \textbf{grantario}, grantee. 

The ending \textbf{-ario} occurs in many words where it is not a suffix, but where \textbf{-ari} belongs to the root of the word, as \textbf{aquario}, \textbf{glosario}, \textbf{granario}, etc. (III, 485). 

\textbf{-ed} denotes the quantity that fills something (III, 322, dec. 54) or the quantity determined by (IV, 591). The latter part of this definition justifies its use with verbal roots (see also VI, 596 below, L. Couturat): \textbf{bushedo}, a mouthful; \textbf{manuedo}, a handful; \textbf{glasedo}, a glassful; \textbf{glutedo}, a swallow (the quantity determined by the act of swallowing); \textbf{pinchedo}, a pinch (the quantity determined by the act of pinching or of grasping with two finger tips). 

\textbf{-er} means being occupied with something, to follow some occupation habitually, but not professionally (IV, 562, dec. 591): \textbf{dansero}, dancer; \textbf{skribero}, writer; \textbf{fotografero}, one who photographs. Differently from the participle it denotes the action done not precisely at the present time (IV, 692, dec. 691): \textbf{suskriptero}, a subscriber; \textbf{inventero}, an inventor. 

The suffix \textbf{-er} is applicable also to animals characterized by a certain habitual action (L. Couturat, IV, 566): \textbf{klimero}, climber; \textbf{reptero}, reptile; \textbf{rodero}, rodent; \textbf{ruminero}, ruminant. It may even designate some things, as instruments: \textbf{flotacero}, float; perhaps even \textbf{krozero}, cruiser, and \textbf{remorkero}, tug boat (see suffix \textbf{-ist} which is applicable only to persons). 

\textbf{-eri} denotes establishment, especially industrial: \textbf{lakterio}, dairy; \textbf{imprimerio}, printing office; \textbf{balnerio}, bathing establishment; \textbf{binderio}, book binding factory; \textbf{kafeerio}, cafe. With substantive roots it signifies the establishment where one is occupied in any manner with the thing indicated by the root without the thing being necessarily manufactured in that place: \textbf{librerio}, library; \textbf{papererio}, paper establishment (VI, 212, dec. 1091). If it be desirable to indicate what is being done with the thing, a special verbal part must be added; thus \textbf{chapeliferio}, hat factory, can be distinguished from \textbf{chapelvenderio}, hat store (V, 155). 

\textbf{-estr} denotes chief of, head of: \textbf{urbestro}, mayor; \textbf{navestro}, captain; \textbf{parokestro}, vicar. 

\textbf{-ey} denotes the place destined for an object or action. This place is, in general, a room, hall, or building: \textbf{kavaleyo}, horse stable; \textbf{tombeyo}, cemetery; \textbf{dormeyo}, sleeping room; \textbf{koqueyo}, kitchen, but \textbf{koquerio}, cookery; \textbf{laveyo}, washing room, but \textbf{laverio}, laundry; \textbf{lerneyo}, a room for studying, but \textbf{lernerio}, a school; \textbf{libreyo}, library (room) in a private dwelling (\textbf{biblioteko publika}, public library); \textbf{manjeyo}, dining room; \textbf{herbeyo}, meadow, grass field; \textbf{viteyo}, vineyard. 

The suffixes \textbf{-eri} and \textbf{-ey} are rather vague in their meanings. More precise words, therefore, must be used when necessary; instead of \textbf{lernerio}, school, the words \textbf{liceo}, \textbf{gimnazio}, \textbf{universitato}, instead of \textbf{pregeyo}, church, the words \textbf{kirko}, \textbf{kapelo}, \textbf{katedralo}, \textbf{baziliko}, etc. 

\textbf{-i} denotes the domain, province, country dependent upon the authority of a person: \textbf{parokio}, parish; \textbf{dukio}, duchy; \textbf{komtio}, county, shire; \textbf{monarkio}, monarchy; \textbf{baronio}, barony; \textbf{episkopio}, diocese. With the suffix \textbf{-ey} instead of \textbf{-i} these words would signify the residences, palaces, castles of the same persons; \textbf{baroneyo}, a baron's castle; \textbf{episkopeyo}, episcopal residence (III, 617; IV, 433). 

\textbf{-id} denotes the descendant, offspring: \textbf{Izraelido}, Israelite; \textbf{Napoleonido}, a descendant of Napoleon. 

\textbf{-ier} denotes the individual characterized by a certain attribute, object, or peculiarity: \textbf{gibiero}, hunchback; \textbf{kurasiero}, cuirassier; \textbf{rentiero}, fund holder, independent gentleman; \textbf{ringiero}, ringed worm (17). 

This suffix denotes also something that serves to hold one object (not to be confounded with the suffix \textbf{-uy}): \textbf{plumiero}, pen holder; \textbf{kandeliero}, candle stick; \textbf{sigariero}, cigar holder (see note to dec. 592, IV, 562). 

By extension this suffix designates also the tree or plant producing or carrying a flower or fruit: \textbf{pomiero}, apple tree; \textbf{teiero}, tea tree; \textbf{roziero}, rose tree. 

\textbf{-il} with a verbal root denotes the instrument, tool, or means of an action: \textbf{hakilo}, axe, hatchet; \textbf{pektilo}, comb; \textbf{telegrafilo}, telegraph (instrument for telegraphing); \textbf{barilo}, barrier; \textbf{impedilo}, impediment (18). 

The meaning of this suffix is rather vague. A special word is, therefore, preferable when an instrument is more closely defined. 

There are many kinds of cutting instruments (\textbf{tranchilo}) or shooting instruments (\textbf{pafilo}). A table knife is \textbf{kultelo}, a gun \textbf{fusilo}. To indicate a machine the word \textbf{mashino} is used: \textbf{skribmashino}, writing machine, typewriter; \textbf{sutomashino}, sewing machine. 

\textbf{-in} with a nominal root denotes an individual of female sex: \textbf{patro}, parent; \textbf{patrino}, mother; \textbf{filio}, child; \textbf{filiino}, daughter; \textbf{spozo}, spouse; \textbf{spozino}, wife; \textbf{doktoro}, doctor; \textbf{doktorino}, lady doctor; \textbf{bovo}, beef (bovine animal); \textbf{bovino}, cow; \textbf{kapro}, goat; \textbf{kaprino}, she-goat. 

With roots essentially feminine the suffix \textbf{-in} is superfluous: \textbf{amazono}, amazon; \textbf{megero}, a Megaera; \textbf{Parco}, one of the Parcae; \textbf{primadono}, prima donna; \textbf{subreto}, soubrette. \textbf{Feino}, fairy, however, is admissible because \textbf{feo} may signify a spirit of undetermined sex. 

\textbf{-ism} denotes a system, doctrine, or party: \textbf{idealismo}, idealism; \textbf{socialismo}, socialism; \textbf{Kristanismo}, Christianity. 

\textbf{-ist} denotes the person occupied professionally with a certain thing (see suffix \textbf{-er}): \textbf{artisto}, artist; \textbf{kantisto}, singer by profession; \textbf{kantero}, one who sings occasionally or habitually; \textbf{kantanto}, one who is just singing; \textbf{telegrafisto}, telegraphist. 

By extension the suffix \textbf{-ist} denotes also the adherent of a party, system, or doctrine: \textbf{socialisto}, socialist; \textbf{idealisto}, idealist; \textbf{Mohamedisto}, Mohammedan; \textbf{Budhisto}, Buddhist; \textbf{Luteristo}, Lutheran (VII, 283). 

The merchant may be distinguished from the producer by using \textbf{vendisto} for the former and \textbf{-ifisto} for the latter: \textbf{moblovendisto}, furniture dealer; \textbf{moblifisto}, furniture maker; \textbf{florvendisto}, dealer in flowers; \textbf{floristo}, floriculturist. (2nd paragraph of note 18.) 

\textbf{-ul} denotes an individual of male sex: \textbf{patro}, parent; \textbf{patrulo}, father; \textbf{frato}, brother or sister; \textbf{fratulo}, brother; \textbf{puero}, boy or girl; \textbf{puerulo}, boy; \textbf{spozulo}, hubsand; \textbf{bovulo}, ox; \textbf{kaprulo}, he-goat. 

\textbf{-ur} with a verbal root denotes the result of an action: \textbf{fenduro}, a split; \textbf{imituro}, imitation (the result of imitating; \textbf{imitajo}, the object undergoing the imitation, the model); \textbf{pikturo}, painting; \textbf{expresuro}, expression; \textbf{suturo}, suture (\textbf{sutajo}, the thing sewed); \textbf{aperturo}, opening. 


\small Remark. Sometimes it may not be possible to distinguish the result of an action from the object of the transitive verb expressing that action: \textbf{konstruktajo} (object), edifice, is the same as \textbf{konstrukturo} (result), construction. \normalsize

\textbf{-uy} with a nominal root denotes a recipient for something (case, sheath, box, chest): \textbf{instrumentuyo}, instrument case; \textbf{sigaruyo}, cigar box; \textbf{inkuyo}, inkwell; \textbf{sukruyo}, sugar box. 

\textbf{-yun} denotes the young of animals: \textbf{bovyunino}, heifer; \textbf{hanyuno}, pullet. \textbf{Yun} is also an independent root: \textbf{yuna}, young. 

63. Suffixes to form especially adjectives are the following: 

\textbf{-al} means relating to, pertaining to, suitable to, depending upon: \textbf{domala}, domestic, relating to the house; \textbf{universala}, universal, relating to the universe; \textbf{regulala}, regular, depending upon the rule. 

Frequently an adjective formed with \textbf{-al} is equivalent to a genitive: \textbf{riverala aquo}, river water; \textbf{maral fluo}, sea current (\textbf{aquo di rivero}, \textbf{fluo di maro}). From this results the practical rule: an adjective in \textbf{-al} is appropriate whenever it can be replaced by the genitive of the substantive it is formed from. 

One should avoid deriving adjectives in \textbf{-al} from proper names: Shakespeareian dramas, \textbf{drami da Shakespeare}; \textbf{Shakespeare-ala drami} means rather dramas worthy of, or analogous to those of, Shakespeare (Gramm. Compl., p. 58) (20).

\textbf{-atr} with non-verbal roots furnishes adjectives signifying of the nature of, approaching: \textbf{metalatra}, metallike; \textbf{kupratra}, copperlike. With a root indicating a color it designates a similar color: \textbf{flavatra}, yellowish; \textbf{bluatra}, bluish. 

\small Remark. Sometimes the suffix may be applied also to a verbal root: \textbf{saltatre}, jumping, as it were (VI, 596, bottom note). \normalsize

Adjectives in \textbf{-atra} never have a pejorative sense. The latter is expressed by the suffix \textbf{-ach} (see this suffix further down).

With technical terms the ending (suffix) \textbf{-oid} is equivalent to the suffix \textbf{-atr}: \textbf{*elipsoida}, ellipsoid; \textbf{*antropoida}, anthropoid; \textbf{*metaloido}, metaloid. 

The roots \textbf{form} and \textbf{simil} are used in the same sense as the suffix \textbf{-atr}. 

\textbf{-e} means having the color or aspect of: \textbf{laktea}, of milk color; \textbf{rozea}, rose colored; \textbf{tigrea}, of the aspect of a tiger; \textbf{sangea}, blood colored; \textbf{violea}, violet colored; \textbf{orea}, golden; \textbf{kremea}, cream colored; \textbf{cindrea}, ash colored; \textbf{pulcea}, puce; \textbf{kastanea}, chestnut colored; \textbf{chamea}, chamois colored. 

\textbf{-ebl} with verbal roots furnishes adjectives signifying capable of and having an essentially passive sense: \textbf{videbla}, visible, capable of being seen; \textbf{ruptebla}, breakable, fragile; \textbf{kredebla}, credible. 

\small Remark. The suffix \textbf{-ebl} is not applicable to intransitive verbs, but only to transitive and mixed verbs: \textbf{variebla}, variable (\textbf{variar}, to vary, being a mixed verb, III, 102, 466). \normalsize

\textbf{-em} with verbal roots furnishes adjectives signifying inclined to, having the tendency to: \textbf{babilema}, talkative; \textbf{laborema}, industrious.

\small Remark. The suffix \textbf{-em}, though usually applied to verbal roots, is sometimes applicable also to non-verbal roots (L. Couturat, VI, 596): \textbf{ebriema}, given to drink; \textbf{maladema}, sickly. In such instances it may perhaps be preferable to say: \textbf{maladesema}, \textbf{ebriesema}. \normalsize

\textbf{-end} with verbal roots furnishes adjectives signifying to be done, and having an essentially passive meaning: \textbf{solvenda}, to be solved; \textbf{kredenda}, what should be believed. 

\textbf{-ik} means ill of, sick with: \textbf{tuberkulosika}, sick with tuberculosis (V, 1); \textbf{diabetika}, sick with diabetes. 

\textbf{-ind} with verbal roots furnishes adjectives signifying worthy of, deserving of: \textbf{aminda}, amiable; \textbf{respektinda}, worthy of respect; \textbf{kredinda}, deserving of belief. 

\textbf{-iv} with verbal roots furnishes adjectives signifying capable of, and having an essentially active meaning: \textbf{instruktiva}, instructive; \textbf{konvinkiva}, convincing; \textbf{mortiva}, mortal (that can die); \textbf{nutriva}, nourishing. 

\small Remark. The suffix \textbf{-iv} denotes a capability that a thing possesses through its very nature (L. Couturat, VI, 597). When, for instance, a substance purges somebody accidentally, it is not \textbf{purgiva}, but only \textbf{purganta}. Metals are natural conductors of electricity, they are properly \textbf{konduktiva}. \normalsize

\textbf{-oz} with nominal roots furnishes adjectives signifying full of, provided with, ornamented with, containing (containing through its very nature, L. Couturat, VI, 595): \textbf{sabloza}, sandy; \textbf{danjeroza}, dangerous; \textbf{kurajoza}, courageous; \textbf{poroza}, porous; \textbf{ambicioza}, ambitious. 

64. Suffixes to form especially verbs are the following: 

\textbf{-ad} with verbal roots denotes the prolongation of the action expressed by the simple verb: \textbf{parolar}, to speak; \textbf{paroladar}, to speak at length; \textbf{pafar}, to shoot; \textbf{pafadar}, to carry on a fusillade. 

\textbf{-es} is the root of the verb \textbf{esar}, to be. Added to the root of a verb it forms the passive of that verb: \textbf{konstruktesar}, to be constructed (see § 39). Verbs formed with this suffix and a non-verbal root signify to be such and such: \textbf{belesar}, to be beautiful; \textbf{sajesar}, to be wise; \textbf{maladesar}, to be sick; \textbf{similesar}, to resemble; \textbf{amikesar}, to be a friend; \textbf{friponesar}, to be a scoundrel. 

It is not recommendable to add the verb \textbf{esar} (the same as the suffix \textbf{-esk}, see further) to a derived or composed word (VII, 161): \textbf{nutrivesar} is better dissolved into \textbf{esar nutriva}, to be nourishing. 

Nouns formed from verbal roots and the suffix \textbf{-es} denote a passive state: \textbf{edukeso}, education (state of being educated); \textbf{exhausteso}, exhaustion; \textbf{expanseso}, expansion; \textbf{konvinkeso}, conviction; \textbf{konstrukteso}, construction. When it is desirable to indicate also the tense of such states, the suffix may be appended to the passive participle: \textbf{konstruktateso}, \textbf{konstruktiteso}. 

Nouns formed from non-verbal roots and this suffix denote a state or abstract quality: \textbf{saneso}, health; \textbf{maladeso}, sickness; \textbf{beleso}, beauty; \textbf{qualeso}, quality. 

\textbf{-esk} with verbal roots forms verbs denoting the beginning of an action: \textbf{ameskar}, to fall in love; \textbf{dormeskar}, to fall asleep, to begin to sleep; \textbf{videskar}, to perceive; \textbf{sideskar}, to sit down (to begin to sit); \textbf{iraceskar}, to fly into a rage. 

This suffix imparts to verbs an inchoative sense without changing their voice. Passive inchoative verbs are therefore formed by adding the suffix to the passive participle: \textbf{vidateskar}, to become (to begin to be) visible; \textbf{acendateskar}, to become lit (to start to burn). Since the suffix may be connected with the root of any verb whatever, passive inchoative verbs may also be formed by adding it to the root of the passive verb built synthetically: \textbf{videseskar}, to become visible; \textbf{acendeseskar}, to become lit. 

With adjective and substantive roots the suffix \textbf{-esk} forms verbs signifying to become (begin to be): \textbf{petreskar}, to become petrified, to turn into stone; \textbf{redeskar}, to blush; \textbf{paleskar}, to turn pale. Instead of the suffix \textbf{-esk} the verb \textbf{divenar} should be used with a composed or derived word; \textbf{nutriveskar} should be dissolved into: \textbf{divenar nutriva}, to become nourishing (see above, suffix \textbf{-es}). 

\small Remark. With non-verbal roots \textbf{-esk} may be considered as abbreviated from \textbf{-esesk}: \textbf{paleskar} = \textbf{paleseskar}; \textbf{petreskar} = \textbf{petreseskar}. \normalsize

A third way to build passive inchoative verbs consists in forming the synthetic passive of active inchoative verbs: \textbf{videskesar} (synthetic passive of the active verb \textbf{videskar} = \textbf{videseskar} = \textbf{vidateskar}.) It is recommended to use preferably the last one of these three forms (VII, 69). 

\textbf{-if} with nominal roots forms verbs denoting production, generation: \textbf{pomifar}, to produce, to bear apples; \textbf{versifar}, to make verses, \textbf{sudorifar}, to perspire; \textbf{urinifar}, to urinate (21). 

\textbf{-ig} with substantive and adjective roots forms verbs signifying to make, to render, to transform into: \textbf{purigar}, to render clean; \textbf{beligar}, to beautify; \textbf{petrigar}, to petrify; \textbf{filigar}, to turn into thread. 

With roots of intransitive verbs the suffix forms transitive verbs signifying to cause to do the action expressed by the intransitive verb. The subject of the intransitive verb transformed becomes the object of the transitive verb formed. \textbf{La matro dormigas l’infanto}, the mother puts the child to sleep (\textbf{l'infanto dormas}). \textbf{La tirano mortigis la kaptiti}, the tyrant killed the prisoners, made the prisoners die (\textbf{la kaptiti mortis}). \textbf{Venigez la medicinisto}, fetch the doctor (\textbf{la medicinisto venos}). 

With roots of transitive verbs it forms transitive verbs signifying to make somebody or something undergo the action expressed by the transformed verb. The object of the verb transformed becomes also the object of the verb formed, and the subject of the transformed verb, if mentioned, is given with the preposition \textbf{da}: \textbf{lu kredigis ca rakonto da l'dupo}, he palmed this tale off on the dupe (\textbf{la dupo kredis la rakonto}). \textbf{Me vidigis la muzeo da mea filiineti}, I showed my little daughters the museum (\textbf{mea filiineti vidis la muzeo}). \textbf{Me sendigas mea letri da mea spozino a mea fratino}, I make my wife send my letters to my sister. \textbf{El mueligas la kafeo da l’muelilo}, she makes the mill grind the coffee (\textbf{la muelilo muelas la kafeo}) (22). 

\textbf{Ig} serves as independent root to form the verb \textbf{igar}, to make, to render, to cause. \textbf{Lu igis sua flechi venenoza}, he made his arrows poisonous. He made poisoned arrows, \textbf{lu facis venenoza flechi}. When \textbf{igar} is followed by a direct object and an infinitive, a strict order of the parts of the sentence is necessary to avoid ambiguities. \textbf{La rejo igis la generalo arestar la policestro}, the king made the general arrest the chief of police. 

\textbf{-iz} with substantive roots forms verbs signifying to furnish, provide, endow, impregnate with: \textbf{honorizar}, to honor; \textbf{kolorizar}, to color; \textbf{armizar}, to arm (23). 

65. Suffixes to form words of every kind (verb, substantive, adjective, and adverb) are the following: 

\textbf{-ach} imparts a pejorative sense to the original word: \textbf{mediko}, physician; \textbf{medikacho}, quack; \textbf{parolar}, to speak; \textbf{parolachar}, to babble; \textbf{populo}, people; \textbf{populacho}, populace; \textbf{ridar}, to laugh; \textbf{ridachar}, to grin; \textbf{dolca}, sweet; \textbf{dolcacha}, sickly sweet, mawkish. 

The suffix must not be used to distinguish animal organs and functions from human ones, except when they are mentioned expressly with the idea of contempt or disgust. 

\textbf{-eg} forms augmentatives, i.e. denotes a higher or extreme degree. At the same time it may modify somewhat the meaning of the original word: \textbf{varma}, warm; \textbf{varmega}, hot; \textbf{granda}, big; \textbf{grandega}, immense; \textbf{pluvo}, rain; \textbf{pluvego}, downpour; \textbf{domo}, house; \textbf{domego}, mansion. 

\small Remark. Very much is not to be translated by \textbf{treege}, but by \textbf{tre multe} or \textbf{extreme} (I, 709). \normalsize

\textbf{-et} forms diminutives or indicates a smaller degree, sometimes modifying thereby the sense of the original word: \textbf{domo}, house; \textbf{dometo}, cottage; \textbf{dormar}, to sleep; \textbf{dormetar}, to slumber; \textbf{ridar}, to laugh; \textbf{ridetar}, to smile; \textbf{bela}, beautiful; \textbf{beleta}, pretty; \textbf{varma}, warm; varmeta, tepid. 

The suffix is also used to express endearment: \textbf{fratineto}, dear little sister; \textbf{Karleto}, dear Charlie. 

As shown by some of the above examples the suffixes \textbf{-eg} and \textbf{-et} frequently imply a qualitative change of the original word besides the quantitative one. 

\textbf{-um} has among the suffixes the same role as \textbf{ye} among the prepositions. It does not indicate a definite relation between the original word and the word formed by it. It is employed whenever no other suffix would suffice to form a new word standing in some relation with the original word: \textbf{kolo}, neck; \textbf{kolumo}, collar; \textbf{kruco}, cross; \textbf{krucumar}, to cross, to traverse; \textbf{folio}, leaf; \textbf{foliumar}, to turn the pages of; \textbf{formiko}, ant; \textbf{formikumar}, to team, to swarm (24) (see end of § 48). 

This suffix is to be employed but rarely. 

66. The numeral endings \textbf{-esm}, \textbf{-foy}, \textbf{-im}, \textbf{-op}, \textbf{-opl } (see §§ 43-47) and the participial endings \textbf{-ant}, \textbf{-int}, \textbf{-at}, etc., are to be counted among the suffixes.

67. Some independent roots may be regarded as quasi affixes. Thus the suffixes \textbf{-es}, \textbf{-foy}, \textbf{-ig}, \textbf{-yun} mentioned before are the roots of the words \textbf{esar}, \textbf{foyo}, \textbf{igar}, \textbf{yuna}. Another independent root frequently used as a suffix is \textbf{-ag} from the verb \textbf{agar}, to act. It is employed to derive from the names of utensils and instruments verbs indicating the action done with the instrument: \textbf{butonagar}, to button; \textbf{krucagar}, to crucify; \textbf{martelagar}, to hammer.

\small Remark, \textbf{-ag} is to be used when the instrument is the primitive word and the action done with it is secondary (III, 212, 292; note 18 to suffix \textbf{-il}). \normalsize

The root \textbf{un} as suffix denotes one individual unit of a substance naturally consisting of many such units: \textbf{sablo}, sand; \textbf{sabluno}, grain of sand; \textbf{grelo}, hail; \textbf{greluno}, hailstone. \textbf{Grano}, grain, is preferable to \textbf{uno} in expressing one individual unit of corn; \textbf{sekalo}, rye; \textbf{sekalgrano}, grain of rye. When a substance has no natural units, \textbf{uno} is not appropriate and instead of it are used words like \textbf{peco}, \textbf{floko}, \textbf{parto}, \textbf{parteto}, etc.: \textbf{nivofloko}, flake of snow; \textbf{sukropeco}, piece of sugar. 

The root prim from the adjective \textbf{prima}, primitive, original, is used as a sort of prefix to form some words: \textbf{primpopulo}, aborigines; \textbf{primavi}, ancestors. 

68. Several affixes may be added to one root: \textbf{medicinistacho}, quack; \textbf{kredebleso}, credibility; \textbf{laboremeso}, diligence; \textbf{mortiveso}, mortality; \textbf{militistaro}, army; \textbf{desfortunozigar}, to render unfortunate; \textbf{prematureso}, prematurity; \textbf{miskomprenigar}, to make one misunderstand; \textbf{neprevidebleso}, impossibility of being foreseen.

\subsection*{FORMATION OF WORDS BY COMPOSITION.}
\addcontentsline{toc}{subsection}{Formation of Words by Composition—§§ 69-74}
69. Another way of forming new words consists in contracting two or more simple words into one. The principal or defined word becomes the last part of the new formation and the defining word (words) its first part (parts). Of the latter only the root is used without grammatical ending: \textbf{plumkultelo}, penknife; \textbf{fervoyo}, railway. Euphony may require that the defining word be used with grammatical ending: \textbf{skribotablo}, writing table; \textbf{manjochambro}, dining room. 

If a composed word be too long, it must be dissolved: \textbf{urbobankdirektisto}, director of a city bank, is to be dissolved into \textbf{direktisto di urbobanko}. 

70. The definition given to the last part of a composed word by the preceding part is variable. In \textbf{urbobanko} it is that of a genitive, \textbf{banko di la urbo}; in \textbf{manjochambro} it is purpose, \textbf{chambro por manjo}; in \textbf{violonmuziko}, violin music, it is means, \textbf{muziko produktata per violono}; in \textbf{kastanbruna}, chestnut brown, it is comparison, \textbf{bruna quale kastano}; in \textbf{agultruo}, needle hole, it is place, \textbf{truo en agulo}. From the above examples it may be seen that in forming one composed word from two connected by a preposition, the latter is left out, the relation of the two parts being easily understood without it. This holds good especially when the principal word is a verb, instead of \textbf{ludar per shako}, to play chess, one word may be used, \textbf{shakludar}. 

71. Particles (prepositions, adverbs, numbers) may also enter into combination with other words to form new words: \textbf{senhara}, hairless; \textbf{neposibla}, impossible; \textbf{interstato}, intermediate state; \textbf{tilnuna}, hitherto prevailing; \textbf{ektirar}, to pull out; \textbf{enpozar}, to put in; \textbf{viceprezidanto}, vice-president; \textbf{undia}, ephemeral (25). 

Care must be taken not to impart to a preposition in a composed word a meaning different from the one it ordinarily has: \textbf{ekparolar} can mean only to speak out (out of a room), but not to pronounce. 

The particles \textbf{ne-} and \textbf{-sen} must not be confounded in composed words. \textbf{Sen} can be applied only to substantive roots to form new words (usually adjectives), while \textbf{ne} is prefixed, as a rule, to adjectives in order to negative them: \textbf{nelumoza chambro}, a room without light, but \textbf{senluma chambro}; \textbf{senhelpa kriplo}, a helpless cripple; \textbf{senduba}, doubtless, but only \textbf{nedubebla}, indubitable. 

Both \textbf{sen} and \textbf{ne} differ from \textbf{des-} in that they simply negative the words they are prefixed to, while \textbf{des} expresses their opposite and is therefore only applicable where there is a middle between two extremes: \textbf{akuta}, sharp; \textbf{neakuta}, not sharp; \textbf{desakuta}, (pronouncedly) dull; \textbf{utila}, useful; \textbf{neutila}, useless; \textbf{desutila}, harmful. From \textbf{movo}, motion, only \textbf{senmova}, motionless, can be formed there being no intermediate condition. 

72. A transitive verb remains transitive when a preposition is prefixed and its object remains the same as without the preposition. \textbf{La gardenisto surspricas aquo sur la arboro}, the gardener sprinkles water over the tree; \textbf{la aquo esas surspricata, ne la arboro}, the water is sprinkled over, not the tree. If tree should be the object, another suitable verb must be used, as \textbf{aspersar}. \textbf{Lu aspersas la arboro per aquo}, he sprinkles the tree with water. \textbf{Malgre densa ne bulo la navestro povis travidar la faro}, in spite of a dense fog the captain could see the light-house through (through the fog). \textbf{La faro esas travidebla, ne la nebulo, ma la nebulo esas diafana malgre sua denseso}, the light-house is visible (visible through, through the fog), not the fog, but the fog is transparent in spite of its density. 

Intransitive verbs sometimes become transitive admitting also of the formation of a passive voice, when a preposition is prefixed. \textbf{La kavalo transsaltis la obstaklo ed en kuris la stablo}, the horse jumped over the obstacle and ran into the stable; \textbf{la obstaklo transsaltita}, the obstacle that has been jumped over (26). 

73. Affixes are frequently not applicable to composed words, especially those composed with a preposition, because the relation which would be indicated by the affix is already sufficiently expressed by the composition itself; \textbf{okulal doloro}, eye pain, but only \textbf{bluokula puerino}, blue eyed girl; \textbf{orelal festeno}, a feast for the ears, but only \textbf{duo rela bruiso}, a noise heard in both ears; \textbf{teral teleskopo}, a terrestrial telescope, but \textbf{subtera voyo}, underground way; \textbf{lumoza karcero}, bright prison, but \textbf{senluma groto}, a grotto without light; \textbf{nivoza monto}, snowy mountain, but \textbf{subniva glacio}, a glacier under the snow; \textbf{maral fluo}, sea current, but \textbf{submara planti}, plants under the sea; \textbf{aquoza sulo}, watery soil, but \textbf{superaqua aernavi}, airships above the water; \textbf{gardenal pordo}, garden door, but \textbf{intergardena pordo}, door between two gardens; \textbf{statal komerco}, State’s commerce, but \textbf{interstata komerco}, interstate commerce; \textbf{nacionala linguo}, national language, but \textbf{internaciona linguo}, international language; \textbf{fingrala ludo}, finger play, but \textbf{longfingra manuo}, longfingered hand.

In instances like the following composed words require an affix: \textbf{urbobankal direktisto}, director of a city bank; \textbf{primaval historio}, ancestral history; \textbf{maraquoza sulo}, soil containing sea water; \textbf{intergardenal pordo}, door of an interposed garden; \textbf{interstatal komerco}, commerce of an intermediate State. 

From the numerous examples given above the practical rule showing when composed words may assume affixes becomes apparent. If a composed word is a noun it may assume a suffix applicable to a nominal root\footnotemark[1]. Thus in \textbf{maraquoza sulo} the adjective consists of the noun \textbf{maraquo} and the suffix \textbf{-oz}; in \textbf{intergardenal pordo} it consists of the noun \textbf{intergardeno} and the suffix \textbf{-al}; in \textbf{primaval historio} of the noun \textbf{primavo} and the suffix \textbf{-al}. On the other hand while \textbf{orelala festeno} is correct, \textbf{duorelala bruiso} is not, for the adjective \textbf{duorelala} could not be explained as consisting of the suffix \textbf{-al} and the noun \textbf{duorelo}. If such a noun exists at all, it is not applicable in this instance. A similar consideration excludes \textbf{subnivoza glacio}, \textbf{superaquoza aernavi}, \textbf{longfingral manuo}, etc., while \textbf{nivoza monto}, \textbf{aquoza sulo}, \textbf{fingral ludo}, etc., are correct. 
\footnotetext[1]{This rule will be more easily understood by the following explanation: The question concerns almost invariably the formation of an adjective from a composed word either by immediate derivation or by the suffixes \textbf{-al} and \textbf{-oz}. Now those suffixes are applicable only to substantive roots (and only by the intermedium of the verbal substantive also to verbal roots). It follows therefore that the composed word must be a substantive, and an appropriate one, i.e., one that fits the case. If it is not, the derivation must be mediate.

The difficult question whether the derivation of a word from a composed word is to be accomplished by a suffix or immediately, has been extensively treated in Progreso, but never solved. The author has answered it fully in an essay entitled: \textbf{Studiuro pri la derivo di vorti de kompozita radiki.} This essay has been mentioned in Progreso of July, 1914 (VII, p. 398). According to information just (June, 1919) received from a publisher in Switzerland the essay was published in the succeeding number (Progreso, Aug., 1914, pp. 486-495) which as yet has not been put into circulation owing to the war.}

The same composed root will sometimes admit a suffix, sometimes not, according to the meaning it has in the phrase and in conformity with the above rule: \textbf{interstatal komerco}, commerce of an intermediate state (\textbf{komerco di interstato}), but \textbf{interstata komerco}, interstate commerce; \textbf{intergardenal pordo}, door of an interposed garden (\textbf{di intergardeno}), but \textbf{intergardena pordo}, door between (two) gardens; \textbf{subaquoza planto}, a plant containing deeper water (\textbf{plena de subaquo}), but \textbf{subaqua planto}, a plant under the water (27). 

74. A numeral in composed words occupies the first or the second place according to whether it is the defining word or the defined one. In \textbf{yarcento}, century, the defining word is \textbf{yaro} (\textbf{cento de yari}, a hundred of years); \textbf{epoki yarcentoza}, epochs of centuries. With adjectives indicating duration, age, etc., the numeral forms the first component: \textbf{triadekyara milito}, war of thirty years; \textbf{puerulo dek-e-duyara}, a boy of 12 years; \textbf{centyara homo}, a person hundred years old, but \textbf{yarcentiero}, centenarian. 

(Note 28 contains the conclusion of Part II) 