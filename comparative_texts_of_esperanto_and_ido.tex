I.
\begin{parcolumns}{2}
    \colchunk{%
        \begin{otherlanguage}{esperanto}
        \begin{center}Esperanto\end{center}
        Ĝu ŝi scias ĉion?\footnotemark[1] \\
        Ĝu ŝi ciam rugigas pri ĉio? \\
        Ĝu ŝia voĉo placas al vi? \\
        Ĝu sia fianĉo sercis sin? \\
        Car ŝi ne scias, ĉu ŝia ĉapelo estas tie-ĉi aŭ ĉe ŝia ĉambro, sercu ĝin ĉie.  \\
        Ŝi ĉiam ĉarmas ĉiujn per sia gentileco; eĉ ŝiaj malamikoj konfesas tion-ĉi. \\
        Mi ne amas ŝiajn infanojn, car ili ĉiam turmentas miajn hundojn kaj eĉ miajn kana riojn, kiujn ili timigas per siaj krioj. \\
        \end{otherlanguage}
    }
    \colchunk{%
        \begin{center}Ido\end{center}
        Kad el savas omno? \\
        Kad el sempre redeskas pri omno? \\
        Kad lua voco plezas a vu? \\
        Kad lua fianco serchis el? \\
        Pro ke el ne savas, kad lua chapelo esas hike od en lua chambro, serchez ol omnube. \\
        El sempre charmas omni per sua jentileso; mem lua ene miki konfesas co. \\
        Me ne amas elua infanti, nam li sempre turmentas mea hundi e mem mea kanarii, quin li timigas per sua krii. \\
    }
    \colplacechunks
\end{parcolumns}
\footnotetext[1]{In the original edition of this book, the author noted here that it was necessary to set this section in larger type due to the six accentuated letters ĉ, ĝ, ĥ, ĵ, ŝ, ŭ peculiar to Esperanto. He considered this an illustration of a substantial defect of that language.}
II.
\newline
\begin{parcolumns}[nofirstindent=true]{2}
    \colchunk{%
        \begin{otherlanguage}{esperanto}
        Mult\emph{aj} kompetent\emph{aj} person\emph{oj} estas konvinkit\emph{aj}, ke kelk\emph{aj} ŝanĝ\emph{oj} k\emph{aj} plibonig\emph{oj} estas dezirindaj, k\emph{aj} eĉ neces\emph{aj}. Ekzemple, la mult\emph{aj} grotestk\emph{aj} k\emph{aj} asburd\emph{aj} fin\emph{aj} ``j'' devas esti aboliciitaj. Ti\emph{aj} k\emph{aj} ali\emph{aj} malbelaĵ\emph{oj} k\emph{aj} malfacilaĵ\emph{oj} estas aboliciit\emph{aj} de la Internaciona Delegacio. \\
        \end{otherlanguage}
    }
    \colchunk{%
        Multa kompetenta personi esas konvinkita, ke kelka chanji e plubonigi esas dezirinda, e mem necesa. Exemple la multa groteska ed absurda finali ``j'' devas esar abolisita. Tala ed altra ledaji e desfacilaji esas abolisita da la Internaciona Linguala Delegitaro. \\
    }
    \colplacechunks
\end{parcolumns}
III. End of Don Juan translated into Esperanto by the president of its Linguistic Committee:
\newline
\begin{parcolumns}[nofirstindent=true]{2}
    \colchunk{%
        \begin{otherlanguage}{esperanto}
        Ha ! mian salajron ! mian sa lajron! Jen estas, per lia mor to, ĉiu kontentigita. Ofendita ĉielo, transpaŝit\emph{aj} leĝoj, delogit\emph{aj} knabinoj, malhonorigit\emph{aj} famili\emph{oj}, insultit\emph{aj} gepatr\emph{oj}, difektit\emph{aj} edzin\emph{oj}, furiozigit\emph{aj} edzoy, ĉiuj estas kontent\emph{aj}; nur mi sola estas malfefiĉa. \\
        
        Mian salajron! Mian salajron! Mian salajron \\ %should be indented
        \end{otherlanguage}
    }
    \colchunk{%
        Ha! Mea salario ! Mea sala rio. Yen esas per lua morto, omnu kontentigita. Ofensita cielo, violacita legi, seduktita puerini, senhonorigita familii, insultita patri, koruptita spo zini, furianta spozuli, omni esas kontenta; nur me sola esas des felica. \\
        
        Mea salario ! Mea salario ! Mea salario! \\  %should be indented
    }
    \colplacechunks
\end{parcolumns}
IV. Here is an extract from the classical reading book of Esperanto (Esperantaj Prozajoj, p. 34):
\newline
\begin{parcolumns}[nofirstindent=true]{2}
    \colchunk{%
        \begin{otherlanguage}{esperanto}
        La vojo al la ebenaĵo, flanke de la ŝaŭmanta rivero Pesio, iras tra klinaj, florant\emph{aj} herbej\emph{oj} k\emph{aj} sub ale\emph{oj} da maljun\emph{aj} kaŝtanujoj; poste pli supre, arbaret\emph{oj} da fagoj, k\emph{aj} avelujoj; k\emph{aj} post ili la  mont\emph{aj} herbej\emph{oj} aŭ la akr\emph{aj} k\emph{aj} ŝtoneg\emph{aj} suproj, kiuj, gajigit\emph{aj} da milspec\emph{aj} alp\emph{aj} floroj, leviĝas ĝis du mil metroj. \ldots La alpist\emph{oj} amant\emph{aj} la pl\emph{ej} malfacil\emph{ajn} supren-rampad\emph{ojn} ne povas esti malkontentaj, k\emph{aj} la kolektant\emph{oj} de kreskaĵ\emph{oj}, papili\emph{oj}, insekt\emph{oj}, aŭ konk\emph{oj} tie trovas mult\emph{ajn} maloft\emph{ajn} spec\emph{ojn}. \\
        \end{otherlanguage}
    }
    \colchunk{%
        La voyo a la planajo, latere di la spumifanta rivero Pesio, iras tra inklinita floroza herbeyi e sub alei de olda kastanieri; pose plu supre, bosketi de fagi ed avelieri; e pos li la montala herbeyi o l'akuta e rokoza supraji, qui, gayigita da milspeca alpala flori, su levas til duamil metri. \ldots La alpisti amanta la maxim desfacila acensi ne povas esar deskon tenta, e la kolektanti di planti, papilioni, insekti o konki trovas ibe multa rara speci. \\
    }
    \colplacechunks
\end{parcolumns}
V. Here is an extract from La Virineto de l'maro (Sea Virgin) by Dr. L. Zamenhof (Fundamenta Krestomatio, p. 52):
\newline
\begin{parcolumns}[nofirstindent=true]{2}
    \colchunk{%
        \begin{otherlanguage}{esperanto}
        Kiam la suno eklumis super la maro, ŝi vekiĝis k\emph{aj} sentis fortan doloron, sed rekte antaŭ ŝi staris la aminda juna reĝido, kiu direktis sur ŝin si\emph{ajn} okul\emph{ojn} nigr\emph{ajn} kiel karbo, tiel ke ŝi devis mallevi la si\emph{ajn}, k\emph{aj} tiam ŝi rimarkis, ke ŝia fiŝa vosto perdiĝis k\emph{aj} ŝi havis la plej graci\emph{ajn} malgrand\emph{ajn} blank\emph{ajn} piedet\emph{ojn}, kiujn bela knabino nur\footnotemark[1] povas havi. Sed ŝi estis tute nuda, k\emph{aj} tial ŝi envolvis sin en si\emph{ajn} dens\emph{ajn} long\emph{ajn} har\emph{ojn}. \\ %add footnote on "nur" 
        \end{otherlanguage}
    }
    \colchunk{%
        Kande la suno brileskis su per la maro, el vekis e sentis forta doloro, ma direte avan el stacis Paminda yuna rejido, qua direktis ad el sua okuli nigra quale karbono, tale ke el devis abasar sui, e lore el re markis, ke el perdis sua fishal kaudo e ke el havis la maxim gracioza mikra blanka pedeti, quin bela puerino irgatempe\footnotemark[1] povas havar. Ma el esis tote nuda, e pro to el envolvis su en sua densa longa bari. \\ %add footnote on "irgatempe"
    }
    \colplacechunks
\end{parcolumns}
\footnotetext[1]{Zamenhof's adverb `\textbf{nur}' in this sentence is indeed idiomatic as Progreso (II, 227) correctly points out. But the lack of any adverb in the Ido translation given in Progreso makes the sentence rather colorless. It would appear that some such adverb as `\textbf{irgatempe},' used by the writer, or \textbf{omnakaze}, etc., would render clear the idea which Zamenhof most probably had in mind and which he expressed idiomatically by the adverb `\textbf{nur}.'}
VI. The following text is by the foremost Esperantist of Belgium, a member of the Linguistic Committee (Belga Sonorilo, 57, p. 112, March 3, 1907):
\begin{parcolumns}[nofirstindent=true]{2}
    \colchunk{%
        \begin{otherlanguage}{esperanto}
        \begin{center}La Lunanoj \end{center}
        La Lunan\emph{oj} havas blu\emph{ajn} har\emph{ojn} verdan haŭton, violkolor\emph{ajn} lip\emph{ojn} k\emph{aj} nigr\emph{ajn} dent\emph{ojn}. Anstataŭ ung\emph{ojn} ili havas je la pied\emph{oj} k\emph{aj} la man\emph{oj} malmol\emph{ajn} bril\emph{ajn} k\emph{aj} polurit\emph{ajn} hokeg\emph{ojn}. La ali\emph{aj} part\emph{oj} de ilia korpo estas kovrit\emph{aj} per blanka lanugo, dolĉa k\emph{aj} silkeca kiel la blank\emph{aj} k\emph{aj} silkec\emph{aj} plu\-m\emph{oj} de ni\emph{aj} cign\emph{oj}. Ili havas ruĝ\emph{ajn} okul\emph{ojn} superigit\emph{ajn} per grand\emph{aj} flav\emph{aj} brov\emph{oj}  kiujn ili starigas laŭvole por sin ŝirmi kontraŭ la blindigant\emph{aj} sunradioj. Ilia dorso estas ornamita per du flugil\emph{oj}  siiml\emph{aj} al tiuj de ni\emph{aj} vespert\emph{oj} sed milkolor\emph{aj}  kiel la flugil\emph{oj} de la plej bel\emph{aj} bird\emph{oj} el la Sud-Ameri\-k\emph{aj} arbaregoj; ili prezentas ĉe la sunbrilo la plej bel\emph{ajn}  la plej divers\emph{ajn} nuanc\emph{ojn} la plej ĉarm\emph{ajn} k\emph{aj} la pl\emph{ej} riĉ\emph{ajn} rebril\emph{ojn}. Fine, kiel plenego da koketeco, la lunan\emph{oj} havas kvadratan kapon, kvadratan korpon, kvadrat\emph{ajn} kru\-r\emph{ojn} k\emph{aj} brak\emph{ojn}! \\
        \end{otherlanguage}
    }
    \colchunk{%
        \begin{center}La Lunani\end{center}
        La Lunani havas blua hari, verda pelo, violkolora labii e nigra denti. Vice ungi li havas ye la pedi e la manui, harda, brilanta e polisita hokegi. La altra parti di ilia korpo esas kovrita per blanka lanugo, dol ca e silkatra quale la blanka e silkatra plumi di nia cigni. Li havas reda okuli superbordizita per granda flava brovi, quin li stacigas segunvole por protek tar su kontre la blindiganta sunradii. Lia dorso esas ornita per du ali, simila a ti di nia vespertilii, ma milkolorizita quale la ali di la maxim bela uceli di la Sud-Amerikal fo resti; li prizentas en la sunala brilo la maxim bela diversa nuanci, la maxim charmanta e richa reflekti. Fine, quale kom pletigo di koketeso, la lunani havas quadrata kapo, quadrata korpo, quadrata gambi e bra kii! \\
    }
    \colplacechunks
\end{parcolumns}
VII. The following is an extract from Ernest Renan's Souvenirs d'enfance et de jeunesse, one of the most beautiful pieces of French literature:
\newline
\begin{parcolumns}[nofirstindent=true]{2}
    \colchunk{%
        \begin{otherlanguage}{esperanto}
        Mi estas, bluokula diino, naskita de barbar\emph{aj} gepatr\emph{oj}  ĉe la Kimerian\emph{oj} bonkor\emph{aj} k\emph{aj} vir tam\emph{aj}  ki\emph{uj} loĝas borde de maro malhela, plena je elstari ĝant\emph{aj} ŝtoneg\emph{oj}  ĉiam batata de l'fulmotondr\emph{oj}  Tie apene estas konata la suno; la flor\emph{oj} estas la mar\emph{aj} musk\emph{oj}  la alg\emph{oj} k\emph{aj} la kolor\emph{aj} konk\emph{oj} trovat\emph{aj} en la fundo de l'golfet\emph{oj} dezert\emph{aj}  Tie senkolor\emph{aj} ŝajnas la nub\emph{oj}  k\emph{aj} eĉ la ĝojo estas iom malgaja; sed font\emph{oj} malvarmakv\emph{aj} elsprucas tie el la ŝtoneg\emph{oj}  k\emph{aj} la okul\emph{oj} de l'junulin\emph{oj} estas kiel tiuj-ĉi fontan\emph{oj} verd\emph{aj}  en ki\emph{uj}  sur fund\emph{oj} de ondolini\emph{aj} herb\emph{oj}  kvazaŭ en spegulo rigardas sin la ĉielo. 
        
        Mi\emph{aj} prapatr\emph{oj}  tiom malproxime kiom ni povas en estinteco malantaŭe\-ni\-ri, estis sin dediĉint\emph{aj} je la for\emph{aj} marveturad\emph{oj} sur mar\emph{oj} neniam konit\emph{aj} de la Argonaut\emph{oj} Mi, dum juneco, aŭ\-dis la kant\emph{ojn} pri forvojaĝ\emph{oj} al poluso; lulata estis mi ĉe la memordir\emph{oj} pri la glaci\emph{oj} naĝant\emph{aj}  pri la nebul\emph{aj} je lakto simil\emph{aj} mar\emph{oj}  pri la insul\emph{oj} loĝa\-t\emph{aj} de bird\emph{oj} ki\emph{uj} kantis je si\emph{aj} hor\emph{oj}  k\emph{aj} ki\emph{uj} ekflugante ĉi\emph{uj} kune, mallumigis la ĉielon. 
        
        Pastr\emph{oj} de fremda religio, devenint\emph{aj} de la Sirian\emph{oj} el Palestino, zorgis mian edukadon. Saĝ\emph{aj} k\emph{aj} sankt\emph{aj} estis ti\emph{uj} ĉi pastr\emph{oj}  Ili instruis a me la long\emph{ajn} histori\emph{ojn} de Krono, kiu kreis la mondon, k\emph{aj} de lia filo, kiu (oni diras) plenumis voyaĝon al la tero. Ili\emph{aj} templ\emph{oj} trifoje kiel cia estas alt\emph{aj}  o Euritmo, k\emph{aj} simil\emph{aj} je arbar\emph{oj}  sed ili ne estas fortik\emph{aj}  ili disfalas ruinigit\emph{aj} post kvin aŭ sescent jar\emph{oj}  ili estas fantaziaĵ\emph{oj} de barbarul\emph{oj}  ki\emph{uj} imagas, ke oni povas fari belaĵon ekstere de la regul\emph{oj} fiksat\emph{aj} de ci por ci\emph{aj} inspirat\emph{oj}  o Racio. Sed ti\emph{uj} templ\emph{oj} estis al me plaĉant\emph{aj}  mi ne estis tiam lerninta cian dian arton; en ili me trovis Dion. En ili oni kantis religi\emph{ajn} kantaj\emph{ojn}  ki\emph{ujn} mi ankoraŭ memoras: ``Saluton, stelo de l'maro \ldots, Reĝino de ti\emph{uj} ĝemant\emph{aj} en nia larma valo”; aŭ plie: ``Mistika Rozo, Turo ebura, Ora Domo, Matena Stelo \ldots'' Jen, Diino, kiam mi ekmemoras ti\emph{ujn} ĉi kant\emph{ojn}  mia koro fandiĝas, mi iĝas preskaŭ malkonfesanto. Pardonu al mi tiun ĉi ridindaĵon; ci ne povas imagi la ĉarmon, kiun ti\emph{uj} barbar\emph{aj} magiist\emph{oj} enigis en ti\emph{ujn} vers\emph{ojn}  k\emph{aj} kiom kustas al mi sekvi la Prudenton tute nudan. 
        \end{otherlanguage}
    }
    \colchunk{%
        Me naskis, deino kun blua okuli, de patri barbara, che la Kimeriani bona e vertuoza, qui habitas la bordo di maro ob skura, herisata per rokaji, sem pre batata da la sturmi. On ibe konocas apene la suno; la flori esas la marala muski, la algi e la konki koloroza, quin on trovas en la fundo di la bayi solitara. La nubi ibe semblas sen koloro, e la joyo ipsa ibe esas kelke trista; ma fonteni di aqua kolda ibe venas ek la ro kaji, e la okuli di la puerini ibe esas quale ca verda. fonteni, ube, sur fundi di herbi ondi fanta, su regardas spegule la cielo.
        
        Mea preavi, tarn fore kam ni povas retroirar, esis devota a la fora navigadi, en mari quin tua Argonauti ne konocis. Me au dis, kande me esis yuna, la kanti di la voyaji polala; me esis bersita ye la memoro di la glacii flotacanta, di la mari ne buloza simila a lakto, di Finsuli habitata da uceli qui kantas ye sua hori, e qui, prenante sua flugo omni kune, obskurigas la cielo. 
        
        Sacerdoti di kulto stranjera, veninta (qua venis) de la Siriani ek Pales\-tina sorgis mea edukeso. Ta sacerdoti esis saja e santa. Li docis a me la longa rakonti pri Kronos, qua kreis la mondo, e pri lua filio, qua realigis, on dicas, voyajo sur la tero. Lia templi esas triople plu alta kam tua, ho Euritmia, e simila a foresti; ma li ne esas solida; li ruinesas ye la fino di kina-o sisacent yari; oli esas fantaziaji di barbari, qui ima ginas, ke on povas facar kelko bona, exter la reguli quin tu trasis por tua inspirati, ho Ra ciono. Ma ta templi plezas a me; me ne studiabis tua arto deala; me ibe trovis Deo. On ibe kantis kantiki quin me me moras ankore: ``Saluto, stelo di la maro, \ldots re\-jino di ti qui jemas en ica valo di lakrimi,” od: ``Rozo mistika, Turo ek ivoro, Domo ek oro, Stelo di la matino \ldots .'' Yen, deino, kande me memoras ta kanti, mea kor dio fuzesas, me divenas preske apostata. Pardonez a me ca ri dindajo. Tu ne povas imaginar la charmo quan la magiisti bar bara pozis en ta versi, e quante kustas de me sequar la raciono tote nuda. 
    }
    \colplacechunks
\end{parcolumns}

The original text of this most beautiful and sublime (I, 484) piece of French literature, the prayer on the Acropolis (la priere sur l'Acropole), is attached below to show that the Ido translation given above agrees with the original most faithfully, almost word for word. Another Ido translation of the prayer was published in 1908 (I, 485). It is very good. But its author has allowed himself deviations from the original that are not at all necessary and do not represent an improvement upon the original. 

The object of the preceding remark is to bring out clearly a general principle to be complied with particularly by students of the international language. \textbf{In rendering model prose of one language by another the original is to be followed most faithfully, word for word, going even as far as to observe the same order of the words, provided only that the forms of good style in the translating language are not infringed.} The observance of this principle is indicated especially in translations into the international language. For one of its objects is to acquaint one nation with the spirit of the language of another nation, and the spirit of a language manifests itself to a great extent in the style. The latter is therefore to be exactly copied as far as is compatible with the requirements of good style in the international language. This only restriction excludes the word for word translation of distinct idioms. They are to be rendered in some logical manner. A word for word translation of phrases that are only slightly idiomatic and therefore well intelligible in such translation is not at all objectionable. It may even serve well the purpose stated above.

\begin{center}\textbf{La Priere sur l'Acropole (Extrait)} \\ par \\ Ernest Renan \end{center} 
\begin{otherlanguage}{french} 
Je suis né, déesse aux yeux bleus, de parents barbares, chez les cimmériens bons et vertueux qui habitent au bord d'une mer sombre, hérissée de rochers, toujours battue par les orages. On y connaît à peine le soleil; les fleurs sont les mousses marines, les algues et les coquillages coloriés qu'on trouve au fond des baies solitaires. Les nuages y paraissent sans couleur, et la joie même y est un peu triste; mais des fontaines d'eau froide y sortent du rocher, et les yeux des jeunes filles y sont comme ces vertes fontaines où, sur des fonds d'herbes ondulées, se mire le ciel.

Mes pères, aussi loin que nous pouvons remonter, étaient voués aux navigations lointaines, dans des mers que tes argonautes ne connurent pas. J'entendis, quand j'étais jeune, les chansons des voyages polaires; je fus bercé au souvenir des glaces flottantes, des mers brumeuses semblables à du lait, des îles peuplées d'oiseaux qui chantent à leurs heures et qui, prenant leur volée tous ensemble, obscurcissent le ciel.

Des prêtres d'un culte étranger, venu des syriens de Palestine, prirent soin de m'élever. Ces prêtres étaient sages et saints. Ils m'apprirent les longues histoires de Cronos, qui a créé le monde, et de son fils, qui a, dit-on, accompli un voyage sur la terre. Leurs temples sont trois fois hauts comme le tien, ô Eurhythmie, et semblables à des forêts; seulement ils ne sont pas solides; ils tombent en ruine au bout de cinq ou six cents ans; ce sont des fantaisies de barbares, qui s'imaginent qu'on peut faire quelque chose de bien en dehors des règles que tu as tracées à tes inspirés ô raison. Mais ces temples me plaisaient; je n'avais pas étudié ton art divin; j'y trouvais dieu. On y chantait des cantiques dont je me souviens encore: ``Salut, étoile de la mer, \ldots reine de ceux qui gémissent en cette vallée de larmes.'' Ou bien: ``Rose mystique, tour d'ivoire, maison d'or, Étoile du matin \ldots'' Tiens, déesse, quand je me rappelle ces chants, mon coeur se fond, je deviens presque apostat. Pardonne-moi ce ridicule; tu ne peux te figurer le charme que les magiciens barbares ont mis dans ces vers, et combien il m'en coûte de suivre la raison toute nue.
\end{otherlanguage}