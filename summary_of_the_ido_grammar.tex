To understand any Ido text without previous study of the language all that is needed is a perusal of the summary, given below, which in a little more than two pages contains all essentials of the Ido grammar. One knowing Latin or one Roman language will not even have to resort to an Ido dictionary. Before commencing the actual study of the language he is thus enabled to estimate its merits by reading a few Ido texts. For this reason the summary of the Ido grammar and a few such texts have been put at the beginning of this text book and not at its end, where they would belong more properly. Those who know no other language besides English will do better to read first parts one to three, then the summary, and finally the Ido texts. 

Pronunciation.—The vowels of Ido have the continental pronunciation (see “continental” in Standard Dictionary). Of two succeeding vowels, as \textbf{au}, \textbf{eu}, \textbf{ia}, \textbf{io}, each one is to be pronounced. \textbf{C} is pronounced like ts in wits and \textbf{j} like the French j. The accent rests on the last syllable but one, and in words ending in \textbf{-ar}, \textbf{-ir}, \textbf{-or} on the last syllable. \textbf{I} and \textbf{u} preceding immediately the end vowel of a word that contains more than two vowels cannot have the accent. The latter must then be put on the vowel which precedes the \textbf{i} or the \textbf{u}. 

\begin{enumerate}
    \item The definite article is \textbf{la} for singular and plural, and \textbf{le} for the plural when no other word indicates the latter. The article \textbf{lo} before an adjective denotes the neuter of it: \textbf{lo bela}, the beautiful. A noun alone includes the indefinite article.
    \item Every word of two or more syllables that ends in \textbf{-o} is a noun; in \textbf{-a} an adjective; in \textbf{-e} an adverb or preposition; in \textbf{-u} a pronoun; in \textbf{-i} the plural of a noun or pronoun; in \textbf{-on}, \textbf{-an}, \textbf{-un}, \textbf{-in} the accusative form of a noun, adjective, pronoun, or of a plural, respectively; in \textbf{-ar}, \textbf{-ir}, \textbf{-or} the infinitive of the present, past, or future, respectively; in \textbf{-as}, \textbf{-is}, \textbf{-os} a verb in the present, past, or future, respectively; in \textbf{-ez}, \textbf{-us} a verb in the imperative or conditional, respectively. The active participle as adjective ends in \textbf{-anta}, \textbf{-inta}, \textbf{-onta} and the passive participle; in \textbf{-ata}, \textbf{-ita}, \textbf{-ota} in the present, past, and future, respectively. As noun or adverb the participle ends in \textbf{-o} or \textbf{-e}, respectively.
    \item The comparative of the adjective and adverb is formed by \textbf{plu}, more, and \textbf{min}, less; the superlative by \textbf{maxim}, most, and \textbf{minim}, least.
    \item There is no distinction of number and person with verbs. 
    \item The passive voice is formed with the passive participle and the verb \textbf{esar}, to be; the compound tenses are formed with \textbf{esar} and the participle of the past. The passive voice may also be formed with the suffix \textbf{-es}, and the compound tenses with the suffix \textbf{-ab}. These suffixes are inserted between the verbal root and the ending. \textbf{Me esas amata} or \textbf{me amesas}, I am loved. \textbf{Me esos aminta} or \textbf{me amabos}, I shall have loved. 
    \item The personal pronouns are: \textbf{me}, I; \textbf{tu}, you, thou; \textbf{il}, he; \textbf{el}, she; \textbf{ol}, it; \textbf{lu}, he, she, it; \textbf{lo}, it (referring to a fact); \textbf{ni}, we; \textbf{vi}, you (several persons); \textbf{vu}, you (one person); \textbf{li}, they; \textbf{ili}, they (masculine); \textbf{eli}, they (feminine); \textbf{oli}, they (neuter); \textbf{on}, one, we, they (indefinite); \textbf{su}, one’s self (reflexive). The possessive pronouns formed from the personal pronouns are: \textbf{mea}, my, mine; \textbf{tua}, your, yours; \textbf{lua}, his, hers, its, etc. 
    \item The demonstrative, relative, and interrogative pronouns are: \textbf{ca}, \textbf{ica}, \textbf{ilea}, \textbf{elca}, \textbf{olca}, this; \textbf{ci}, \textbf{ici}, etc., these; \textbf{ta}, \textbf{ita}, \textbf{ilta}, \textbf{elta}, \textbf{olta}, that; \textbf{ti}, \textbf{iti}, etc., those; \textbf{qua}, \textbf{ilqua}, etc., who, which, that; plural: \textbf{qui}, who, which, that; neuter: \textbf{co}, this; \textbf{to}, that; \textbf{quo}, what. 
    \item The determinatives are: \textbf{altra}, another; \textbf{ipsa}, self; \textbf{irga}, any whatever; \textbf{kelka}, some; \textbf{multa}, much; \textbf{nula}, none; \textbf{omna}, all, every; \textbf{plura}, several; \textbf{poka}, little; \textbf{quala}, what kind of; \textbf{quanta}, how much; \textbf{sama}, same; \textbf{singla}, every single; \textbf{tala}, such; \textbf{tanta}, so much; \textbf{ula}, some, a certain. As pronouns these words end in singular in \textbf{-u} and in plural in \textbf{-i}. 
    \item The numbers are: \textbf{zero}, 0; \textbf{un}, 1; \textbf{du}, 2; \textbf{tri}, 3; \textbf{quar}, 4; \textbf{kin}, 5; \textbf{sis}, 6; \textbf{sep}, 7; \textbf{ok}, 8; \textbf{non}, 9; \textbf{dek}, 10; \textbf{dek e un}, 11; \textbf{dek e du}, 12, etc.; \textbf{duadek}, 20; \textbf{triadek}, 30; \textbf{quaradek}, 40, etc.; \textbf{kinadek e sis}, 56; \textbf{sisadek e sep}, 67, etc.; \textbf{cent}, 100; \textbf{cent e un}, 101; \textbf{cent e nonadek e tri}, 193; \textbf{okacent}, 800; \textbf{mil}, 1000; \textbf{sepamil}, 7000; \textbf{milion} (accent on the o), million.
    
    The ordinal and fractional numbers are formed by the endings \textbf{-esma} and \textbf{-imo} respectively. When these numbers are nouns they end in \textbf{-esmo} and \textbf{-imo}. 
    \item The prepositions have no characteristic endings: \textbf{a} (\textbf{ad}), to; \textbf{de}, from; \textbf{di}, of; \textbf{da}, by; \textbf{ek}, from; \textbf{lor}, at the time of; \textbf{kun}, with; \textbf{pri}, about; \textbf{sen}, without; \textbf{ye}, at, in, on, with; etc. 
    \item Coordinate conjunctions are: \textbf{do}, then, therefore; \textbf{e} (\textbf{ed}), and; \textbf{ka} (\textbf{kad}) introduces a direct question; \textbf{ma}, but; \textbf{o} (\textbf{od}), or; \textbf{nam}, for; \textbf{ya}, certainly, indeed; etc.
    
    Subordinate conjunctions are: \textbf{ka} (\textbf{kad}), whether, if; \textbf{kande}, when; \textbf{ke}, that; etc. 
    \item Word Formation. New words are formed by prefixes and suffixes. The former precede the word, the latter are inserted between the root and the grammatical endings. 
\end{enumerate}
\begin{center}The prefixes are:\end{center}
\textbf{anti-}, against: \textbf{antipolo}, antipole;  \\
\textbf{arki-}, arch: \textbf{arkianjelo}, archangel; \\
\textbf{bo-}, relationship by marriage: \textbf{bopatro}, parent-in-law; \\
\textbf{des-}, contrary: \textbf{deshonoro}, dishonor; \\ 
\textbf{dis-}, dispersion: \textbf{disdonar}, to distribute; \\ 
\textbf{ex-}, former office: \textbf{exprezidanto}, ex-president; \\ 
\textbf{gala-}, gala: \textbf{galadio}, galaday; \\ 
\textbf{ge-}, persons of both sexes: \textbf{gepatri}, parents; \\ 
\textbf{mi-}, half: \textbf{mifratino}, half sister; \\ 
\textbf{mis-}, wrong action: \textbf{misguidar}, to misguide; \\ 
\textbf{ne-}, negation: \textbf{nevidebla}, invisible; \\ 
\textbf{par-}, thoroughness: \textbf{parlernar}, to learn thoroughly;
\textbf{para-}, protection against: \textbf{parapluvo}, umbrella; \\ 
\textbf{pre-}, pre-: \textbf{prematura}, premature; \\ 
\textbf{retro-}, back: \textbf{retrosendar}, to send back; \\ 
\textbf{ri-}, repetition: \textbf{rividar}, to see again; \\ 
\textbf{sen-}, less: \textbf{senmova}, motionless; \\ 
\textbf{stifa-}, step-: \textbf{stifafilio}, stepchild. 

\begin{center}The suffixes are:\end{center}
\textbf{-ach}, pejorative: \textbf{medikacho}, quack; \\ 
\textbf{-ad}, duration: \textbf{pafado}, fusillade; \\ 
\textbf{-ag}, acting with: \textbf{krucagar}, to crucify; \\ 
\textbf{-aj}, something, consisting of: \textbf{molajo}, something soft; \textbf{lignajo}, woodwork; \\ 
\textbf{-al}, relating to: \textbf{nacionala}, national; \\ 
\textbf{-an}, member of: \textbf{senatano}, senator; \\ 
\textbf{-ar}, collection of: \textbf{vortaro}, vocabulary; \\ 
\textbf{-ari}, passive participator: \textbf{pagario}, payee; \\ 
\textbf{-atr}, of the nature of: \textbf{kupratra}, copperlike; \\ 
\textbf{-e}, of the color of: \textbf{rozea}, rose colored; \\ 
\textbf{-ebl}, capable of being: \textbf{videbla}, visible; \\ 
\textbf{-ed}, certain quantity: \textbf{glasedo}, a glassful; \\ 
\textbf{-eg}, augmentative: \textbf{grandega}, immense; \\ 
\textbf{-em}, inclined to: \textbf{babilema}, talkative; \\ 
\textbf{-end}, to be done; \textbf{adjuntenda}, to be added; \\ 
\textbf{-er}, -er: \textbf{fumero}, smoker; \\ 
\textbf{-eri}, establishment: \textbf{kafeerio}, cafe; \\ 
\textbf{-es}, abstract quality, passive of a verb: \textbf{saneso}, health; \textbf{amesar}, to be loved; \\ 
\textbf{-esk}, to become, to begin: \textbf{paleskar}, to become pale; \textbf{dormeskar}, to fall asleep; \\ 
\textbf{-esm}, ordinal number; \textbf{quaresma}, fourth; \\ 
\textbf{-estr}, head of: \textbf{navestro}, captain; \\ 
\textbf{-et}, diminutive: \textbf{ridetar}, to smile; \\ 
\textbf{-ey}, place destined for: \textbf{tombeyo}, cemetery; \\ 
\textbf{-foy}, time: \textbf{trifoya}, of three times; \\ 
\textbf{-i}, domain: \textbf{episkopio}, diocese; \\ 
\textbf{-id}, offspring: \textbf{Napoleonido}, offspring of N.; \\ 
\textbf{-ier}, characterized by, bearer of: \textbf{kurasiero}, cuirassier; \textbf{pomiero}, apple tree; \\ 
\textbf{-if}, to produce; \textbf{pomifar}, to bear apples; \\ 
\textbf{-ig}, to make, to render: \textbf{purigar}, to purify; \textbf{dormigar}, to put asleep;
\textbf{-ik}, sick with: \textbf{diabetika}, sick with diabetes; \\ 
\textbf{-il}, instrument: \textbf{tranchilo}, cutting instrument; \\ 
\textbf{-im}, fraction: \textbf{duimo}, half; \textbf{quarimo}, quarter; \\ 
\textbf{-in}, female sex: \textbf{spozino}, wife; \textbf{bovino}, cow; \\ 
\textbf{-ind}, worthy: \textbf{imitinda}, worthy of imitation; \\ 
\textbf{-ism}, system, doctrine: \textbf{socialismo}, socialism; \\ 
\textbf{-ist}, professional occupation: \textbf{artisto}, artist; \\ 
\textbf{-iv}, capable of doing: \textbf{instruktiva}, instructive; \\ 
\textbf{-iz}, to provide with: \textbf{armizar}, to arm; \\ 
\textbf{-op}, distributive number: \textbf{triope}, by three; \\ 
\textbf{-opl}, multiplicative number: \textbf{dekopla}, tenfold; \\ 
\textbf{-oz}, containing: \textbf{sabloza}, sandy; \\ 
\textbf{-ul}, male sex: \textbf{spozulo}, husband; \textbf{bovulo}, ox; \\ 
\textbf{-um}, indefinite or general suffix: \textbf{krucumar}, to cross; \textbf{kolumo}, collar; \\ 
\textbf{-ur}, result of an action: \textbf{pikturo}, painting; \\ 
\textbf{-uy}, recipient: \textbf{sukruyo}, sugar box. 

Keeping in mind the preceding 11 short paragraphs of grammar and considering to some extent the 12th one about word derivation, the intelligent student will have no difficulty in understanding the following Ido texts. That Ido is intelligible almost at first sight is due to the internationality of its vocabulary which is built strictly a posteriori. This may be seen from the same Esperanto texts. They are much less comprehensible mainly because the Esperanto vocabulary is far less international and to a great extent arbitrarily selected. 

Grammatically Esperanto differs from Ido by the plural endings \textbf{-oj}, \textbf{-aj}, \textbf{-uj} of noun, adjective, and pronoun respectively; by the endings \textbf{-i} and \textbf{-u} of the infinitive and imperative respectively; by the obligatory application of the accusative (\textbf{-on}, \textbf{-ojn}, \textbf{-ajn}, \textbf{-ujn}) for the direct object and otherwise; finally by the variability of the adjective in case and number. 

The comparison of the texts will also bring out clearly the euphony of Ido and the extreme cacophony of Esperanto which is due to the abundance of the sibilants and the excessive repetition of the combinations \textbf{oj}, \textbf{aj}, \textbf{uj}, \textbf{ojn}, \textbf{ajn}, \textbf{ujn}. (The Esperanto j is pronounced like the English y in young.) 

It will be noticed that the unconnected sentences given below are not at all far fetched like the English catch: Chichester church stands in Chichester church yard. On the contrary, they are ordinary, every-day phrases, yet they sound in Esperanto like that catch in English.
