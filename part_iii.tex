\subsection*{ORDER OF THE PARTS OF A SENTENCE.}
\addcontentsline{toc}{subsection}{Order of the Parts of a Sentence: Normal order; logical position of the particle \textbf{ne}; facultative accusative; inversion; interrogation—§§ 75-77}
75. The normal order of the parts of a sentence is (a) subject, (b) predicate, (c) direct object. There being in Ido, ordinarily, no accusative form to designate the direct object, this order must, as a rule, be observed. \textbf{La navestro duktas la navo}, the captain leads the ship. \textbf{Alexandro la granda konquestis granda parto di Azia}, Alexander the Great conquered a great part of Asia. \textbf{Me komencis ca laboro ante du monati}, I commenced this work two months ago. 

Sometimes, however, it is desirable to place the object at the beginning of the sentence, as for the sake of emphasis or of imitating, in translations, other languages (see p. 17). The direct object then receives the accusative ending \textbf{-n}. \textbf{La seglin me povas vidar, ma ne la tota navo}, the sails I can see, but not the whole ship. \textbf{La mesajon me ya audas, ma a me mankas la kredo}, the message I hear indeed, but faith I miss (``Die Botschaft hor' ich wohl, allein mir fehlt der Glaube,'' Goethe in Faust). \textbf{Karegan amikon ni perdis; la perdon ni nultempe povas reparar, nam homin di tala ecelenteso on renkontras tarn rare kam stelo nove lumeskanta en la cielo}, a very dear friend we lost; the loss we can never make good, for people of such excellence one encounters as rarely as a star newly shining forth on the sky. 

Change of position is not applicable to emphasize the subject. This is accomplished through the conjunction \textbf{ya} (see note 66). 

The indirect object (dative object) may occupy any place in the sentence, but preferably it is put after the predicate before or after the direct object. \textbf{Lu ofris helpo a sua amiko}, he offered help to his friend. \textbf{El sendis a sua matro detaloza letro}, she sent her mother a detailed letter. 

An object, direct or indirect, enlarged by long complements, especially by a relative clause, is better placed after than before a shorter object, direct or indirect. \textbf{Me anuncis a mea amiko novajo qua multe gayigis lu}, I brought my friend news that cheered him very much. \textbf{Me penis donar konsolaco a mea amiko trista pro la morto di sua filio}, I tried to give consolation to my friend who was sad about the death of his child. 

The adverbial definition of the predicate, if expressed by an adverb, is not to be separated from the predicate. The short adverbs \textbf{ne}, not, and \textbf{tre}, very, always precede the predicate. \textbf{El tre dankis el}, she thanked her very much. \textbf{Ni ne konocas la voyo}, we do not know the way. In the compound tenses or in the analytic passive voice the participle must not be separated from the auxiliary verb except by a short adverb. \textbf{La urbestro esis ofte vidata en la stradi di la chef-urbo}, the mayor was often seen on the streets of the capital. 

The particle \textbf{ne}, not, always precedes the word it negatives. \textbf{Me vidis vu, ne il}, I saw you, not him. This is to be borne in mind especially when the predicate is to be negatived because in many natural languages the negation succeeds the predicate. \textbf{Me ne esas}, I am not. \textbf{Ni ne savas}, we know not (we do not know). 

When the predicate consists of the verb \textbf{esar}, to be, \textbf{divenar}, to become, etc., and an attribute, the latter must follow the former immediately. \textbf{Mea libro Esperantala divenis tro chera pro la literi acentizita}, my Esperanto book became too dear on account of the accentuated letters (not: \textbf{divenis pro la literi acentizita tro chera}). \textbf{Vu esas nejusta segun mea opiniono}, you are wrong according to my opinion (not: \textbf{esas segun mea opiniono nejusta}). 

Complements belonging to one of the parts of a sentence must not be separated from it and must succeed, not precede it. \textbf{La papiliono flugetanta cirkum la floro ofras granda plezuro a la puerulo stacanta apude}, the butterfly that flutters about the flower offers great pleasure to the boy standing nearby (see also § 93). 

An adjective or participle with complements must succeed immediately the noun it belongs to and be succeeded by the complements: \textbf{nigra makulo simila a kruco}, a black spot similar to a cross (not: \textbf{a kruco simila nigra makulo}; neither: \textbf{simila a kruco nigra makulo}, etc.). \textbf{La ramo flori fanta an la arboro en la bosketo}, the twig blooming on the tree in the grove, is different from: \textbf{la ramo an la arboro florifanta en la bosketo}, the twig on the blooming tree in the grove. 

When a predicate is to be negatived which consists of the verb \textbf{devar}, to have the duty, \textbf{mustar}, to be under the necessity, \textbf{oportar}, to be required, sometimes also \textbf{volar}, to have the will, followed by an infinitive, it is important to consider whether the duty, necessity, will requires the negation or the infinitive. In the first case the particle \textbf{ne} is to be placed before the finite verb, in the second case before the infinitive (A. Dudouy, VI, 192-195). In the sentence: you must not lie, evidently the infinitive requires the negation, the sense being: you have the duty not to lie. The logical translation, therefore, is: \textbf{vu devas ne mentiar}. Thou shalt not steal, \textbf{vu devas ne furtar}. Thou shalt not commit adultery, \textbf{vu devas ne adulterar} (II, 356). In the sentence: you have not to pay the tax, evidently the duty is to be negatived. The logical translation is therefore: \textbf{vu ne devas pagar la taxo} (VI, 195) (29). 

76. The described order of the parts of a sentence is the same in dependent clauses as in independent sentences. In relative clauses, however, the relative pronoun always stands at the beginning (with one exception, § 101). The direct or indirect object, therefore, when expressed through the relative pronoun, will always precede the other parts of the clause. For the direct object the accusative form must then be used. \textbf{La siorino, a qua il ofris sua plaso, tre dankis il}, the lady to whom he offered his seat thanked him very much. \textbf{La libro, quan me lektas, esas interesanta}, the book that I read is interesting. 

77. Inversion means a deviation from the normal order of the parts of a sentence. Inversion of the direct object is especially important because it necessitates the employment of the accusative ending for the direct object whenever it precedes the predicate and is not preceded by the subject. \textbf{Oron ed arjenton me ne havas}, gold and silver I have none. When the direct object stands after the subject, it is not necessary to use the accusative form even when it precedes the predicate. This is especially the case with the personal pronouns. \textbf{Me lu renkontris}, I met him, but \textbf{lun me renkontris}, him I met. The introduction of the facultative accusative makes inversion of the direct object possible and thus offers a sufficient measure of freedom in the order of the parts of a sentence without entailing the many difficulties which the obligatory accusative causes. Inversion must be employed in relative clauses, otherwise it is optional. \textbf{Nia kamarado, quan omnu admiris ed amis ye unesma vido, mortis tro yuna}, our comrade whom everybody admired and loved on first sight died too young. Unnecessary or too frequent use of inversion, however, is to be avoided. 

The accusative does not always remove ambiguities. In a sentence like: \textbf{me observis hundo persequar} (\textbf{perse quanta}) \textbf{kato}, I observed a dog persecuting a cat, the normal order of the parts is the best means to remove any ambiguity (30). 

Inversion is unnecessary in interrogative sentences. When there is an interrogative word, the sentence is recognized as a question through this word, but when there is no interrogative word, the particle \textbf{kad }(\textbf{ka}) introduces the sentence making a question of it. \textbf{Kande vu vizitos me}, when will you visit me? \textbf{Kad vu vizitos me morge}, will you visit me to-morrow? \textbf{Kad vu ne vidis li hiere}, did you not see them yesterday? \textbf{Me ne vidis li hiere}, I did not see them yesterday.

\subsection*{PUNCTUATION.}
\addcontentsline{toc}{subsection}{Punctuation: Period; comma; comma in rudimental clauses; other punctuation marks—§ 78}
78. Period. The use of the period in Ido is the same as in other languages. 

Comma. For the use of comma the most simple principle is: ``separate all clauses by commas'' (II, 171). Accordingly every clause is included between two commas or separated by a comma from the principal sentence. \textbf{Odiseo, balde pos ke il revenabis en sua patrio, mortigis la kurtezanti, qui molestabis lua spozino Penelopeo, tale ke la vivo tedis el}, Odysseus, shortly after he had returned to his fatherland, killed the suitors, who had molested his wife Penelope so that she was disgusted with life (IV, 531; dec. 866, VI, 52). 

The principle just mentioned which comprises the German method has been adopted for all clauses except the relative ones. For the latter, however, the French method, with which the English method almost coincides, has been recommended by the academy because the German method may cause ambiguities. The following definition of this complicated method will be quite serviceable. A relative clause is included between commas only when it forms an accessory, accidental part of the sentence, but no comma is used when the clause is an indispensable part of the sentence. \textbf{Me acensis monto, di qua la somito esis kovrita per nivo}, I ascended a mountain, the summit of which was covered with snow. \textbf{En la krepuskulo on vidas monto di qua la somita esas kovrita per nivo plu bone kam monto qua ne havas nivo}, in the twilight one sees a mountain the summit of which is covered with snow better than a mountain that has no snow (dec. 1062, VI, 211). 

Also rudimental clauses resulting from the fusing of two or more clauses into one require separation by comma, as the rudimental clauses joined by the particles \textbf{ed}, \textbf{od}, \textbf{nek}. \textbf{Nek oro, nek alta rango igas ni felica}, neither gold, nor high rank makes us happy. \textbf{Lu devos, o submisar, o demisar}, he will have, either to submit, or to resign. \textbf{Streko, plu longa kam streketo, esas, tote kontre, separilo, e la maxim grava}, a dash, longer than a hyphen, is, on the contrary, a means of separation, and the most important one (IV, 533). The last part of this sentence, `\textbf{e la maxim grava}' is equivalent to a principal sentence and the part `\textbf{plu longa kam streketo}' stands for a relative clause, so does the part `\textbf{tote kontre}.' 

One may also separate by commas circumstancial complements, especially when they are long or when they commence the sentence, obviating thereby that they be erroneously taken for the subject. \textbf{Dum la duro di l'internaciona expozo, eventos periodala koncerti}, during the international exposition periodical concerts will take place (L. Couturat, IV, 717, 718). 

Several terms of equal rank are separated by commas, but no comma is used before the last one when it is joined to the preceding ones by the particle \textbf{e}. \textbf{Klareso, simpleso e facileso esas esencala qualesi di la linguo internaciona}, clearness, simplicity, and facility are essential qualities of the international language. \textbf{La linguo internaciona devas esar klara, simpla e facila} (IV, 718), the international language must be clear, simple, and easy (31).

Other punctuation marks are: \textbf{punto-komo}, semicolon (;); \textbf{bi-punto}, colon (:); \textbf{parentezi}, parentheses ( () ); \textbf{streketo}, hyphen (-); \textbf{streko}, dash (—); \textbf{cito-hoketi}, quotation marks (« »); \textbf{klamo-punto}, exclamation sign (!); \textbf{question-punto}, question mark (?); etc. They are employed in the same manner as in English (IV, 531) (32). 

\small Remark. After the vocatives Sir, Dear Friend, My Son, etc., at the beginning of a letter different punctuation is used in different languages, in English usually a colon, in French a comma, and in German an exclamation sign. In Ido the French method is to be employed, being the most logical one (L. de Beaufront in a private communication to the author), therefore: \textbf{Sioro, \ldots}; \textbf{Kara amiko, \ldots}; \textbf{Mea filio, \ldots} \normalsize

\subsection*{ARTICLE.}
\addcontentsline{toc}{subsection}{Article: With substantive; with adjective; with proper names—§§ 79-84}
79. The definite article is used when a noun indicates all the individuals of a species. \textbf{La kato esas bestio insidiema}, the cat is an insidious beast. \textbf{La anado esas bona natero}, the duck is a good swimmer. 

The definite article is further used when a noun indicates a special individual. \textbf{Nia pretendo esas tante yusta, ke la advokato sendube ganos la proceso}, our claim is so just that the lawyer (our lawyer) will undoubtedly gain the trial: \textbf{advokato} alone would mean any lawyer. \textbf{La infanto esas malada, venigez la medicinisto}, the child is sick, fetch the doctor (our doctor); \textbf{medicinisto} alone would mean any physician. 

80. The definite article is used after the genitive of a relative pronoun. \textbf{La filantropo, di qua la jenerozeso joyigis la povri, mortis ante du semani}, the philanthropist whose generosity made the poor rejoice, died two weeks ago. \textbf{La filantropo, di qua omna civitani laudas la jenerozeso, esas tre modesta}, the philanthropist whose generosity all citizens praise, is very modest. 

81. When a noun belonging to an adjective is not expressed, the definite article is used before the adjective. \textbf{Prenez la plu granda parto e lasez a me la plu mikra}, take the bigger part and leave me the smaller one. \textbf{New York, la dek-e-quaresma} (\textbf{dio di}) \textbf{decembro 1913}, New York, December 14th, 1913. With possessive pronouns the article is to be omitted. \textbf{En nia lando la statani imaginas esar plu libera kam la statani di vua} (not \textbf{la vua}), in our country the citizens imagine they have more freedom than the citizens of yours (§ 99). 

82. The definite article is not used with proper names (cities, rivers, mountains, etc., included). The names of stars, journals, societies are regarded as proper names. Proper names followed by an adjective, too, have no article: \textbf{Scipio vinkoza}, the victorious Scipio; \textbf{Caesar glorioza}, the glorious Caesar. 

It is also preferable (not necessary) to omit the article with a proper name preceded by an adjective (Gramm. Compl., §6, note): \textbf{bela Helena}, the beautiful Helena; \textbf{saja Solon}, the wise Solon. The article is required, however, when an adjective serves to distinguish a person from an other one of the same name: \textbf{la olda Smith serchis vu, ne la yuna}, the old Smith was looking for you, not the young one (P.~de~Janko, III, 172). 

An adjective as historical epithet of a historical person is placed with the article after the proper name in conformity with international use: \textbf{Alexandro la granda}, Alexander the Great; \textbf{Ludoviko} (Louis) \textbf{la santa}, Louis the Saint; \textbf{Petro la kruela}, Peter the Cruel. 

Just as \textbf{So.~X}, Mr.~X; \textbf{doktoro~Y}, Dr.~Y, so also \textbf{urbo New York}, the city of New York, has no article, because New York is a proper name. 

The names of the days and months are not regarded as proper names. They have no capital letters and the article is used in the same manner as with other substantives: \textbf{la lasta sundio}, last Sunday; \textbf{la monato julio}, the month of July (see end of §92). 

No article is used with a title preceding a proper name: \textbf{prezidanto Lincoln}, president Lincoln; \textbf{rejo David}, king David. Vocation or profession is not to be confounded with title: \textbf{la kantisto Orpheus}, the singer Orpheus; \textbf{la astronomo Ptolemaeus}, the astronomer Ptolemy. 

83. No article is used with a substantive in an indefinite or general sense, as in proverbs. \textbf{Ditreso ne konocas lego}, necessity knows no law. \textbf{Pacienteso e tempo akordigas omna kozi}, patience and time make all things chime. 

84. There is no indefinite article, indefiniteness being indicated by the absence of any article: \textbf{stelo}, a star; \textbf{flori}, flowers. To emphasize indefiniteness the indefinite (semi indefinite) pronoun \textbf{ula}, some, any, a certain, is used and for maximum indefiniteness the pronoun \textbf{irga}, any \ldots what ever: \textbf{ula tasko}, some task; \textbf{irga maniero}, any manner what ever.

\subsection*{SUBSTANTIVE.}
\addcontentsline{toc}{subsection}{Substantive: Accusative; proper names; names of countries and peoples; title of honor; combining of nouns—§§ 85-92}
85. A substantive receives the accusative ending \textbf{-n} when as direct object it precedes the predicate and is not preceded by the subject (§ 75). \textbf{La Franca metodon uzar la komo mezagrada skolano ne povas aplikar facile}, the French method of using comma an average scholar cannot apply easily. 

There are some other instances where the accusative is required to avoid ambiguity (P. de Janko, II, 477, III, 153). \textbf{Me amas il kom amikon}, I love him as friend (he being my friend); \textbf{me amas il kom amiko}, I love him as friend (I being his friend). \textbf{Me amas il quale amikon}, I love him like a friend (as one loves a friend); \textbf{me amas il quale amiko}, I love him like a friend (as a friend does). \textbf{Adorez nulo kom deon}, worship nothing as God; \textbf{adorez nulo kom deo}, may mean: \textbf{kom deo adorez nulo}, as God (being God) worship nothing. \textbf{Me amas vu plu kam mea fraton}, I love you more than my brother (than I love my brother); \textbf{me amas vu plu kam mea frato}, could mean: \textbf{me amas vu plu, kam mea frato amas vu}, I love you more than my brother loves you (33). 

86. Proper names of every kind retain the pronunciation and spelling which they have in the language they belong to if it employs the Roman alphabet, Greek names being included herein. Diacritic signs of the language should be written and the pronunciation indicated in parentheses the best way possible. Names belonging to languages which do not employ the Roman alphabet are to be written as phonetically as possible. For this purpose a special phonetic alphabet is used, containing letters with diacritical marks, as ä, ö, ü, etc., and various digraphs, as dh, kh, th, etc.: Caesar, Gracchus, Scipio, Phryne, Goethe, Shakespeare, Molière, Büchner, Schönbrunn. 

Purely national or local expressions denoting institutions, customs, coins, measures, weights are to be considered as foreign words and treated like proper names: geisha, llama, nagayka, pound, pud, verst, dollar, cent, etc. The plural of such foreign words is best formed by adding the plural ending of Ido preceded by a hyphen: \textbf{pound-i}, \textbf{dollar-i}, \textbf{cent-i}. It is not necessary to indicate the plural when units of such coins, measures, weights are preceded by a number (II, 167; VII, 102). 

Likewise the names of cities, provinces, rivers, mountains, etc., follow the rule for proper names: Berlin, New York, Paris, Tauroggen, Texas, Holstein, Hudson, Seine, Warthe, Jungfrau, Monto McKinley, Watzmann. Only a few international mountains and rivers as well as oceans and great seas are named in an international manner: \textbf{Alpi}, \textbf{Blanka Monto}, \textbf{Danubio}, \textbf{Rheno}, \textbf{Atlantiko}, \textbf{Pacifiko}, \textbf{Mediteraneo}, \textbf{Nigra Maro}, \textbf{Norda Maro}. 

87. The names of some countries keep their original forms: Anhalt, Brabant, Paraguay, Peru, Portugal, Siam, Tirol, Würtemberg, Zanzibar, etc. The names of most countries, however, have the ending -a or -ia giving them a more international aspect in writing and a more international sound in speaking (O. Jespersen, III, 350): \textbf{Europa}, \textbf{Azia}, \textbf{Afrika}, \textbf{Amerika}, \textbf{Anglia}, \textbf{Australia}, \textbf{Austria}, \textbf{Bavaria}, \textbf{Belgia}, \textbf{Britania}, \textbf{Chinia}, \textbf{Dania}, \textbf{Egiptia}, \textbf{Francia}, \textbf{Germania}, \textbf{Grekia}, \textbf{Guinea}, \textbf{Helvetia}, \textbf{Hispania}, \textbf{India}, \textbf{Italia}, \textbf{Japonia}, \textbf{Judea}, \textbf{Kanada}, \textbf{Korea}, \textbf{Lituania}, \textbf{Mexikia} (country, while the city is \textbf{Mexiko}, VII, 355), \textbf{Norvegia}, \textbf{Oceania}, \textbf{Palestina}, \textbf{Rusia}, \textbf{Sant-Helena}, \textbf{Savoya}, \textbf{Suedia}, \textbf{Suisia}, \textbf{Turkia}, \textbf{Usa} (United States, \textbf{Unuigita Stati}), etc. 

Note the names: \textbf{Aljer} (city, while country is \textbf{Aljeria}), \textbf{Afganistan} (country), \textbf{Beludjistan} (country; people: \textbf{Afgano}, \textbf{Beludjo}), \textbf{Chili}, \textbf{Equador}, \textbf{Fidji}, \textbf{Kordilieri}, \textbf{Luxemburg} (city; the country is \textbf{Luxemburgia}; \textbf{Luxemburgani}, inhabitants of the city; \textbf{Luxemburgiani}, inhabitants of the country), \textbf{Turkestan} (country; people: \textbf{Turkestanano}), etc. (VII, 194, 195). 

88. The name of the inhabitant of a city is derived from the name of the. latter by the suffix \textbf{-an}: \textbf{Cincinnatiano}, \textbf{New Orleansano}, \textbf{Parisano}. The adjective relating to a city is formed with the suffix \textbf{-al} from the name of the city: \textbf{Liverpoolala}, \textbf{Marseilleala}. The adjective thus formed is used chiefly in relation to a thing. If the adjective refers to a person (the inhabitant) it may be derived from the name of the inhabitant by changing its ending \textbf{-o} into \textbf{-a}: \textbf{Bostonana}, \textbf{Detroitana}. 

The \textbf{i} in the names of countries ending in \textbf{-ia} belongs to the root and is therefore not to be suppressed in the derivations. In general the name of a people of a country, the same as the name of its inhabitant, is derived from the name of the country by the suffix \textbf{-an}: \textbf{Amerikano}, \textbf{Australiano}, \textbf{Braziliano}, \textbf{Chiliano}, \textbf{Europano}, \textbf{Kanadano}, \textbf{Peruano}, \textbf{Usano}, etc. 

The names of some peoples are not derived from the names of their countries, but from original adjectives. Such national names are: \textbf{Anglo}, \textbf{Arabo}, \textbf{Belgo}, \textbf{Britano}, \textbf{Burgundo} (VI, 323), \textbf{Dano}, \textbf{Franco}, \textbf{Germano}, \textbf{Goto}, \textbf{Greko}, \textbf{Hispano}, \textbf{Medo}, \textbf{Skito}, \textbf{Skoto}, \textbf{Suedo}, \textbf{Parto}, \textbf{Perso} (historical), \textbf{Polono}, \textbf{Ruso}, \textbf{Saxono}, \textbf{Tataro}, \textbf{Trako}, \textbf{Turko}, \textbf{Zuluo}, etc. (I, 652; VII, 97). 

There is no rule to determine whether the name of a people is derived from the name of the country or from an original adjective so that the dictionary has to be consulted to avoid mistakes. The form in \textbf{-ano} is always correct because it is regularly constructed. \textbf{Suediano}, Suede or inhabitant of Sweden, is correct, no matter whether \textbf{Suedo} exists or not. \textbf{Arabiano} means an inhabitant of Arabia (an Arab) although the form \textbf{Arabo} is preferable (34). 

90. In the instances of the preceding paragraph the adjectives corresponding to the countries are not derived from the names of the latter, but are original. Such adjectives are: \textbf{Angla}, \textbf{Araba}, \textbf{Belga}, \textbf{Dana}, \textbf{Franca}, \textbf{Germana}, \textbf{Greka}, \textbf{Hispana}, \textbf{Polona}, \textbf{Rusa}, \textbf{Turka}, etc. These adjectives relate to things as well as to persons (inhabitants). 

In the majority of cases, however, two kinds of adjectives are formed. When referring to things adjectives are derived from the names of the countries by the suffix \textbf{-al}, but in relation to persons (inhabitants) they are constructed from the names of the inhabitants by changing the ending \textbf{-o} into \textbf{-a}: \textbf{Afrikala}, \textbf{Afrikana}, African; \textbf{Austriala}, \textbf{Austriana}; \textbf{Aziala}, \textbf{Aziana}; \textbf{Bavariala}, \textbf{Bavariana}; \textbf{Prusiala}, \textbf{Prusiana}, etc. (35). 

English, French, etc., is expressed by: \textbf{la Angla}, \textbf{la Franca}, etc. (\textbf{linguo} being understood). 

The adjective relating to the language is the same as the one relating to the inhabitant: \textbf{la linguo Italiana}, \textbf{Hungariana}, the Italian, Hungarian language. 

91. In general substantives denote undetermined sex or both sexes: homo, man (human being); \textbf{kato}, cat (male or female); \textbf{autoro}, author (man or woman); \textbf{sekretario}, secretary (man or woman); \textbf{patro}, parent (father or mother); \textbf{frato}, brother, sister. In case of need the suffix \textbf{-ul} is used for the male sex and the suffix \textbf{-in} for the female sex: \textbf{kavalo}, horse; \textbf{kavalulo}, stallion; \textbf{kavalino}, mare; \textbf{bovo}, beef, bovine animal; \textbf{bovulo}, bull; \textbf{bovino}, cow; \textbf{patro}, parent; \textbf{patrulo}, father; \textbf{patrino}, mother; \textbf{spozo}, spouse; \textbf{spozulo}, husband; \textbf{spozino}, wife. 

Some substantives, however, denote essentially male persons, others essentially female persons: \textbf{viro}, a man; \textbf{maskulo}, a male; \textbf{femino}, a female, a woman; \textbf{matro}, mother; \textbf{amazono}, Amazon; \textbf{megero}, Megaera, termagant; \textbf{subreto}, soubrette (dec. 1090, VI, 212; dec. 1346, VII, 194). 

Both sir and lady are translated by \textbf{sioro}, if the context is sufficient to indicate whether a man or woman is meant, otherwise sir is \textbf{siorulo} and lady \textbf{siorino} (see § 7). Miss is \textbf{damzelo}. 

A married lady, and also a widow, is called \textbf{damo}, an unmarried lady \textbf{damzelo}. \textbf{Siorino} (sometimes \textbf{sioro}) includes both \textbf{damo} and \textbf{damzelo}. 

The title of honor in addressing, or speaking of, a person of high rank is \textbf{sinioro}, Monseigneur, Eminence, Excellence, Highness, Sire, Majesty, etc. The title proper, when expressed, follows this word: \textbf{sinioro episkopo}, his Eminence the Bishop; \textbf{sinioro prezidanto}, his Excellence the President. When it is necessary to express the gender, the masculine is \textbf{siniorulo} and the feminine \textbf{siniorino}. Care must be taken not to indicate the gender twice: \textbf{sinioro duko}, his Highness the duke or her Highness the Duchess; \textbf{siniorulo rejo} or \textbf{sinioro rejulo}, his Majesty the King; \textbf{siniorino rejo} or \textbf{sinioro rejino}, her Majesty the Queen. \textbf{Siniorulo rejulo} or \textbf{siniorino rejino} would contain a tautology (36). 

92. Combining of nouns. Nouns are combined in various ways. 

a) Two or more nouns are melted into one word, of the first or defining one the root alone being used, generally, but when euphony requires the complete word (see § 28): \textbf{pluvaquo}, rain water; \textbf{tablokovrilo}, table cover. 

b) Two nouns are connected by various prepositions: \textbf{artiklo pri socialismo}, article on socialism; \textbf{amo por }(\textbf{a})\textbf{ la patri}, love for the parents; \textbf{glaso por vino}, a wine glass. 

c) Nouns denoting measure or quantity are combined with the noun expressing the object measured by the preposition \textbf{de}: \textbf{tri metri de rubando}, three meters of ribbon; \textbf{glaso de aquo}, a glass of water (37). 

\small Remark. The preposition \textbf{de} after terms of quantity may be suppressed without inconvenience (Gramm. Compl., §42, p. 28): \textbf{taso teo}, a cup of tea; \textbf{metro drapo}, a meter of cloth. \normalsize

d) A noun is connected with another one indicating the material of which the first one consists, by the preposition \textbf{ek}: \textbf{domo ek petro}, stone house; \textbf{ponto ek ligno}, wooden bridge. 

e) Simple juxtaposition is employed in expressions like: \textbf{urbo New York}, the city of New York; \textbf{la monato julio}, the month of July (see § 82).

\subsection*{ADJECTIVE.}
\addcontentsline{toc}{subsection}{Adjective: Adjective as epithet; adjective used substantively; complement of an adjective—§§ 93-95}
93. An adjective as epithet of a substantive may precede or follow the latter. The first position is preferably employed with short, the second with long adjectives. The second position must be used with adjectives accompanied by long complements: \textbf{rekta lineo}, a straight line; \textbf{orta angulo}, a right angle; \textbf{triangulo nekonstruktebla}, a triangle that cannot be constructed; \textbf{enemiko nekonciliebla}, an irreconcilable enemy; \textbf{tre simpla afero} or better \textbf{afero tre simpla}, a very simple affair, but only \textbf{afero ne tre simpla pro kompliki}, an affair not very simple because of complications; \textbf{kastelo situata an la lago}, a castle situated on the lake (see § 75).

94. An adjective is used substantively only when a substantive can readily be added. \textbf{Me ne bezonas la nova parapluvo, la anciena suficos}, I do not need the new umbrella, the old one will do. \textbf{Per via pekunio e mea ni pevos entra prezar ca afero}, with your money and mine we shall be able to undertake this business (see §§ 12 and 99). 

When, however, there is no substantive to be added, the article lo before the adjective is used or some other construction. \textbf{Esas saja ligar lo agreabla kun lo utila}, it is wise to combine the agreeable with the useful (see § 54 and note 14). 

\textbf{Esas bitra mustar esar kontenta pri lo pasabla pos pose dir longatempe lo maxim bona di la speco}, it is bitter to have to be satisfied with the passable after having been for a long time in possession of the best of the kind. 

95. Complement of an adjective. An adjective may be defined by 

1) an adverb: \textbf{tre granda}, very big; \textbf{tro mikra}, too little; \textbf{extraordinare ruzoza}, extraordinarily cunning; \textbf{marveloze bela}, wonderfully beautiful; 

2) a noun and various prepositions, as \textbf{de}, \textbf{ek}, \textbf{pri}, \textbf{pro}, etc.: \textbf{malada pro febro}, sick with fever (also \textbf{malada de febro} is permissible); \textbf{trista pro la desfortuno}, sad because of the misfortune; \textbf{gaya pri la bona avizo}, glad about the good news. If the choice of the preposition be doubtful, ye is used. 

After adjectives denoting dimension, size, contents, the preposition \textbf{de} is used: \textbf{ponto longa de du angla milii}, a bridge two English miles long; \textbf{barelo plena de vino}, a barrel full of wine; \textbf{konstruktajo alta de duadek etaji}, a building twenty stories high (38). 

An adjective in the superlative is connected with the second part of the comparison by the preposition \textbf{ek}: \textbf{la maxim bela ek omni}, the most beautiful of all. The prepositions \textbf{de} and inter may also be used.

\subsection*{PRONOUN.}
\addcontentsline{toc}{subsection}{Pronoun: Personal; reflexive; etc.; distinction of \textbf{singla} and \textbf{omna}; \textbf{ipsa}, \textbf{ipse}—§§ 96-103}
96. Explanation. Personal in a grammatical sense means: referring to a definite noun or pronoun; impersonal means: referring to the contents of an infinitive or of a sentence, i.e., to a fact, or replacing the subject with predicates which by their nature can have no subject. The English pronoun it is personal and impersonal. 

The personal pronoun `it' is translated by \textbf{ol}, the impersonal `it' is translated by \textbf{lo} when referring to a fact and is not expressed at all with predicates which by their nature can have no subject or the subject of which is a dependent clause or an infinitive. \textbf{La domo ne plus esas distanta, me ja vidas ol}, the house is not far away any more, I see it already. \textbf{La formiko esas inteligenta animalo, ol }(\textbf{lu})\textbf{ sorgas en la somero por la vintro}, the ant is an intelligent animal, it provides in summer for the winter. \textbf{Il dicis omno ad el, ed el retenis ol en la memoro}, he has told her everything and she has kept it in her mind. \textbf{Nia amiko ja advenis, kad vu savas lo? \ldots Lo joyigas me}, our friend has already arrived, do you know it? \ldots It affords me joy. \textbf{Nivas}, it snows; \textbf{pluvas}, it rains; \textbf{esas kolda}, it is cold; \textbf{esas varma}, it is warm; \textbf{decas}, it behooves; \textbf{konvenas}, it is befitting; \textbf{suficas}, it suffices. \textbf{Mentiar esas shaminda}, it is shameful to lie. \textbf{Esas tre trista, ke il mustis livar}, it is very sad that he had to leave us. 

\textbf{Lo} may be replaced by \textbf{to} (and also by \textbf{co}). \textbf{Kad vu savas, ke el ja departis? \ldots Me savas lo }(\textbf{to}, \textbf{co}), do you know that she has departed already? \ldots I know it (that) (39). 

97. The reflexive pronoun of the 3rd person is \textbf{su} for all genders and numbers. It is used when it relates to the subject of the verb of which it forms a complement (direct or indirect object). \textbf{La soldati defensis su brave}, the soldiers defended themselves bravely. \textbf{Egoisto sorgas nur por su}, an egotist cares only for himself. 

1) \textbf{Petro pregis Karlo defensar su}, Peter asked Charles to defend himself. 
2) \textbf{Petro pregis Karlo defensar lu}, Peter asked Charles to defend him. 
3) \textbf{Petro promisis a Karlo defensar lu}, Peter promised Charles to defend him. 

In 1) \textbf{su} relates to \textbf{Karlo} who is the subject of \textbf{defensar}. In 2) \textbf{lu} relates to \textbf{Petro} who is not the subject of \textbf{defensar}. In 3) \textbf{lu} relates to \textbf{Karlo} who is not the subject of \textbf{defensar}. 

98. Regarding the possessive pronouns of the 3rd person, \textbf{lua} (\textbf{ilua}, \textbf{elua}, \textbf{olua}), \textbf{lia} (\textbf{ilia}, \textbf{elia}, \textbf{olia}), \textbf{sua}, two factors must be kept apart, the possessing person or thing, the possessor, and the possessed person or thing. The second factor decides in which number and case \textbf{lua}, \textbf{lia}, \textbf{sua} are to be taken (the plural form \textbf{le lua}, or \textbf{la lui} may be needed, when a noun in plural is not expressed but understood, and the accusative form may be required when the possessive pronoun as object precedes the predicate), but does not decide which of them is to be taken. The first factor decides which of these words must be used. 

a) \textbf{Lua} is used when there is one possessor. If it be desirable to indicate the gender of the possessor, the unabbreviated forms \textbf{ilua}, \textbf{elua}, \textbf{olua} are employed as the case may be. 

b) \textbf{Lia} is used when there are two or more possessors. If it be desirable to indicate the gender of the possessors, the unabbreviated forms \textbf{ilia}, \textbf{elia}, \textbf{olia} are employed. 

c) \textbf{Sua} is used reflexively, i.e., when the possessor (possessors) is (are) the subject of the verb of which the possessed person or thing is the complement (direct or indirect object). 

According to this explanation it is easy to see that \textbf{sua} would be wrong in the following sentence: \textbf{Kande lua }(\textbf{elua})\textbf{ matro reprimandis lu }(\textbf{el})\textbf{, Elsa ploris}, when her mother scolded her, Elsa cried. Elsa is the possessor but not the subject of \textbf{reprimandis}, \textbf{sua} is therefore not permissible. \textbf{Matro} is the subject of \textbf{reprimandis}, but being the possessed person \textbf{matro} does not decide at all which of the two must be taken, \textbf{lua} or \textbf{sua}. 

1) \textbf{Petro deziras pruntar de Karlo lua parapluvo}, Peter desires to borrow from Charles his umbrella. Peter is the subject of \textbf{pruntar}, but not the possessor of the umbrella, therefore \textbf{lua}, not \textbf{sua}. 

2) \textbf{Karlo pregas Petro retrodonar lua parapluvo a lu}, Charles asks Peter to give him back his umbrella. Peter is the subject of \textbf{retrodonar}, but not the possessor of the umbrella, therefore \textbf{lua}, not \textbf{sua}. 

3) \textbf{Petro pregas Karlo prestar a lu sua parapluvo}, Peter asks Charles to lend him his umbrella. Charles is the subject of \textbf{prestar} and the possessor of the umbrella, therefore \textbf{sua}, not \textbf{lua}. 

4) \textbf{Karlo saveskis, ke lua amiko Petro mortis}, Charles learned that his friend Peter died. Peter is the subject of \textbf{mortis}, but not the possessor; Charles is the possessor, but not the subject of \textbf{mortis}, therefore \textbf{lua}, not \textbf{sua} (40). 

99. The possessive pronouns imply the definite article and have the same sense when they are isolated as when they are joined to a substantive. \textbf{Mea} equals \textbf{la \ldots di me} (dec. 950, VI, 161): \textbf{mea amiko} = \textbf{la amiko di me}. Mine, thine, etc., is not \textbf{la mea}, \textbf{la tua}, etc., but only \textbf{mea}, \textbf{tua}, etc. To indicate indefiniteness, therefore, \textbf{di me} has to be used, for \textbf{mea} includes the article: \textbf{amiko di me}, a friend of mine. 

\small Remark. The expression some of our friends implying indefiniteness cannot be translated by \textbf{kelka nia amiki}, but by \textbf{kelka amiki di ni} or still better by \textbf{kelki de nia amiki} (111,522). \normalsize

When indefiniteness is to be indicated in a sentence such as this: I have lost my book, lend me one of yours, \textbf{un de \ldots} is to be used: \textbf{me perdis mea libro, prestez a me un }(\textbf{ula})\textbf{ de vui }(\textbf{de vua libri});\textbf{ prestez vua}, would mean: \textbf{prestez la }(\textbf{libro})\textbf{ di vu}, lend me yours (i.e., indefiniteness would not be indicated) (IV, 146). 

100. The neuter of the substantivized possessive pronoun is expressed by the suffix \textbf{-ajo}: \textbf{la meajo}, mine; \textbf{la tuajo}, thine (III, 146). 

101. After the genitive of a relative pronoun the definite article is used differently from the English (§§ 76, 80). When such a genitive has as complement a substantive with a preposition, this complement introduces the relative clause and the genitive follows. \textbf{Mea amiko, kun la kon sento di qua me entraprezis ca afero, esas nun nekontenta}, my friend, with whose consent I have undertaken this affair, is now dissatisfied. \textbf{Mea amiki, por la bonstando di qui me sakrifikis omno, abandonis me}, my friends for whose welfare I have sacrificed everything, have forsaken me. \textbf{Kad vu vi das ta fora navo, ek la kamentubo di qua densa fumuro levas su (levesas)?} Do you see that distant ship from the chimney of which dense smoke arises? 

102. The pronouns \textbf{irga}, \textbf{irgu}, \textbf{irgo}, and the adverb \textbf{irge} are used alone or with a succeeding relative pronoun or conjunction. In the latter case they correspond to the English ever, so, soever attached to a relative word. \textbf{Ne esez tante ociema, facez irgo}, do not be so lazy, do anything whatsoever. \textbf{Se vu ne savez quale, facez irge}, if you do not know how, do it in any manner whatever. \textbf{Irge quanta pekunion l'avaro havas, lu deziras plu multa}, no matter how much money the miser has, he wants more. \textbf{Irge quan lu renkon tris, lu salutis}, whomever he met, he greeted. \textbf{Irge quante mikra la dona, ol esos bonvenanta}, however small the gift, it will be welcome. 

\textbf{Ula}, differently from \textbf{irga}, does not indicate complete indefiniteness, but semi-indefiniteness. It is therefore appropriate for translating the determinatives one, a certain. \textbf{Me recevis ca mesajo ula dio en la lasta monato}, I received this message one day in the last month. \textbf{Me havis ula pre sento di la venonta desfortunajo}, I had a certain presentiment of the coming misfortune. \textbf{Ula viro decensis de Jerusalem a Jericho}, a certain man went down from Jerusalem to Jericho (see § 30; II, 349; III, 28). 

\textbf{Multa}, much, denotes a large quantity, the opposite \textbf{poka}, little (\textbf{poki}, few), a small quantity, and \textbf{kelka}, some (\textbf{kelki}, a few), an undetermined, but not large quantity. \textbf{Poka homi esas kontenta pri sua posedaji}, few people are satisfied with what they possess. \textbf{Ne omna homi perisis en ca dizastro, kelki salvesis}, not all the people perished in this disaster, a few were saved (III, 23, 593). 

Attention is to be paid to the difference between \textbf{singla} and \textbf{omna}, both of which are translated with each, every. \textbf{Omna} is collective, \textbf{singla} distributive (II, 665). \textbf{Omna} generalizes, \textbf{singla} individualizes, the former refers to the species in its uniformity, the latter to the peculiarity which distinguishes the individuals of the same species. \textbf{Singla} implies also the idea of a restricted number. \textbf{Omna} = all; \textbf{singla} = every single. \textbf{Singla} may be replaced by \textbf{un po un} or by \textbf{unopa}. \textbf{Nokte omna ucelo aspektas nigra}, at night every bird looks black. \textbf{Singla ucelo amas sua propra nesto}, every bird loves its own nest (a chaque oiseau son nid parait beau). \textbf{Omna testo juris dicar la verajo, e singlu naracis la evento diverse}, every witness swore to tell the truth, and each one related the event differently. \textbf{Omna osto kontenas kalko}, every bone contains chalk. \textbf{La explozo ruptis preske singla osto di lua korpo}, the explosion broke almost every bone of his body (idea of a restricted number). 

\textbf{Omna singla} is an unjustified pleonasm, \textbf{singla} alone being fully sufficient to translate the English every single (VI, 343). \textbf{Omna} may be used in singular as well as \textbf{singla} in the plural (II, 665, dec. 595; IV, 562). \textbf{Omna soldato esis brava} = \textbf{omna soldati esis brava}, every soldier was brave, all soldiers were brave. \textbf{Singla soldato} or \textbf{singla soldati havis fusilo}, each soldier or every single soldier had a gun. 

The English word single is not \textbf{singla}, but \textbf{sola}, \textbf{izolita}, \textbf{unika} (V, 30). \textbf{Homo izolita }(\textbf{sola})\textbf{ iras sur la strado}, a single man walks over the street. \textbf{Izolita kolono esas testo di desaparinta splendideso}, a single (only one) column gives evidence of splendor gone (Uhland, Des Sängers Fluch). 

The pronoun \textbf{ipsa} is frequently used in adverbial form, \textbf{ipse}. This adverb, just as the participial adverb (§ 120) always refers to the subject of the sentence and is sometimes clearer than the adjective form, \textbf{ipsa}. \textbf{La mastro inspektis ipse la domo}, the master inspected the house himself (or personally). If \textbf{ipsa} were used in this sentence, it could be referred to \textbf{domo}, \textbf{la domo ipsa}. The adjective \textbf{ipsa} is always to be placed nearest the word it belongs to: \textbf{la mastro ipsa inspektis la domo}. The same holds good with \textbf{omna}. \textbf{Li omni admiras splendida flori}, they all admire splendid flowers. This sentence must not have the order: \textbf{li admiras omna splendida flori}, for in this construction \textbf{omna} refers necessarily to \textbf{flori}.

By itself is not to be translated by \textbf{ipse}, but by \textbf{spontane} (V, 96, 97).

103. Of several succeeding pronouns only one has the substantive form, while the others are to be treated as its adjectives: \textbf{omna to} (not \textbf{omno to}), all this; \textbf{nula altra}, nothing else; \textbf{irgu altra}, somebody else; \textbf{nuli altra} (also \textbf{nula altri}), no others; \textbf{omna ti qui} (not \textbf{omni ti qui}), all those who (II, 721, III, 147).

\subsection*{ADVERB.}
\addcontentsline{toc}{subsection}{Adverb: Original; derived; distinction of \textbf{plus} and \textbf{plu}; of \textbf{kom} and \textbf{quale}; adverbs as prepositions—§§ 104-111}
104. The adverbs may be divided into original and derived adverbs. The latter are obtained from other words either by changing their grammatical ending into \textbf{-e}, as with substantives, adjectives, and verbs, or by adding an \textbf{-e} when there is no grammatical ending, as with prepositions. The original adverbs have no uniform ending. Many of them, however, do terminate in \textbf{-e}, but this \textbf{-e} is original, it has not been obtained by changing another ending or by addition.

105. Original adverbs not ending in \textbf{-e} are:

\begin{tabular}{l l}
\textbf{forsan}, perhaps; & \textbf{nun}, now; \\
\textbf{ja}, already; & \textbf{nur}, only; \\
\textbf{jus}, just a moment ago (41); & \textbf{olim}, once, formerly; \\
\textbf{kam}, than, as (I, 218) (42); & \textbf{plu}, more (comparative); \\
\textbf{kom}, as (not comparing, but attributive); & \textbf{plus}, more (additive); \\
\textbf{maxim}, most; & \textbf{quik}, at once, directly; \\
\textbf{mem}, even, still; & \textbf{retro}, back, backwards; \\
\textbf{min}, less (comparative); & \textbf{sat}, enough; \\
\textbf{minim}, least; & \textbf{tam}, so, so much (41); \\
\textbf{minus}, less (subtractive); & \textbf{tre}, very\footnotemark[1] (see suffix -eg); \\
\textbf{ne}, not (negativing a word; contrary: the word); & \textbf{ya}, indeed, certainly; \\
\textbf{no}, no (contrary: yes); & \textbf{yes}, yes.
\end{tabular}
\footnotetext[1]{\textbf{Tre}, being monosyllabic, may be treated in this paragraph.}

\textbf{Ne esas merito en ne aspektar nekontenta e malhumoroza, kande la vivo ne esas desagreabla; nur la maxim bona homi prizentas ad omnu serena mieno mem en desoportunesi}, there is no merit in not looking discontented and morose when life is not disagreeable; only the best present a serene countenance to everybody even in hardships.

The adverbs \textbf{maxim}, \textbf{min}, \textbf{minim}, \textbf{plu}, \textbf{plus}, \textbf{tam}, \textbf{tro} (II, 667) sometimes assume the ending \textbf{-e}: \textbf{maxime}, \textbf{mine}, \textbf{minime}, \textbf{plue}, \textbf{pluse}, etc. The shorter forms are used when the adverbs are followed by an adjective or by an adverb which they modify (II, 603, 604): \textbf{maxim bona}, \textbf{min bela}, \textbf{minim alta}, \textbf{plu frue}, \textbf{tro tarde}, etc. The longer forms are used when the adverbs are not followed by a word they modify. \textbf{Lu pensis, ke lu donabis a me un cent troe, ma me trovis du centi minus en mea burso}, he thought he had given me one cent too much, but I found two cents less in my purse. \textbf{Lu laboras marime}, he works most. \textbf{El lu amas mineme}, she loves him least.

With verbs the long form is used when the adverb follows the verb, as in the examples just cited, and the short form before the verb. \textbf{El tro amas lu. Lu plu ocias kam laboras}, he is more idle than working. In both cases the short form with the addition of an appropriate adjective or adverb may be used instead of the long or short form alone. \textbf{Me donis a lu un dollar tro multa}, I gave him a dollar too much. \textbf{La laboras maxim multe}, he works most. \textbf{Lu tro multe mentias}, he lies too much.

The adverbs \textbf{plu}, \textbf{tro}, and sat appear in the short form before adjectives and adverbs, but with substantives \textbf{multa} must be added: \textbf{plu multa aquo}, more water; \textbf{tro multa homi}, too many people; \textbf{sat multa pekunio}, enough money. With verbs the adverb \textbf{tro} is used alone or in combination with \textbf{multe}, but \textbf{plu} and sat preferably with \textbf{multe}. \textbf{Me manjis sat multe}, I ate enough. \textbf{Drinkez plu multe}, drink more.

Note the expression: \textbf{un glaso tro multe}, one glass too much.

\textbf{Plus} (\textbf{minus}) does not belong to the class of original adverbs not ending in \textbf{-e}, but to the class of derived adverbs, for the actual adverb is pluse (minuse) which is formed from the preposition plus employed in mathematics (VI, 488). But because of its relation to \textbf{plu}, it has been treated in this paragraph. Where \textbf{plus} occurs as adverb it is merely an abbreviation of \textbf{pluse}. The short form is contained in: \textbf{ne plus}, no more, when the adverb precedes the verb. \textbf{Il ne plus vivas}, he does not live any more. After the verb the longer form \textbf{pluse} is used.

Attention is to be paid to the difference between \textbf{plus} (\textbf{minus}) and \textbf{plu} (\textbf{min}; \textbf{pluse}, \textbf{minuse}, \textbf{plue}, \textbf{mine}, when the adverb stands alone or after the verb). The former is additive (subtractive), the latter is comparative (II, 66, 667). \textbf{Pluse} corresponds to the English moreover. \textbf{Me mustis pagar plue }(\textbf{plu multe}), I had to pay more (to add something to the price). \textbf{Il pagis a me un dollar minuse}, he paid me one dollar less (he took off one dollar from the price); \textbf{il pagis un dollar mine}, he paid one dollar less (than he should have paid). \textbf{Vino e kafeo pagesas pluse}, for wine and coffee one pays more (one has to add something to the price); \textbf{plue} would mean dearer (than before, than somewhere else). One inquiring about the price of one thing and then about that of another thing may receive this answer: \textbf{ica kustas plue }(\textbf{plu multe}), this costs more (than the other thing). A seller wishing to be obliging may say: \textbf{me donas ico pluse}, I give this besides (for the same price).

The distinction of \textbf{plu} and \textbf{plus} is important in the expressions \textbf{ne plu} and \textbf{ne plus}. \textbf{Me ne plu pagos}, \textbf{me ne pagos plue} (\textbf{plu multe}), I shall not pay more (than I did). \textbf{Me ne plus pagos} or \textbf{me ne pagos pluse}, I shall not pay any more (from now on I shall not pay). \textbf{Me ne plus kantos}, \textbf{me ne kantos pluse}, I shall sing no more.

Note the expression: \textbf{un glaso pluse} (\textbf{minuse}), one more glass (one glass less).

The adjectives \textbf{plus} (\textbf{mina}) and \textbf{plusa} (\textbf{minusa}) differ from each other in the same manner as the adverbs. \textbf{Me donis ad il plus sumo}, I gave him a bigger sum; \textbf{me donis ad il plusa sumo}, I gave him an additional sum. \textbf{Kad vu deziras plua }(\textbf{plu multa})\textbf{ exempleri}, do you wish more copies (than I gave you)? \textbf{Plusa exempleri} would mean: other, additional copies.

The noun \textbf{pluso} means continuation: \textbf{pluso sequos}, to be continued (VI, 528).

106. Original adverbs ending in \textbf{-e} are:

\begin{tabular}{l l}
\textbf{admaxime}, at most; & \textbf{morge}, to-morrow; \\
\textbf{adminime}, at least; & \textbf{posmorge}, the day after to-morrow; \\
\textbf{anke}, also; & \textbf{ofte}, often; \\
\textbf{ankore}, still, yet; & \textbf{passable}, passably, pretty; \\
\textbf{apene}, hardly, scarcely; & \textbf{preske}, almost; \\
\textbf{balde}, soon; & \textbf{quaze}, as it were, so to speak; \\
\textbf{erste}, not before, only; & \textbf{sempre}, always; \\
\textbf{hiere}, yesterday; & \textbf{supre}, on top; \\
\textbf{prehiere}, the day before yesterday; & \textbf{ube}, where; \\
\textbf{hike}, here; & \textbf{altrube}, somewhere else; \\
\textbf{ibe}, there; & \textbf{irgube}, anywhere; \\
\textbf{infre}, below, at the bottom; & \textbf{nulube}, nowhere; \\
\textbf{irgekande}, no matter when; & \textbf{omnube}, everywhere; \\
\textbf{kande}, when (I, 218); & \textbf{ulube}, somewhere.
\end{tabular}

\textbf{La maxim proxima numero aparos erste komence di julio}, the next number will not appear before beginning of July. \textbf{Charmanta rideto sempre esis sur elua vizajo; ol nultempe aparis neafabla od acerba}, a charming smile was always on her face; it never appeared inaffable or sour.

\textbf{Hiere marine}, \textbf{hiere vespere}, yesterday morning, yesterday evening; \textbf{morge matine}, \textbf{matinmorge}, to-morrow morning; \textbf{morge vespere}, to-morrow evening (see § 56, a, b).

Still, yet with a comparative is not to be translated by \textbf{ankore}, but by \textbf{mem}: \textbf{mem plu mala}, still worse; \textbf{mem plu bona}, better yet (III, 486; dec. 64, IV, 502).

\textbf{Pasable}, passably, tolerably, pretty, indicates a medium degree, the lowest still acceptable. \textbf{Cadie la vetero esas pasable varma por marbalno, ma hiere ol ne esis sat varma}, to-day the weather is pretty warm for a sea bath, but yesterday it was not warm enough. (II, 531; III, 205, 407, 529) (43).

107. Derived adverbs occurring frequently are the following:

\begin{center}I.—Adverbs of Time:\end{center}
\begin{tabular}{l l}
\textbf{antee}, before; & \textbf{unfoye}, one time, once; \\
\textbf{cadie}, to-day; & \textbf{jorne}, in day time; \\
\textbf{omnadie}, daily; & \textbf{lore}, then, at that time; \\
\textbf{dume}, meanwhile; & \textbf{lore \ldots  lore}, now \ldots now; \\
\textbf{frue}, early; & \textbf{marine}, in the morning; \\
\textbf{irgatempe}, at any time whatever; & \textbf{camatine}, this morning; \\
\textbf{kelkatempe}, sometimes; & \textbf{nokte}, at night; \\
\textbf{nultempe}, never; & \textbf{omnamonate}, monthly; \\
\textbf{samtempe}, at the same time; & \textbf{omnasemane}, weekly; \\
\textbf{ultempe}, at some time; & \textbf{omnayare}, yearly; \\
\textbf{kelkafoye}, some time; & \textbf{tarde}, late; \\
\textbf{multafoye}, many times; & \textbf{vespere}, in the evening; \\
\textbf{omnafoye}, every time; & \textbf{cavespere}, this evening. \\
\textbf{plurfoye}, several times; &
\end{tabular}

\textbf{Nultempe kosmetikala pudri desornis elua bela, docla, infantara vizajo}, never did cosmetic powders defile her beautiful, sweet, childlike face.

\begin{center}II.—Adverbs of Place:\end{center}
\begin{tabular}{l l}
\textbf{admonte}, up-stream; & \textbf{irgaloke}, in any place whatever; \\
\textbf{advale}, down-stream; & \textbf{kelkaloke}, in some place; \\
\textbf{avane}, in front; & \textbf{nulloke}, nowhere; \\
\textbf{apude}, nearby; & \textbf{omnaloke}, everywhere; \\
\textbf{cirkume}, around; & \textbf{ulloke}, in some place; \\
\textbf{cise}, on this side; & \textbf{proxime}, nearby; \\
\textbf{dextre}, to the right; & \textbf{sinistre}, to the left; \\
\textbf{dope}, behind; & \textbf{sube}, below; \\
\textbf{fore}, away, far away; & \textbf{supere}, above; \\
\textbf{exeter}, outside; & \textbf{sure}, above (and in contact with); \\
\textbf{interne}, inside; & \textbf{transe}, on the other side. \\
\textbf{altraloke}, in another place; & 
\end{tabular}

\textbf{Aernavi ofte flugis super la rivero, lore admonte, lore advale}, airships often flew over the river, now up-stream, now down-stream.

\small Remark. The exact meaning of the adverbs \textbf{infre}, \textbf{supre}, \textbf{supere}, \textbf{sure}, etc. is explained in note 62  to § 136. \normalsize

\begin{center}III.—Adverbs of Quantity:\end{center}
\begin{tabular}{l l}
\textbf{kelke}, \textbf{kelkete}, somewhat, a little; & \textbf{poke}, little; \\
\textbf{multe}, much; & \textbf{quante}, how much; \\
\textbf{minus}, less; & \textbf{irgequante}, no matter how much, in any quantity whatever; \\
\textbf{pluse}, more; & \textbf{tante}, so much.
\end{tabular}

\textbf{Poke} is the opposite of \textbf{multe} just as \textbf{poka} of \textbf{multa} (see § 30).
The correlative of \textbf{tante} is \textbf{ke}, not \textbf{kam} which is the correlative of \textbf{tam}. \textbf{Lu esas tam habila kam vu}, he is as dexterous as you. \textbf{Lu esas tante habila, ke lu povas facar omno en la domo}, he is so dexterous that he can do everything in the house (44) (45). \textbf{El esis tam sereniganta kam la sennuba cielo ed elua karnaciono tente pura, ke omna homini envidiis el}, she was as cheering as the cloudless sky and her complexion so pure that all women envied her.

\begin{center}IV.—Adverbs of Manner or Modality:\end{center}
\begin{tabular}{l l}
\textbf{adverse}, to be sure, it is true; & \textbf{kune}, together (42), (46); \\
\textbf{afranke}, post free; & \textbf{memore}, by heart; \\
\textbf{altravorte}, in other words; & \textbf{nome}, namely; \\
\textbf{aparte}, apart; & \textbf{nule}, in no manner; \\
\textbf{cetere}, besides, moreover; & \textbf{omne}, in every manner; \\
\textbf{cirkume}, approximately; & \textbf{precipue}, principally; \\
\textbf{entote}, on the whole; & \textbf{pokope}, by and by; \\
\textbf{fine}, finally; & \textbf{prefere}, preferably (47); \\
\textbf{gratuite}, for nothing; & \textbf{proxime}, nearly, approximately; \\
\textbf{irge}, in any manner whatever; & \textbf{quale}, as, like; \\
\textbf{irgemaniere}, in any manner whatever; & \textbf{segunvole}, at will; \\
\textbf{irgequale}, no matter how; & \textbf{seque}, then, (here)after; \\
\textbf{intence}, intentionally; & \textbf{sole}, only, in a unique manner; \\
\textbf{itere}, again, anew; & \textbf{tale}, suchwise, so; \\
\textbf{komprenende}, of course; & \textbf{vane}, in vain; \\
\textbf{konseque}, consequently; & \textbf{volunte}, willingly. \\
\textbf{kontree}, on the contrary; &
\end{tabular}

\textbf{Advere il venus, ma il facis nulo}, it is true, he came, but he did nothing. \textbf{Il venis, advere sen instrumenti}, he came, but, to be sure, without instruments. (E. Küppers, III, 24). \textbf{Lu furnisis la solvo sole vera}, he furnished the only true solution.

108. Note the adverbial phrases: \textbf{maxim \ldots posible}, as much as possible; \textbf{minim }(\textbf{minime})\textbf{ \ldots posible}, as little as possible; \textbf{maxim frue posible}, as early as possible; \textbf{minime bruise posible}, as little noise as possible. As much as possible is also \textbf{tam \ldots kam posible} (§ 34).

\textbf{Sempre plu}, more and more; \textbf{sempre min}, less and less; \textbf{plu o min multe}, more or less (VII, 378, note 1).

\textbf{Irgaloke} and \textbf{irgatempe} are indefinite adverbs of place and time respectively, while \textbf{irgube}, \textbf{irgekande}, \textbf{irgequale}, \textbf{irgequante}, are rather conjections than adverbs. \textbf{Agez irgamaniere}, act in any manner whatever. \textbf{Irgequale me agas, il eseas nekontenta}, no matter how I act, he is dissatisfied.

109. \textbf{Kom} and \textbf{quale}. The distinction of the adverbs \textbf{kom} and \textbf{quale} offers some difficulty. \textbf{Quale} introduces a comparison, \textbf{kom} is not comparing, but adds an attribute. \textbf{Quale} is to be used whenever \textbf{tam \ldots kam} or \textbf{tale \ldots quale} could be used instead, \textbf{kom} whenever the particle in question could be replaced by \textbf{esar} in a participial construction or by \textbf{esar} in a dependent clause. \textbf{Kom} is further used with verbs requiring in some languages the double accusative in the active voice and the double nominative in the passive voice (§ 116). \textbf{Vu trompesis kom dupo}, you have been deceived as a dupe (\textbf{esante dupo vu trompesis}, being a dupe you were deceived).\textbf{Vu trompesis quale dupoe}, you have been deceived like a dupe (comparison between \textbf{vu} and \textbf{dupo}, \textbf{vu trompesis tale quale dupo}). \textbf{Quale patro lu amas la filii}, he loves the children as a parent (like a parent does, he is not the father of the children). \textbf{Kom patro lu amas la filii}, as parent he loves the children (he is the parent of the children). \textbf{Me amas il quale mea fraton} (see § 85), I love him as my brother, as I love my brother (he is not my brother). \textbf{Me amas il kom mea fraton}, I love him as my brother (he is my brother). \textbf{Kom monarko lu sola deklaris la milito}, as monarch he alone declared the war (he is the monarch). \textbf{Quale monarko lu sola deklaris la milito}, he alone declared the war like a monarch (he is not the monarch). \textbf{Il esis aceptata quale rejo}, he was received like a king (comparison, he is not the king). \textbf{Il esis aceptata kom rejo}, he was received as king (he is the king). \textbf{La aktorino plezis kom Julia}, the actress pleased as Julia (she represented Julia on the stage). \textbf{La aktorino plezis quale Julia}, the actress pleased as Julia (comparison between this actress and another one named Julia). \textbf{Me judikas ca homino kom bela}, I consider this woman beautiful; \textbf{me judikas ca homino bela} could mean: I judge (give judgment about) this beautiful woman, for it does not differ from: \textbf{me judikas ca bela homino}. \textbf{Lejera quale feo, pezanta quale hoplito}, light like a fairy, ponderous like a hoplite. (C. A. Janotta, I, 708; P. de Janko, II, 395, 396, III, 153).

In sentences such as: the soldiers showed themselves courageous, she feels offended, the particle \textbf{kom} may be used, but it is better omitted as no unclearness can arise without it: \textbf{la soldati montris su kurajoza, el su sentas ofensita}.

110. The phrase: I feel (or am) warm, cold (French: j'ai chaud, froid; German: mir ist warm, kalt) is expressed by the reflexive verb \textbf{sentar su}: \textbf{me sentas me varma}, \textbf{kolda} (or \textbf{kom varma}, \textbf{kom kolda}) (dec. 667, IV, 691; see also III, 226, 227, where the construction: \textbf{me esas varme}, \textbf{kolde}, has been proposed and the adverb justified as defining the verb \textbf{esar}).

111. Many derived adverbs are used prepositionally: \textbf{ecepte}, except; \textbf{kompare}, in comparison with; \textbf{koncerne}, concerning; \textbf{relate}, relating to; \textbf{admonte}, up the stream from, above; \textbf{advale}, down the stream from, below (IV, 33); \textbf{latere}, at the side of; \textbf{dextre}, to the right of; \textbf{sinistre}, to the left; \textbf{okazione}, at the occasion of; \textbf{meze}, in the midst of; vice, instead of; \textbf{danke}, thanks to; etc. \textbf{Danke vua helpo}, thanks to your help; \textbf{latere di la lago}, at the side of the lake (§ 135).

\subsection*{VERB.}
\addcontentsline{toc}{subsection}{Verb: Mixed verbs; invariability of the direct object; predicate consisting of two parts; infinitive after prepositions; participle replacing clause; passive voice for ``false'' reflexive verbs; reciprocity—§§ 112-124}
112. Some verbs may be called mixed verbs because they are used both transitively and intransitively. No ambiguity can arise since the first case the verbs are accompanied by a direct object. On the other hand a frequent source of difficulties is removed through such use. Such verbs are: \textbf{cesar}, to cease; \textbf{chanjar}, to change; \textbf{durar}, to last, to continue; \textbf{finar}, to end; \textbf{komencar}, to commence; \textbf{kommunikar}, to communicate; \textbf{movar}, to move; \textbf{pendar}, to hang; \textbf{turnar}, to turn, etc. \textbf{Lua rolo cesas}, \textbf{komencas}, his part ceases, commences; \textbf{lu cesas}, \textbf{komencas sua rolo}, he ceases, commences his part. \textbf{Kelka fishi chanjas sua koloro}, some fishes change their color; \textbf{la koloro di ca fisho chanjis en un instanto}, the color of this fish changed in one moment (48).

The verb \textbf{laborar}, to work, is usually intransitive. Sometimes, however it may have also a transitive sense: \textbf{laborar la tero}, \textbf{fero}, \textbf{ligno}, to work the earth, iron, wood (IV, 66, dec. 260). (49) (50)

113. A verb cannot have two direct objects, not even alternately. The person to whom, and the subject about which, knowledge is imparted cannot be direct object of the same verb, neither at the same time nor at different times. The verb \textbf{docar}, to teach, has as object the subject taught. It can, therefore, never have as object the person taught. The latter is joined to the verb by a preposition, the construction being always \textbf{docar ulo ad ulu}, no matter whether one of the two, either the subject taught or the person taught, is omitted or not. The verb \textbf{instruktar}, to instruct, has as a direct object the person instructed. The subject about which instruction is given can, therefore, never become object of the verb, but it is joined to it by a preposition, the construction being \textbf{instruktar ulu en }(\textbf{pri})\textbf{ ulo}. The verb \textbf{respondar}, to answer, is generally used intransitively, i.e., the object is rarely mentioned. When used transitively the verb has as object the words given as an answer. The person to whom, or the words to which, the answer is given can, therefore, never be direct object to the verb, but must be joined to it by a preposition. The construction is: \textbf{respondar ulo a persono}, \textbf{a remarko}, \textbf{a letro}, to answer a person something, to answer something to a remark, to a letter. Instead of the construction \textbf{respondar ad ulo} one may use \textbf{respondizar ulo}, to provide something with an answer. The second construction is more convenient in the passive voice: \textbf{letro respondizita}, a letter to which an answer has been given (III, 30: VI, 296) (51).

114. Reflexive verbs express an action which passes over to the subject of the verb as object. \textbf{Su} is used for the latter if it be a personal pronoun of the 3rd person singular or plural. \textbf{Me lezis me}, I have hurt myself. \textbf{Defendez vu, se vu povas}, defend yourself if you can. \textbf{Li tro laudas su}, they praise themselves too much.

115. Impersonal verbs have no subject. \textbf{Pluvas}, it rains; \textbf{tondras}, it thunders; \textbf{fulminas}, it lightens, there is lightening; \textbf{nivas}, it snows. In the compound tenses the participle is used in adjective form. \textbf{Esus pluvinta, se la vento ne esus veninta} (\textbf{pluvabus, se la vento ne venabus}), it would have rained if the wind had not come.

Some verbs are only apparently impersonal, the subject being an infinitive or the contents of a sentence, i.e., a fact. \textbf{Eventas ofte, ke on eroras}, it occurs often that one errs. \textbf{Oportas departar}, it is necessary to depart.

Note the expressions: \textbf{esas}, there is, there are;\textbf{ yen esas}, here is, here are. \textbf{Esas homi, qui opinionas}, there are people who are of the opinion.

116. Predicate consisting of two parts. The predicate sometimes consists of two parts, a verb and an attribute which may be either a substantive or an adjective. Such verbs are: \textbf{nomizar}, to name, to call; \textbf{igar}, \textbf{facar}, to make, to render; \textbf{trovar}, to find; \textbf{elektar}, to elect; \textbf{titolizar}, to give a title to; \textbf{judikar}, \textbf{opinionar}, \textbf{konsiderar}, to consider; \textbf{montrar su}, to show one's self; \textbf{sentar su}, to feel; \textbf{lasar}, to leave; etc.

\textbf{Lu nomizis lu mentiero}, he called him a liar; predicate: \textbf{nomizis mentiero}.

\textbf{Vu montris vu karitatoza}, you have shown yourself charitable; predicate: \textbf{montris karitatoza}.

\textbf{El su sentas febla}, she feels weak; predicate: \textbf{sentas febla}.

\textbf{La rejo imperis facar la soldao oficiro}, the king ordered to make the soldier an officer.

\textbf{El lasis me sola en mea mizero}, she left me alone in my misery.

With such verbs the attribute is frequently indicated by the particle \textbf{kom} (§ 109), and sometimes the accusative is necessary to avoid ambiguities (§ 85). \textbf{Lu elektis lu kom chefon di la ministraro}, he elected him chief of the cabinet. \textbf{Lu elektis lu kom chefo di la ministraro}, could also mean: he, as chief of the cabinet, elected him. Ambiguity can be avoided either by the use of the accusative, as just shown, or by a different order of the parts of the sentence as follows: \textbf{lu kom chefo di la ministraro elektis lu}. \textbf{Me trovis la libro amuzanta} or \textbf{la amuzanta libro}, I found the amusing book. \textbf{Me trovis la libro kom amuzanta}, I found the book amusing. Short adjectives as epithets are preferably put before the noun qualified, but since this position is not obligatory, they should be preceded by \textbf{kom} when they are at attributes. \textbf{Me judikas la bela domo} or \textbf{la domo bela}, I give a judgment about the beautiful house. \textbf{Me judikas la domo kom bela}, I consider the house beautiful.

Where an appropriate verb renders the meaning of a sentence unequivocal, the particle \textbf{kom} is not used before the attribute. \textbf{Ni nomizis nia unesma filiino Frieda e nia duesma Elsa}, we have called (named) our first daughter Frieda and our second one Elsa. \textbf{Ante forirar el vokis sua filiino Elsa e kisis el}, before going away she called (summoned) her daughter Elsa and kissed her.

\subsection*{TENSES AND MODES.}
117. The compound tenses (§ 37, b), especially the perfect, are rarely used. \textbf{Me skribis letro}, I have written (I wrote) a letter. When, however, the completion of the an action is to be emphasized, they are more expressive than the simple tenses. \textbf{Vu ne povas vidar li, nam li esas departinta de longe}, you cannot see them any more, for they have departed long ago. \textbf{La civitani ja abandonabis }(\textbf{esis abandoninta})\textbf{ la urbo, kande l'enimiki arivis}, the citizens had already left the city when the enemies arrived.

118. In a dependent clause the same tense and mode are to be used as would be if the clause were independent (Gramm. Compl., § 105). \textbf{Il dicis, ke il skribas}, he said that he had written (he said: I have written). \textbf{Me pensis, ke il esas hike}, I thought he was here; he would be here, \textbf{ke il esos hike}. \textbf{Me kredis, ke il venos}, I believed that he would come. \textbf{Ke kredis, ke il venus, se il ne impedesus}, I believed that he would come, if he were not prevented (here there is a distinct condition, not merely a future to be expressed). \textbf{Oportas, ke vu departez}, it is necessary that you depart (depart, imperative).

The above rule determines the use in dependent clauses of the conditional and of the imperative (or better expressed the optative). The former presupposes a condition implied or expressed, the latter indicates a command, intention or desire. There is no subjunctive mode proper.

119. An infinitive, just as a verbal noun, may be used after any preposition. \textbf{On manjas por vivar, on ne vivas por manjar}, we eat in order to live, we do not live in order to eat. \textbf{Sen dicar un vorto}, without saying a word; \textbf{ante departar}, before departing; \textbf{pos skribir}, after having written; \textbf{per lektar}, by reading; \textbf{pro mentiir}, for having told a lie; \textbf{en demonstrar}, in demonstrating.

The prepositions \textbf{ad}, \textbf{to}, and \textbf{di}, of, before an infinitive are, as a rule, not expressed; \textbf{la deziro dominacar}, the desire to rule; \textbf{la timo dronar}, the fear of drowning; \textbf{la inklineso mentiar}, the inclination to lie.

The infinitive may even be preceded by the definite article like a verbal noun. \textbf{La alkoholismo konsisatas ne en la ebriigar su, ma en la drinkar kustume alkoholo}, alcoholism consists not in making one's self drunk, but in drinking alcohol habitually (Gramm. Compl., § 81) (52).

120. A participle may replace a dependent clause when the latter has the same subject as the principal sentence. It is then used 1) in adverbial form, when the clause is adverbial or is intended as an adverb; 2) in adjective form, when the clause is attributive replacing an adjective or substantive. \textbf{Enirante la chambro lu salutis singlu aparte}, when he entered the room (entering the room) he saluted every one separately. \textbf{La enemiki omnalatere batite cesis kombatar}, since the enemies had been beaten on all sides they ceased to fight. \textbf{Batita} would mean who had been beaten. \textbf{La pueruli diligente lerninta esis rekompensata}, the boys who had learned diligently were rewarded. \textbf{Lerninte} would mean: because they had learned. \textbf{La chasisto observis volfo trakuranta la foresto}, the hunter observed a wolf running through the forest. If the wolf was running only \textbf{trakuranta} is permissible, for \textbf{trakurante} would refer to the hunter. If in such a sentence the participle refers to the subject, it is better to place it at the beginning of the sentence or before the verb and then the adjective or adverbial form may be used. \textbf{Trakuranta }(or \textbf{trakurante})\textbf{ la foresto la chasisto observis volfo}; \textbf{la chasisto trakuranta }(or \textbf{trakurante})\textbf{ la foresto observis volfo}. The rule is, therefore, that a participial adverb always refers to the subject of the sentence (V, 96; § 102, ipsa, ipse) (53).

There is also a so-called absolute participial construction. The participle has herein the adverbial form. It is employed when the dependent clause has another subject than the principal sentence (Gramm. Comp., § 106).

\textbf{La suno desparinte la kavaliero forkavalkis}, the sun having set, the knight rode away. This construction should not be employed too frequently, although it is useful and sometimes even necessary, as in mathematics and jurisprudence. \textbf{Donite un rekto ed un punto \ldots}, given one straight line and one point \ldots .

121. It is not permissible to use a preposition before a participle to replace a clause. If the relation of the dependent clause to the principal sentence is to be indicated more precisely, an infinitive with a suitable preposition is employed. \textbf{Pos finir sua aferi lu livis la urbo}, after having finished his business he left the city; pos finite would be incorrect. \textbf{Lu departis sen dicir adio ad ulu}, he departed without saying good-bye to anybody; \textbf{ne dicinte} could also be used, while \textbf{sen dicinte} would be incorrect. (3rd paragraph of note 53.)

122. The active participle with the substantive ending \textbf{-o} denotes the performer of an action, who is usually a person, rarely a thing, and indicates the tense of the action. \textbf{La vinkonto}, the future victor; la \textbf{konstruktino di la ponto}, the builder of the bridge. \textbf{La aboninto di la jurnalo}, the subscriber to the journal. \textbf{On admisis ka kom segunvola remplastanto di kad} (dec. 16, II, 579), \textbf{ka} has been admitted as optional substitute of \textbf{kad}. \textbf{Ico esas konsequanto di \ldots}, this is a consequence of \ldots (VI, 595). The participial substantive differs from the noun in \textbf{-isto} in that the latter denotes professional occupation, and from the noun in \textbf{-ero} in that tense is not indicated by this noun: \textbf{skribanto}, one who is just writing; \textbf{skribinto}, one who has written; \textbf{skribisto}, a writer by profession; \textbf{skribero}, writer; \textbf{la skribero di ca letro}, the writer of this letter.

The substantive of the passive participle also denotes usually a person: \textbf{la sendito}, the messenger; \textbf{la deputato}, the deputy; \textbf{la delegato}, \textbf{delegito}, the delegate; \textbf{la amorato}, the sweetheart.

Expressions like the following require the active participle instead of the English passive participle: \textbf{la mortinto}, the deceased; \textbf{la signatinto}, the undersigned (55).

123. Voice. Sometimes a reflexive verb is used where the sense is that of a passive. This occurs rarely in English, but frequently in French, and still more often in Italian and Spanish. In German also a reflexive verb is used in such instances, but more often a construction with the indefinite pronoun ``man.'' In the French language verbs used reflexively where the sense is passive are called false reflexive verbs (faux verbes réfléchis, Gramm. Compl., § 35). In English they are translated usually by the passive voice, sometimes by an active verb used intransitively, and in rare instances also in a reflexive manner. In Ido, however, the passive voice, preferably in the synthetic form, is to be employed for the French false reflexive verbs. \textbf{Nekorekta kustumo establisesis}, a wrong custom has established itself. \textbf{La karbono acendesis spontane}, the coal has ignited (itself) spontaneously. \textbf{To videsas omnadie}, this is seen every day (cela se voit tous les jours). \textbf{Tala paroli ne darfas uzesar}, such words must not be used. \textbf{Esas vivi qui lektesas quale un longa trauro}, there are lives that read like one long sorrow. \textbf{La krucho iras a l'aquo tam longe, til ke fine ol ruptesas}, the pitcher goes to the water so long until it finally breaks (tant va la cruche à l'eau qu'à la fin elle se brise).

124. Reciprocity. Verbs expressing reciprocity have as complement the pronouns \textbf{l'unu l'altru} (\textbf{unu altra}, \textbf{l'una l'altra} when a substantive can readily be supplied, VI, 343) when there are two subjects and \textbf{l'uni l'altri} (\textbf{uni altri}) when there are more than two subjects. \textbf{La fianciti amoras l'una l'altra }(\textbf{una altra}), the betrothed love one another. \textbf{La kombatinti vundis l'uni l'altri }(\textbf{uni altri}), the combatants wounded one another. When the reciprocal relation requires a preposition, the latter is placed between the two pronouns. \textbf{Li vovis fideleso l'unu a l'altru}, they solemnly promised to be faithful to one another. \textbf{Li ektiris la hari l'unu di l'altru}, they pulled each other's hair out. \textbf{La kaptiti imploris la tirani l'uni por l'altri}, the prisoners implored the tyrant one for the other.

Reciprocity is also expressed by a verb combined with the preposition inter. \textbf{Li interkonsentas}, they agree with each other; \textbf{li interparolas}, they talk with each other (54).

\subsection*{NUMERAL.}
\addcontentsline{toc}{subsection}{Numeral: Time of the day; term per cent; decimal fractions; date; age; etc.—§§ 125-132}
125. The numbers \textbf{miliard}, \textbf{milion}, \textbf{bilion}, etc., are treated like all other numbers, being connected with the objects numbered immediately and not by the preposition \textbf{de} like terms of quantity: \textbf{milion steli}, a million stars.

The numeral nouns proper, however, as duo, a pair; \textbf{dekeduo}, a dozen; \textbf{duadeko}, a score; \textbf{sisadeko}, three-score; \textbf{cento}, a hundred; \textbf{milo}, a thousand, are treated like other terms of quantity (§ 92, c): \textbf{duo de rejini}, a pair of queens; \textbf{duadeko de yari}, a score of years; \textbf{duadek yari}, twenty years.

126. Time of the day is expressed by means of the noun \textbf{kloko} as follows:

\textbf{quar kloki dek }(\textbf{minuti}), ten minutes past four; \\
\textbf{quar kloki e quarimo}, a quarter past four; \\
\textbf{quar kloki e duimo}, half past four; \\
\textbf{quar kloki e tri quarimi}, a quarter of five; \\
\textbf{quar kloki kinadek }(\textbf{minuti}), ten minutes of five; \\
\textbf{quar kloki kinadek e kin }(\textbf{minuti}), five minutes of five; \\
\textbf{esas un kloko}, it is one o'clock; \\
\textbf{esas du kloki}, it is two o'clock; \\
\textbf{ante tri kloki marine}, before three o'clock in the morning; \\
\textbf{pos ok kloki e duomo vespere}, after half past eight in the evening; \\
\textbf{qua tempo esas?} what time is it? (see end of note 56).

Subtraction should never be used in expressing the time of the day (56).

127. \textbf{Plus} and \textbf{minus} are prepositions used in mathematics and from them are derived the adverbs \textbf{pluse} (\textbf{plus}), \textbf{minuse} (\textbf{minus}) (see § 105).

The mathematical signs +, -, ×, ÷ or / are translated by \textbf{plus}, \textbf{minus}, \textbf{per}, and \textbf{sur}.

$6^4$ = six in the fourth power, \textbf{sis potenco quara};

$\sqrt[5]{1000}$ = fifth root of thousand, \textbf{radiko kina }(\textbf{kina radiko})\textbf{ de mil} (Gramm. Compl., § 98).

\small Remark. It would not be incorrect to designate the exponent of a power or of a root by an ordinal number instead of by a numeral adjective: \textbf{quaresma potenco de sis}; \textbf{kinesma radiko de mil}. \normalsize

128. Note the expressions: \textbf{omni tri}, \textbf{omni quar}, all three, all four; \textbf{omna duesma dio}, every second day, every other day (57); \textbf{la du}, both (adjectively); \textbf{due}, both (adverbially). The English term both may also be expressed in various other ways (L. Couturat, III, 559): \textbf{same }(\textbf{egale})\textbf{ en milito ed en paco}, both in war and in peace; \textbf{kune la mastri e la servisti}, both the masters and the servants. \textbf{Il havas du filiuli, qui omna }(\textbf{qui omni}, used in III, 559, is objectionable, see § 103)\textbf{ esas soldati}, he as two sons who are both soldiers.

129. The term per cent is to be expressed by \textbf{po cent}: \textbf{tri po cent}, three per cent; \textbf{dek po cent}, ten per cent (L. Couturat, V, 681). The substantive \textbf{procento} indicates a relation of two numbers. \textbf{Lu prestis a me kinacent dollar-i ye la procento di kin po cent}, he lent me five hundred dollars the percentage of five per cent. When the idea is one of extraction rather than of correspondence, as in the statistics, it is preferable to express the percentage as a fraction. In this army five per cent have been killed, ten per cent wounded, twenty per cent sick, \textbf{en ica armeo kin ek cent ocidesis, dek ek cent vundesis, duadek ek cent esis malada}, or better \textbf{kin centimi }(\textbf{kin sur cent})\textbf{ ocidesis, dek centimi }(\textbf{dek sur cent})\textbf{ vundesis, duadek centimi }(\textbf{duadek sur cent})\textbf{ esis malada} (see prepos. \textbf{ek}).

The sign \% for per cent and ‰ per mil should rather be avoided, it being more logical to use a fraction instead: 3\% = 3/100; 15‰ = 15/1000; \textbf{tri centimi}; \textbf{dek e kin milimi}.

130. Decimal fractions are separated from the integral numbers by a comma. They are enunciated by enumeration each digit in proper succession: 5,48709 = \textbf{kin, komo, quar, ok, sep, zero, non} (Prof.~Pfaundler, IV, 354). When there are no more than three decimals, they may be enunciated as a fraction: 2,4 = \textbf{du, komo, quar dekimi}; 0,68 = \textbf{zero, komo, sisadek e ok centimi}; 0,701 = \textbf{zero, komo, sepacent e un milimi} (V, 682; VII, 39, 193, dec. 1345) (58).

\small Remark. In a decimal number the name of the unit must not be placed between the integral number and the fraction; it is not permitted to write: \textbf{6 d. 50}, but only \textbf{6,50 d.} = \$6.50. One fourth of a dollar is \textbf{0,25 d.}; one fourth of a cent is \textbf{0,25 c.} The name of the unit always refers to the figure before the comma (VII, 163). \normalsize

131. In dates the succession is day, month, year. When numbers are used to designate them, they are separated by periods: \textbf{10. 2. 1913} (IV, 470; VI, 52, 418). An ordinal number is required when the year and its number are enunciated in a date. \textbf{Nia unesma filiino naskis ye la duadek-e-duesma }(\textbf{dio}) (\textbf{di, en})\textbf{ oktobro en }(\textbf{dum})\textbf{ la yaro mil e nonacent e sepesma, e nia duesma filiino ye la nonesma }(\textbf{dio}) (\textbf{di, en})\textbf{ julio en }(\textbf{dum})\textbf{ la yaro mil e nonacent e nonesma}, our first daughter was born on October 22nd in the year one thousand nine hundred and seven, and our second daughter on July 9th in the year one thousand nine hundred and nine.

132. Age is expressed by the verb \textbf{evar}. \textbf{Quante }(also \textbf{quanton})\textbf{ vu evas?} how old are you? (dec. 668, IV, 691). \textbf{Me evas quaradek e sep yari}, I am 47 years old. \textbf{Evoza}, old; \textbf{mezeva}, of middle age; \textbf{mezeva homino}, a woman of middle age. The Middle Ages is translated by \textbf{mezepoko} (59).

\subsection*{PREPOSITION.}
\addcontentsline{toc}{subsection}{Preposition: Definitions; derived adverbs as prepositions; adverbs and adjectives derived from prepositions—§§ 133-136}
133. Because of the great difficulties in using prepositions, inherent to all languages, an exact definition of every preposition is required. Only by keeping strictly to such a definition will the student be able to apply the proper preposition at the proper place.

\textbf{A} (\textbf{ad}), to, indicates a dative object: \textbf{Donez a me libro}, give me a book. \textbf{El sendis letro a lu}, she sent him a letter. \textbf{La urbestro furnisis manjajo e lojeyo a la neokupati}, the mayor furnished eating and lodging to the unemployed.

Direction: \textbf{Lu voyajas omnoyare a Paris}, he travels every year to Paris.

Thought of: \textbf{El pensas a sua filio}, she thinks of her child.

Sentiment toward: \textbf{Amo a la patri}, love for the parents (60).

\small Remark. Elision of the \textbf{d} of \textbf{ad} must not be employed in compositions: \textbf{adbase} (not \textbf{abase}), down with! (II, 165.) \normalsize

\textbf{Alonge}, along, at the side of: \textbf{alonge la rivero}, along the river.

\textbf{An}, at, on, indicates contiguity, juxtaposition (IV, 409). \textbf{On frapas an la pordo}, somebody knocks at the door; \textbf{klovo an la muro}, a nail on the wall. \textbf{La gasti sidas an la tablo}, the guests are sitting at the table. \textbf{New York jacas an Hudson}, New York lies on the Hudson. \textbf{La redingoto pendas an hoko an la muro}, the coat hangs on a nail on the wall.

\textbf{Apud}, at, near: \textbf{apud la domo esas puteo}, at the house is a well. \textbf{La furnelo esas apud la pordo}, the stove is at the door. \textbf{Apud la generalo stacis lua adjutanto}, near the general stood his adjutant.

\textbf{Ante}, ago, before (of time): \textbf{lu arivis ante me}, he arrived before me; \textbf{ante du kloki e quarimo}, before a quarter past two o'clock; \textbf{ante tri hori e duomo}, three and a half hours ago; \textbf{ante nelonge}, not long ago. The opposite of \textbf{ante} is \textbf{pos}, after.

\textbf{Avan}, in front of, before (of place); opposite: \textbf{dop}, behind, back of. \textbf{La akuzato sidas avan la jurinti}, the defendant sits before the jury. \textbf{Komence la princo kavalkadis dop sua kortani, plu tarde lu preterpasis li, e fine lu arivis avan li en la kastelo}, in the beginning the prince rode behind his courtisans, later he overtook them and finally he arrived in the castle ahead of them.

\textbf{Che}, at, in the house or home of, with to indicate customs:\textbf{ che la bakisto}, at the baker's. \textbf{Hiere me esis che mea onklulo}, yesterday I was at my uncle's. \textbf{Lu lojas che me}, he lives with me (in my house). \textbf{Che la Spartani esis kustumo ekpozar febla infanti}, it was customary with the Spartans to expose weakly infants. \textbf{Che la Usani prezidanto elektesas omni quaresma yaro}, in the United States a president is elected every four years.

\textbf{Cirkum}, around, about (of time, place, and quantity): \textbf{circum la urbo}, around the city; \textbf{cirkum tri kloki}, about three o'clock. \textbf{Lu restis che me cirkum tri hori}, he remained with me about three hours. \textbf{El spensis cirkum duadek dollar-i}, she spent about twelve dollars.

\textbf{Cis}, this side of (opposite: \textbf{trans}, on the other side of). \textbf{Cis la lago esas humila kabani, trans ol stacas superba kastelo}, this side of the lake are humble huts, on the other side of it stands a superb castle.

\textbf{Da}, by, through, marks the acting subject of the passive verb. \textbf{La patri esas amata da sua filii}, the parents are loved by their children. \textbf{Lu esis trapikata da kuglo}, he was pierced by a bullet. \textbf{La arboro esis faligata da hakilo}, the tree was felled by an axe (\textbf{hakilo} being considered, figuratively, as acting subject; see also \textbf{per}). \textbf{Da} indicates also the author: \textbf{poemo da Longfellow}, a poem by Longfellow; \textbf{dramati da Shakespeare}, dramas by Shakespeare.

\textbf{De}, of, from, marks the starting point of time and place, origin, derivation, descent from. \textbf{Lu venas de Paris}, he comes from Paris. \textbf{De tri dii}, since three days. \textbf{Me expektas vu de dek ed un kloki e duimo}, I am waiting for you since half past eleven. \textbf{De nun}, from now on; \textbf{de longe}, since a long time. If since, used of time, includes the duration of the whole time interval, \textbf{depos} (from \textbf{de} and \textbf{pos}) is preferable to \textbf{de} alone. \textbf{Depos la morto di sua spozulo el esis vidata nultempe o rare en loko di amuzo}, since the death of her husband she was never or seldom seen in a place of amusement. \textbf{De} denotes contents after terms of measure or quantity: \textbf{litro de vino}, a liter of wine; \textbf{trupo de volfi}, a pack of wolves; \textbf{metro de silka rubando}, a meter of silk ribbon; \textbf{glaso de aquo}, a glass of water (see § 92, c). Such terms include also adjectives indicating dimension, size contents, the actual measure for these values being connected with the preceding adjective by \textbf{de} (§ 95, 2).

There are special verbs, \textbf{evar} and \textbf{prezar}, to indicate age and weight, and these verbs do not require a prepositional complement: \textbf{infanto eventa sis yari or infanto sis-yara}, a child six years old or a child of six years; \textbf{lucio pezanta ok kilogrami} or \textbf{lucio ok-kilograma}, a pike weighing eight kilograms or a pike of eight kgr. It would be idiomatic to say: \textbf{infanto de sis yari}, for infanto is not a term of quantity nor is \textbf{yaro} a unit of measure for children. More admissible would be: \textbf{lucio de ok kilogrami}, and still more so \textbf{lucio de quar pedi}, a pike of four feet (four feet long), \textbf{pedo}, \textbf{kilogramo} being units of measure (V, 342) (61).

\textbf{Di}, of, marks the genitive, possession, belonging to: \textbf{la domo di mea patrulo}, the house of my father; \textbf{la filio di mea fratino}, my sister's child. \textbf{Di qua esas ca bastono? Di la gardenisto}, to whom does this stick belong? To the gardener.

A fine shade of meaning is obtainable by various use of the prepositions \textbf{da}, \textbf{de}, \textbf{di}: \textbf{la fotografuri da So.~X}, the photographs by Mr.~X (Mr.~X is the photographer); \textbf{la fotografuri de Sino.~Y}, the photographs of Madame~Y. (Madame~Y has been photographed); \textbf{la fotografuri di Sino.~Y}, the photographs of Madame~Y (Madame~Y is the possessor of the photographs). \textbf{La domo esis komprata da mea amiko de lua vicino}, the house has been bought by my friend from his neighbor. \textbf{La verki da Darwin di ca librovendisto esis komprata da me de lua agento}, the works of Darwin in possession of this book seller have been bought by me from his agent.

\textbf{Dop}, behind, back of, after (of place): \textbf{dop la domo}, behind the house. \textbf{La adjutanto kavalkadis dop la princo}, the adjutant was writing behind the prince.

\textbf{Dum}, during, for marks the interval of time an event has taken place in: \textbf{La mesajo arivis dum lua absenteso}, the message arrived during his absence. \textbf{Me esis okupata pri ca verbo dum tri monati}, I was busy with this work for three months.

\textbf{Ek}, from, out of, indicating motion. \textbf{Adportez stulo ek ta chambro}, bring a chair out of that room. \textbf{Lu prenis ca folio ek la tirkesto}, he took this leaf from the drawer. \textbf{Plura soldati iris ek la kampeyo}, several soldiers went outside the camp. \textbf{Ek} is not to be confounded with \textbf{de} where point of departure is to be indicated: \textbf{letro de Boston}, not \textbf{letro ek Boston}, a letter from Boston.

The preposition \textbf{ek} means also extracted or taken from a collection and is, therefore, used with numbers and after a superlative: \textbf{nonadek e un ek cent}, 91 of 100 (91\%); \textbf{kinadek e tri ek mil}, 53 of 1000 (53 mills); \textbf{fablo ek mil e un nokti}, a fable from thousand and one nights; \textbf{la maxim diligenta ek omna lerneri}, the most diligent of, among all pupils. The prepositions \textbf{de} and inter may be used in the same sense: \textbf{la maxim olda de }(\textbf{inter})\textbf{ la fratini}, the oldest of the sisters; \textbf{un de mea amiki}, one of my friends (cfr., \textbf{amiko di me}, a friend of mine, § 99).

By extension \textbf{ek} denotes the matter a thing consists of, is made of, extracted from, as it were. \textbf{La vazo esas ek vitro}, the vase is of glass; \textbf{kabano ek ligno}, a wooden hut; \textbf{botelo ek vitro}, a bottle of glass (consisting of); \textbf{botelo de lakto}, a bottle of milk (containing milk).

\textbf{En}, in, into, on, with or without motion and of time and place: \textbf{en la nokto}, in the night. \textbf{Li venis en la urbo}, they came into the city. \textbf{Lu sidas en la salono}, he is sitting in the parlor (60).

\textbf{Exter}, outside (without motion), except, besides. \textbf{Plura soldati stacis exter la kampeyo, la ceteri esis en la kampeyo}, several soldiers stood outside the camp, the others were in the camp. \textbf{Exter la dicita decidi, l'akademio ne facis altri}, besides the decisions named the academy has made no others.

\textbf{For}, away from, far from. \textbf{Dum plu kam dek yari li esis for sua patrio}, for more than ten years they were away from their fatherland.

\textbf{Inter}, between (of place and time), among: \textbf{inter Kristnasko e novyaro}, between Christmas and new year; \textbf{inter sep e ok kloki matine}, between seven and eight o'clock in the morning; \textbf{lakuno }(\textbf{desdensajo})\textbf{ inter la arbori}, a glade between the trees.

\textbf{Inter} denotes also reciprocity, exchange, division among. \textbf{La pueri ludas inter su}, the children play with each other. \textbf{Li kambiis sua redingoti inter su}, they exchanged their coats (between themselves). \textbf{La havajo di ca filantropo esis dividata inter la povri}, the fortune of this philanthropist was divided among the poor.

\textbf{Inter} may be used also after a superlative. \textbf{Sirius esas la maxim lumoza inter la fixa steli}, Sirius is the brightest among the fixed stars. \textbf{El esis la maxim bela e racionoza inter omna mea konocitini}, she was the prettiest and most sensible of all women I have known.

\textbf{Kontre}, against, contrary to, opposite: \textbf{kontre mea deziro}, against my wish. \textbf{La Greki kombatis non yari kontre la Troyani}, the Greeks fought nine years against the Trojans. \textbf{Kontra la kastelo}, opposite the castle.

\textbf{Koram}, in the presence of: \textbf{koram omna deputiti}, in the presence of all the deputies.

\textbf{Kun}, with, in connection with, in company of: \textbf{Il venus kun lua fratino}, he came with his sister. \textbf{Lu drinkas teo kun lakto}, he drinks tea with milk (45).

\textbf{Lor}, at the time of, at the same time as: \textbf{lor la milito di tria dek yari}, at the time of the war of thirty years (62).

\textbf{Minus}, minus, is used only in mathematical expressions: \textbf{triadek minus dek ed un esas (facas) dek e non}, thirty minus eleven are nineteen.

\textbf{Malgre}, in spite of, notwithstanding, contrary to the wish of: \textbf{malgre lua pregi}, in spite of his entreaties. \textbf{El mariajis su kun il malgre sua matro}, she married him contrary to her mother's wish.

\textbf{Per}, by, by means of, through, with, marks instrument, means: \textbf{trapikar per poniardo}, to pierce with a dagger. \textbf{Ni povas nun sendar paketi mezgrada per la posto}, we can send now medium-sized packages with the post. \textbf{Per pekunio on povas komprar mem honori}, with money one can buy even honors. \textbf{Malada per}, sick with (see \textbf{pro}).

\textbf{Plus}, plus, is used in mathematical expressions: \textbf{dek e du plus triadek e kin esas }(\textbf{facas})\textbf{ quaradek e sep}, twelve plus thirty-five are forty-seven.

\textbf{Po}, for, at the price of, in exchange for, indicates value received or given. \textbf{Me kompris ca vesto po triadek dollar-i. Vu pagis tro multa pekunio po ol}, I bought this suit for thirty dollars. You have paid too much money for it. \textbf{Tri po quarimo}, three for a quarter. \textbf{Lu kompris du kamizi po tri dollar-i}, he bought two shirts for three dollars. \textbf{Du kamizi po tri dollar-i single}, two shirts three dollars apiece. \textbf{Rubando po kinadek cent-i singla ulno}, ribbon fifty cents a yard. \textbf{Po quante vu vendas ol?} for how much do you sell it? \textbf{Quante vu vendas ol}, would mean: how much of it do you sell?

\textbf{Me ne bezonas kultelo, voluntez kamiar ol po cizo}, I do not need a knife, please exchange it for a pair of scissors.

The unit of time relating to price is indicated with \textbf{por}: \textbf{luar chambro po tri dollar-i por un semano} (also \textbf{semane}), to rent a room for three dollars a week (weekly).

\textbf{Po} is not to be used when the price is the direct object of a verb. \textbf{Quante kustas ca chapelo? Ol kustas du dollar-i kinadek}. How much does this hat cost? It costs two dollars fifty (not \textbf{po quante}, nor \textbf{po du dollar-i kinadek}).

\textbf{Por}, for, in favor of, in the interest of: \textbf{kombatar por la patrio}, to fight for the fatherland; \textbf{sukrajo por la infanti}, candy for the children. \textbf{Por qua vu adportas ca glaso?} For whom do you bring this glass? \textbf{Ol esas por vu}, it is for you. \textbf{Ca skolano esas inteligenta, omna tasko esas facila por lu}, this pupil is intelligent, every task is easy for him. \textbf{Sisadek pollar-i por un monato} (\textbf{monate}), sixty dollars a month (monthly). \textbf{Por quo vu bezonas ol?} What do you need it for?

\textbf{Pos}, after (of time): \textbf{pos quar kloki}, after four o'clock; \textbf{pos quar hori}, after four hours.

\textbf{Preter}, passing, past by, indicates motion in front of something from one side to the other. \textbf{La kavaliero ne haltis apud la kirko, ma kavalkadis preter ol}, the knight did not stop at the church, but rode past it.

\textbf{Pri}, about, on concerning: \textbf{diskurso pri ekonomio}, lecture on economy. \textbf{Lu paroladis pri la neceseso di linguo internaciona}, he spoke about the necessity of an international language.

\textbf{Pro}, on account of, because of, in consequence of, out of: \textbf{pro lua insolenteso}, on account of his arrogance: \textbf{pro to}, on this account, therefore; \textbf{pro jaluzeso}, out of jealousy; \textbf{malada pro diabeto}, sick with diabetes. \textbf{Lu perisis pro ca maladeso}, he perished of this disease. \textbf{La migreri tra la dezerto tre sufradis pro dursto}, the wanderers through the desert suffered greatly from thirst. \textbf{Pro quo}, what for, why? \textbf{De} has sometimes a similar meaning: \textbf{malada de tuberkloso}, sick with tuberculosis (V, 1).

\textbf{Proxim}, near (opposite: for): \textbf{Dum la nokto la enemiki ne foriris a sua kampeyo, ma restis proxim la siejita urbo}, during the night the enemies did not retire to their camp, but remained near the beleaguered city.

\textbf{Segun}, according to, in conformity with, after: \textbf{segun mea judiko}, according to my judgement; \textbf{segun ca modelo}, after this model; \textbf{segun vua deskripto}, in conformity with your description; \textbf{formulo segun Lagrange}, formula after Lagrange.

\textbf{Sen}, without, deprived of: \textbf{la kavaliero sen timo e reprocho}, the knight without fear and reproach. \textbf{El ne volis vivar sen sua amorato}, she did not want to live without her sweetheart.

\textbf{Sub}, under, below, beneath: \textbf{sub la tablo}, under the table.

\textbf{Super}, over, above (but not in contact with). \textbf{La muevo flugas super la maro}, the sea gull flies over the sea. \textbf{La glavo di Damokles pendas super lu}, the sword of Damocles hangs over him.

\textbf{Sur}, on, upon, over (in contact with): \textbf{kamentubo sur la tekto}, a chimney on the roof. \textbf{On vidas fore navo sur la maro ed aernavo super la maro}, one sees in the distance a ship on the sea and an airship above the sea.

\textbf{Til}, till, until, from \ldots to: \textbf{til la rivido}, au revoir, until we see each other again; \textbf{tri til quar hori}, from three to four hours; \textbf{de Boston til New York}, from Boston to New York; \textbf{de january til julio}, from January till July. \textbf{Til} marks also a maximum. \textbf{Lu demandis po sua domo de okamil til dekamil dollar-i}, he asked for his house from eight to ten thousand dollars; or simply \textbf{til dekamil dollar-i}.

\textbf{Tra}, through, through the middle of: \textbf{tra la muro}, through the wall; \textbf{tra la foresto}, through the forest.

\textbf{Trans}, on the other side of, over. \textbf{Ni irez trans la rivero}, let us go to the other side of the river.

\textbf{Ultre}, beside, in addition to. \textbf{Ultre mea patrala linguo me savsa plura altri}, besides my mother tongue I know several others.

\textbf{Vers}, towards, -ward, in direction of: \textbf{vers sudo}, southward. \textbf{Perdinte nia exakta voyo en la obskureso ni navigez nun sempre vers westo spektante ofte la stelo polala ye nia dextra latero}, having lost our exact way in the darkness we will now sail always towards west frequently looking at the polar star to our right side.

\textbf{Vers} refers only to direction in space and is not to be used in the sense of approximately (see \textbf{cirkum}; VI, 144).

\textbf{Vice}, instead of: \textbf{Vice Iphigenia cervino sakrifikesis ad Artemis}, instead of Iphigenia a hind was sacrificed to Artemis.

\textbf{Ye} is a preposition of indefinite meaning. It is used whenever no other preposition meets the requirement. It denotes especially time or place of an event: \textbf{ye la noktomezo}, at midnight; \textbf{ye la fino di l'koridoro}, at the end of the corridor. \textbf{Ye} marks the part of the body touched or grasped: \textbf{prenar ulu ye la manuo}, to take one by the hand. \textbf{Lu kaptis la kolombo ye la pedo}, he caught the dove by the foot.

134. The verb \textbf{konfundar}, to confound, is connected with the prepositions \textbf{kun} and \textbf{ad}. The former is used when the sense is to make a muddle of several things in not distinguishing them properly. \textbf{La vinkeri penetris la urbo e masakris omni konfundante pacema civitani kun kombatinti}, the victors penetrated into the city and massacred all, confounding peaceful citizens with combatants. When the sense is to confuse the identity of something with something else the preposition \textbf{ad} is used. \textbf{La basoto sendesis por querar la kamelo, e renkontrante kato facanta grandega gibo, lu konfundis lu a kamelo}, the dachshund was sent to fetch the camel, and meeting a cat that put up her back to an immense hump, he confounded her with the camel (took her for the camel) (dec. 1205, 1206, VI, 513).

135. No preposition is added to a derived adverb used prepositionally (§ 111) when the adverb is derived from a transitive verb: \textbf{danke vua helpo}, thanks to your help; \textbf{ecepte to}, except this; \textbf{kompare la exemplo}, in comparison with the example; \textbf{concerne la kontrato}, concerning the contract; \textbf{relate la kondiciono}, relating to the condition; \textbf{supoze lua vinko}, in case he is victorious; etc.

When however, the adverb is derived from an intransitive verb, from a substantive, an adjective, or from a somewhat longer expression, the preposition required by the word or the expression is to be added: \textbf{konkorde kun}, in accordance with; \textbf{kontraste a} (\textbf{ye}), in contrast to; \textbf{\ldots funde di la maro}, at the bottom of the sea (\textbf{la fundo di la maro}); \textbf{latere di la riveo}, at the side of the river (\textbf{la latero di la rivero}); \textbf{meze di la chambro}, in the middle of the room; \textbf{okazione di lua dio naskala}, at the occasion of his birth day; \textbf{\ldots diverse de}, differently from (\textbf{diversa de}, different from); \textbf{konforme a}, in comformity with (\textbf{konforma a}); \textbf{simile a}, similarly to (\textbf{simila a}, similar to); \textbf{\ldots dextre di la statuo}, at the right side of the statue (\textbf{la dextra latero di la statuo}); \textbf{sinistre di la prezidanto}, to the left of the president (\textbf{la sinestra di la prezidanto}); \textbf{an Hudson admonte de Newburgh}, on the Hudson above Newburgh (\textbf{de N. a la monto}, from N. to, towards, the mountain, source of the river); \textbf{advale de Albany}, below Albany (\textbf{de A. a la valo}, from A. to, towards, the valley, downstream). \textbf{Aernavo flugas trans la rivero admonte de la katarakto}, the airship flies over (crosses) the river above (upstream from) the waterfall (between the waterfall and the source); \textbf{super la katarakto} would mean that the airship was over the waterfall during the crossing of the river (dec. 1094, VI, 212; dec. 1137, VI, 273; dec. 1183, VI, 417; IV, 353).

136. From many prepositions adjectives and adverbs are derived immediately (§ 57). In this way the adverbs and adjectives \textbf{sube}, below; \textbf{suba}, lower; \textbf{supere}, above; \textbf{supera}, upper; \textbf{sure}, above; \textbf{sura}, upper are obtained from the prepositions \textbf{sub}, \textbf{super}, \textbf{sur} (63).

\subsection*{CONJUNCTION.}
\addcontentsline{toc}{subsection}{Conjunction: Coordinate; subordinates; mode governed by; transformation of subordinate clause into infinitive phrase—§§ 137-140}
The conjunctions are divided into two classes: (1) the coordinating conjunctions which connect two words, sentences, or clauses of equal rank, and (2) the subordinating conjunctions which unite a clause with the principal sentence.

137. The coordinating conjunctions are: \\
\begin{tabular}{l l}
\textbf{anke}, also, too; & \textbf{o \ldots o}, either \ldots or; \\
\textbf{ankore}, yet, still; & \textbf{o} (\textbf{od}), or (inclusive and exclusive); \\
\textbf{do}, therefore, then, consequently; & \textbf{or}, now (63); \\
\textbf{e} (\textbf{ed}), and; & \textbf{pro to}, therefore; \\
\textbf{e \ldots e}, as well as; & \textbf{seque}, then; \\
\textbf{kad} (\textbf{ka}), introduces a question & \textbf{sive \ldots sive}, either \ldots or, whether \ldots or whether, be it that \ldots be it that; \\
\textbf{konseque}, consequently; & \textbf{tamen}, however, yet, still; \\
\textbf{lore \ldots lore}, now \ldots now; & \textbf{tante min}, so much less; \\
\textbf{ma}, but; & \textbf{tante plu}, so much more; \\
\textbf{nam}, for; & \textbf{vel \ldots vel}, either \ldots or; \\
\textbf{ne nur }(\textbf{sole})\textbf{ \ldots ma }(\textbf{anke}) , not only \ldots but also; & \textbf{ya}, indeed, certainly; \\
\textbf{nek \ldots nek}, neither \ldots nor; & \textbf{yen}, here is (65);
\end{tabular}

\textbf{Kad lu venos}, will he come? \textbf{Kad vu do ne kontestas co}, then you do not deny this? \textbf{Se lu entraprezos to malgre omna danjeri, tante plu lu meritos laudo}, if he will undertake this in spite of all danger, so much more praise he will deserve. \textbf{Tante plu lu ridis, quante lu vidis, ke lua opozanto iracis}, he laughed the more as he saw that his opponent was fretting. \textbf{Vu ya mentiis}, you lied indeed. \textbf{Lu ya esas mea amiko}, why, he is my friend (§ 85) (66).

\small Remark. \textbf{Vel} as inclusive alternative is useful in some cases which require technical precision (VI, 539, 540; VII, 158, 159). Its use may offer some difficulty. Those to whom it may appear too subtle will do better to employ always the conjunction \textbf{od}, they will then never make mistakes (VII, 399, 400). \normalsize

138. Subordinating conjunctions are: \\
\begin{tabular}{l l}
\textbf{kad} (\textbf{ka}), whether, if; & \textbf{se}, if; \\
\textbf{kande}, when, at the time when; & \textbf{ecepte se}, except if; \\
\textbf{ke}, that; & \textbf{se ne}, if not; \\
\textbf{quale}, as, how; & \textbf{se nur}, \textbf{nur se}, if only; \\
\textbf{quankam}, although; & \textbf{quale se}, as though, as if; \\
\textbf{quante}, as much as; & \textbf{ube}, where.
\end{tabular}

Most of the prepositions added to the conjunction \textbf{ke} furnish subordinating conjunctions: \\
\begin{tabular}{l l}
\textbf{de ke}, because, from (with participle); & \textbf{pos ke}, after; \\
\textbf{depos ke}, \textbf{de kande ke}, since; & \textbf{pro ke}, because; \\
\textbf{dum ke}, while, during the time when; & \textbf{segun ke}, according to whether; \\
\textbf{kontre ke}, while (contrast); & \textbf{sen ke}, without (with participle); \\
\textbf{malgre ke}, notwithstanding that; & \textbf{til ke}, until; \\
\textbf{per ke}, through the fact that; & \textbf{ultre ke}, besides that; \\
\textbf{por ke}, in order that; & \textbf{vice ke}, instead that.
\end{tabular}

\textbf{Lua eruditeso venas, de ke lu sempre lernis diligente}, his extensive knowledge comes from his always having studied diligently. \textbf{Dum ke me foresis, grava mesajo arivis por me}, while I was away, a grave message arrived for me. \textbf{Lu restis tranquila, kontre ke me emoceskis}, he remained quiet, while I got excited. \textbf{Lu sucesis vinkar, per ke lu uzis ruzo}, he succeeded to carry the victory through employing a ruse.

Likewise adverbs form subordinate conjunctions through addition of \textbf{ke}: \\
\begin{tabular}{l l}
\textbf{kaze ke}, in case that; & \textbf{por ke}, in order that; \\
\textbf{kondicione ke}, under the condition that; & \textbf{tante ke}, so much that; \\
\textbf{omnafoye ke}, every time that; & \textbf{tante longe ke}, so long that; \\
\textbf{supoze ke}, supposing that; & \textbf{tante ofte ke}, so often that; \\
\textbf{tale ke}, notwithstanding that; & \textbf{tante plu \ldots ke}, the more so as; \\
\textbf{per ke}, through the fact that; & \textbf{time ke}, for fear that.;
\end{tabular}

Again other conjunctions are formed by the combination of two adverbs or of a preposition and an adverb: \\
\textbf{quante plu \ldots tante plu}, the more \ldots the more, the more \ldots so much more. \\
\textbf{quante min \ldots tante min}, the less \ldots the less, the less \ldots so much less. \\
\textbf{quik kande}, as soon as. \\
\textbf{same kam}, in the same manner as. \\
\textbf{segun quante}, in the measure as, as far as, in so far as (III, 41). \\
\textbf{tam \ldots kam}, as \ldots as; \textbf{tam longe kam}, as (so) long as; \textbf{tam ofte kam}, as (so) often as. \\
\textbf{tante plu \ldots quante plu}, so much more as. \\
\textbf{tante min \ldots quante min}, so much less as.

\textbf{Quante plu la homi posedas, tante plu li volas posedar}, the more man possesses, the more he wants to possess.

The correlative of \textbf{tam} is always \textbf{kam}, that of \textbf{tante} always \textbf{ke}. An adverb of time has as correlative \textbf{kande}, and an adverb of place \textbf{ube}: \textbf{nun kande} (not \textbf{nun ke}), now that; \textbf{hike}, \textbf{ube}, here where (VI, 141).

139. The conjunctions do not govern a fixed mode. The mode is determined by the sense of the sentence, a condition being indicated by the conditional, a wish by the optative (imperative, subjunctive), etc. The conjunction \textbf{por ke}, expressing always a desire, is, as a rule, followed by the subjunctive (imperative). \textbf{Se} is frequently followed by the conditional.

140. A subordinate clause may frequently be transformed into an infinitive phrase. This construction is especially convenient whenever the clause has the same subject as the principal sentence. With the conjunctions composed of a preposition and \textbf{ke} the construction is accomplished by omitting \textbf{ke} and changing the finite mode into the infinitive. No infinitive construction is applicable when the dependent clause and the principal sentence have different subjects. \textbf{Lu sucesis en sua granda entraprezo per deceptir omni}, he succeeded in his great undertaking by deceiving everybody (\textbf{per ke lu deceptis omni}). \textbf{Ante forirar }(\textbf{ante ke lu forisis}), \textbf{lu presis la manuo di singlu}, before leaving he shook hands with everybody, but only: \textbf{ante ke lu foriris, singlu presis lua manuo}, before he left, everybody shook hands with him.

\RaggedLeft
\textbf{Jovdio, la dek e okesma february 1914.} \\
Thursday, February 28th, 1914.

\textbf{Merkurdio, la dek e unesma mayo 1919.} \\
Wednesday, May 21st, 1919. \par \justifying
