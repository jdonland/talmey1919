\subsection*{Alphabet and Pronunciation.}
\addcontentsline{toc}{subsection}{Alphabet and Pronunciation—§§ 1–7}
1. The alphabet of Ido contains 26 letters: \textbf{a}, \textbf{b}, \textbf{c}, \textbf{d}, \textbf{e}, \textbf{f}, \textbf{g}, \textbf{h}, \textbf{i}, \textbf{j}, \textbf{k}, \textbf{1}, \textbf{m}, \textbf{n}, \textbf{o}, \textbf{p}, \textbf{q}, \textbf{r}, \textbf{s}, \textbf{t}, \textbf{u}, \textbf{v}, \textbf{w}, \textbf{x}, \textbf{y}, \textbf{z}. The five vowels \textbf{a}, \textbf{e}, \textbf{i}, \textbf{o}, \textbf{u} are named and pronounced according to the continental pronunciation (see `continental' in Standard Dictionary). They are neither too short nor too long and the \textbf{e} and \textbf{o} may be pronounced either closed or open (1).\footnotemark[1] The names of the consonants are as follows: \textbf{be}, \textbf{ce}, \textbf{de}, \textbf{fe}, \textbf{ge}, \textbf{he}, \textbf{je}, \textbf{ke}, \textbf{le}, \textbf{me}, \textbf{ne}, \textbf{pe}, \textbf{que}, \textbf{re}, \textbf{se}, \textbf{te}, \textbf{ve}, \textbf{we}, \textbf{xe}, \textbf{ye}, \textbf{ze}.
\footnotetext[1]{Indices like this (1) refer to the notes in part IV.}

2. Two apposed vowels are pronounced separately, each according to its own pronunciation, as in the combinations \textbf{au}, \textbf{eu}, \textbf{ia}, \textbf{ie}, \textbf{io}, \textbf{ua}, \textbf{ue}, \textbf{uo}. The first two are considered as consisting of one syllable or as diphthongs, the \textbf{u} becoming a consonant (Grammaire Complète, p. 8, 2): \textbf{Augusto}, \textbf{Europa} (three syllables); \textbf{kauzo}, \textbf{kaulo} (two syllables). When, however, \textbf{a} and \textbf{e} are joined to \textbf{u} through the composition of two words, each vowel forms a syllable by itself. Thus \textbf{neutila} (composed of \textbf{ne} and \textbf{utila}) consists of four syllables while \textbf{neutra} contains only two (2).

3. \textbf{U} becomes a (sort of) consonant, with the pronunciation of the English w, after q, as in \textbf{qua}, \textbf{quik}, and also after some other consonants when a vowel follows, as in \textbf{guidar}, \textbf{linguo}, \textbf{Suiso} (2). 
    
4. \textbf{C} is pronounced always like ts in wits, tsar, \textbf{g} always like g in go, and \textbf{s} always like the s in son. \textbf{J} is pronounced like the French j. \textbf{X} may be pronounced either like ks, as in excuse, or like gz, as in example. \textbf{Y} is pronounced always like the y in young or like the German j. All other consonants are pronounced like the same English consonants.
    
\textbf{Q} occurs only in the combination \textbf{qu} which is pronounced like the same English combination in queer. The \textbf{u} in this combination is a quasi-consonant and can therefore never receive the accent: \textbf{quâ}, \textbf{quâla}, \textbf{quânkam} (2). 
    
\textbf{W} sounds like the English \textbf{w}. It occurs only in a few words taken from the English: \textbf{wato}, \textbf{westo}, \textbf{wisto}.
    
There are two digraphs, \textbf{ch} and \textbf{sh}, which are pronounced like the same English digraphs in chin and shine.
    
When \textbf{c} and \textbf{s} are joined to \textbf{h} through the composition of two words, each consonant must be pronounced for itself. A hyphen is used between the two consonants to show at a glance that they do not form a digraph: \textbf{chas-hundo}. The names of these two digraphs are \textbf{che}, \textbf{she}.
    
5. Syllable. A word consists, or ought to be considered as consisting, of so many syllables as it contains vowels: \textbf{boao}, \textbf{muzeo}, \textbf{amanta} have three syllables. \textbf{Opiniono} should be considered as a word of five syllables. The combinations \textbf{au}, \textbf{eu}, however, are regarded as monosyllabic except in composed words: \textbf{neutra}, accordingly, has two syllables, \textbf{Augusto}, \textbf{Europa} have three syllables, but \textbf{neutila} has four syllables (3). 
    
6. Accent. The accent rests on the last but one syllable in a complete word (tonic accent): \textbf{âmo}, \textbf{amânta}, \textbf{nîa}, \textbf{pîe}, \textbf{dîo}, \textbf{glûo}, \textbf{kavâlo}, \textbf{muzêo}, \textbf{herôo}, \textbf{boâo}, \textbf{opiniôno}. When \textbf{au} and \textbf{eu} forming one syllable occur in the penultimate, the accent rests on the \textbf{a} or \textbf{e}: \textbf{kâulo}, \textbf{kâuzo}, \textbf{nêutra}. The infinitives, however, have the accent on the last syllable: \textbf{amâr}, \textbf{amîr}, \textbf{amôr}. 

\small Remark. Other words ending in a consonant, as some adverbs and prepositions, have the accent on the penultimate, in conformity with the general rule, for instance: \textbf{mâxim}, \textbf{mînim}, \textbf{âvan}, \textbf{cîrkum}, \textbf{prôxim}, etc. In their derivations the accent moves one syllable towards the end, also in conformity with the general rule: \textbf{maxîme}, \textbf{avâne}, \textbf{proxîma}. \normalsize
    
In words of a polysyllabic root i and u before a vowel cannot receive the tonic accent so that the latter has to be put on the third vowel from the end: \textbf{acadêmio}, \textbf{enêrgio}, \textbf{fîlio}, \textbf{pricîpua}, \textbf{vâkua}, \textbf{perpêtue}, \textbf{tênue}, \textbf{lînguo}, \textbf{mânuo}, \textbf{sêxuo}. According to this rule in a word like \textbf{Austria} even the fourth vowel from the end has the tonic accent. 

In composed words, however, i and u before the end vowel must have the tonic accent if the second component be a word of a monosyllabic root: \textbf{cadîe}, to-day; \textbf{omnadîe}, daily; \textbf{despîa}, sinful; \textbf{fishglûo}, fish-glue. The days of the week, however, do not have the accent on the i: \textbf{sûndio}, \textbf{lûndio}, \textbf{mârdio}, \textbf{satûrdio}, etc. These words are not to be considered as composed words, and some of them could not even be considered as composed words, for there are no independent roots mar, merkur, etc. (see VI, 136). 
    
7. Capital letters are used for the first letter of proper names and of all their derivations. Geographical names and names of nations are considered as proper names: \textbf{Azia}, \textbf{Aziala}, \textbf{Aziano}, \textbf{Anglo}, \textbf{Idala}, \textbf{Idisto}. The names of the days of the week and the names of the months have small initials: \textbf{jovdio}, Thursday; \textbf{januaro}, January. Likewise \textbf{sioro}, \textbf{siorulo}, \textbf{siorino}, Mr., Mrs., have small initials. Their abbreviations, however, should be written with capital letters to render them more conspicuous (V, 745): \textbf{So.}, \textbf{Sulo.}, \textbf{Sino.} (This is not official.) 

\textbf{Kristano}, Christian, is written with a capital letter, so is \textbf{Mohamedisto} (\textbf{Kalvinisto}, \textbf{Budhisto}), Mohamedan (VII, 283) (4).

A capital letter is used for the first letter of the word beginning a complete sentence and therefore after every period.

\addtocontents{toc}{\centering \textbf{Parts of Speech}}
\subsection*{I. ARTICLE.}
\addcontentsline{toc}{subsection}{Article—§§ 8–9}
8. The definite article is \textbf{la} for singular and plural: \textbf{la domo}, the house; la domi, the houses. In plural the article \textbf{le} is used when no other word indicates the plural either through form (plural ending \textbf{-i}) or through meaning (numeral or pronoun), as with proper names, names of letters, and adjectives used substantively: \textbf{le Cato}, the Catos; \textbf{le ze}, the zeds; \textbf{le mea}, mine (plural); \textbf{le bona}, the good ones.

Before an adjective expressing in the most general sense anything whatsoever that carries the quality indicated by the adjective \textbf{lo} is used as a sort of article (§ 16): \textbf{lo bona}, the good; \textbf{lo mala}, the bad; \textbf{lo bela}, the beautiful. 

There is no indefinite article; \textbf{homo}, a human being; \textbf{animali}, animals. 

9. The \textbf{a} of the article may be elided before a vowel. An apostrophe is to be used instead of the \textbf{a} so that no isolated \textbf{1} remains: \textbf{la una—la altru} or \textbf{l'unu—l'altru}, one another; \textbf{la akuzativo} or \textbf{l'akuzativo}, the accusative; \textbf{la imprimo} or \textbf{l'imprimo}, the printing (not 1 akuzativo, 1 imprimo; II, 408, 477). 

Before a consonant, however, the \textbf{a} of the article may be elided only after the prepositions \textbf{a}, \textbf{da}, \textbf{de}, \textbf{di}. This elision is indicated either by an apostrophe or by combining the \textbf{1} of the article with the preceding preposition into one word and omitting the apostrophe: \textbf{a l'regulo} or \textbf{al regulo}, to the rule; \textbf{di l'fero} or \textbf{dil fero}, of the iron (decisions 587, 588, 713, 949; IV, 561; V, 65; VI, 162). 

Elision is not permissible where it may cause ambiguity. Thus in \textbf{la afero}, the affair, the \textbf{a} must not be elided, for \textbf{l'afero}, the affair, may be confounded with \textbf{la fero}, the iron (Gramm. Compl., § 14).

\subsection*{II. SUBSTANTIVE.}
\addcontentsline{toc}{subsection}{Substantive—§§ 10–11}
10. Every substantive ends in \textbf{-o} in singular and in \textbf{-i} in plural, and inversely every polysyllabic word ending in \textbf{-o} or in \textbf{-i} is a noun in singular or plural: \textbf{la libro}, the book; \textbf{la infanti}, the children; \textbf{kato}, a cat; \textbf{hundi}, dogs.\footnotemark[1]
\footnotetext[1]{The pronouns \textbf{ulo}, something; \textbf{ico}, this; \textbf{ito}, that; etc., being used only substantively do not form an exception to this rule. The same holds good with proper names, for they are unchangeable and must keep their original forms. The names of countries ending in -a or otherwise are regarded as proper names.

About the plural of foreign words see § 86, p. 61.}

11. Declension. There is no declension. Nominative and accusative are alike and the other cases are formed by means of prepositions: \textbf{viro}, a man; \textbf{di viro}, of a man, a man's; \textbf{a viro}, to a man; \textbf{la viri}, the men; \textbf{di la viri}, of the men, the men's; \textbf{a la viri}, to the men. 

There is an accusative form, obtained by adding \textbf{-n} to the forms in \textbf{-o} or \textbf{-i}. It is used to prevent ambiguities and therefore always when the noun as direct object precedes the predicate and is not preceded by the subject: \textbf{la soldatin la rejo laudis, la generalon il reprochis}, it is the soldiers that the king praised, the general he reproached (see §§ 75, 85).

\subsection*{III. ADJECTIVE.}
\addcontentsline{toc}{subsection}{Adjective—§§ 12–15}
12. Every adjective ends in \textbf{-a}, and inversely every polysyllabic word ending in \textbf{-a} is an adjective. 
Remark. Names of countries ending in \textbf{-a}, as \textbf{Anglia}, England; \textbf{Francia}, France; \textbf{Germania}, Germany, do not furnish an exception to this rule. For they are to be regarded as proper names, and proper names may have any ending whatsoever.

The adjective has the same form in singular and plural: \textbf{bela floro}, a beautiful flower; \textbf{verda arbori}, green trees. There is no plural form of the adjective (VII, 68, dec. 1218). When, therefore, a noun in plural is omitted because of being easily understood, while an adjective referring to it is expressed, the plural is to be indicated in some other manner, as by the plural article \textbf{le} or by the plural of some suitable indefinite pronoun: \textbf{tu makulizis tua kayero, pue reto; ektirez la despura folieti e retenez nur le pura en la kayero}, you have soiled your writing book, my child; tear out the soiled leaves and retain only the clean ones in the book. Also \textbf{kelki pura}, a few clean ones; or \textbf{uli pura}, some clean ones will sometimes suffice. 

13. The adjective remains invariable also in regard to case, but in some instances, which occur rarely, it assumes the accusative ending \textbf{-n}, namely when as direct object it precedes the predicate and the noun it qualifies is not expressed: \textbf{la bona lernantin la instruktisto laudis, le malan lu punisis}, the good pupils the teacher praised, the bad ones he punished. When however the adjective is accompanied by its substantive, usually the latter alone receives the accusative ending if necessary, but it is permissible to give the accusative ending also to the former: \textbf{panon nia omnadia donez ad ni cadie}, our daily bread may you give us to-day; \textbf{panon nian omnadian} would also be admissible (Gramm. Compl., § 102).

14. The \textbf{a} of the adjective may be elided before a consonant as well as before a vowel when no accumulation of consonants would be produced through the elision. It is not necessary, but permissible, to indicate the elision by an apostrophe (dec. 583, Progr., IV, p. 562). This elision is recommendable with derived adjectives, especially those derived by the suffix \textbf{-al}: \textbf{maral aquo}, sea water; \textbf{domal spensi}, house expenses. \textbf{Domala laboro}, domestic labor, is preferable to \textbf{domal laboro}. Elision of the \textbf{a} of the adjective is not to be used too frequently. 

The elision of the \textbf{a} of the adjective leaves the place of the tonic accent unchanged: \textbf{domâla}, \textbf{domâl}. 

15. Comparison. Comparison of equality is expressed by \textbf{tam \ldots kam}, as \ldots as, so \ldots as; comparison of inequality of superiority by \textbf{plu}, more; of inequality of inferiority by \textbf{min}, less; the superlative by \textbf{maxim}, most; \textbf{minim}, least. Equality: \textbf{tam richa kam}, as rich as; \textbf{tam saja kam}, as wise as.
Comparative: \textbf{plu} (\textbf{min}) \textbf{bela}, more (less) beautiful; \textbf{plu bona}, better; \textbf{min bona}, less good. 
Superlative: \textbf{maxim} (\textbf{minim}) \textbf{kurajoza}, most (least) courageous; \textbf{maxim bona}, best; \textbf{minim bona}, least good. The particle than after a comparative is translated by \textbf{kam} (5). 

The phrase: as clear as possible, is best translated by: \textbf{maxim klara possible} (5).

\subsection*{IV. PRONOUN.}
\addcontentsline{toc}{subsection}{Pronoun—§§ 16–31}
\Centering (a) Personal Pronoun. \\ \justifying 

16. The personal pronouns are:

\Centering Singular:  \\ \justifying 
\begin{tabular}{p{0.33\linewidth} p{0.66\linewidth}}
1. Person, & \textbf{me}, I \\
2. Person (familiar), & \textbf{tu}, you, thou \\
\phantom{2. Person} (polite), & \textbf{vu}, you \\
3. Person (masculine), & \textbf{il}, he \\
\phantom{3. Person} (feminine), & \textbf{el}, she \\
\phantom{3. Person} (neuter), & \textbf{ol}, it \\
\phantom{3. Person} (general), & \textbf{lu}, she, she, it, when distinction of gender is unnecessary (III, 386, dec. 67) \\
& \textbf{lo}, it, referring to the contents of a sentence or infinitive, i.e., to a fact (II, 150; VI, 161, 238, 440) \\
\phantom{3. Person} (indefinite), & \textbf{on}, one, we, you, they.
\end{tabular}
\begin{center}Plural:\end{center} \vspace{-1em}
\begin{tabular}{p{0.33\linewidth} p{0.66\linewidth}}
1. Person, & \textbf{ni}, we \\
2. Person, & \textbf{vi}, you \\
3. Person (general), & \textbf{li}, they \\
\phantom{3. Person} (masculine), & \textbf{ili}, they \\
\phantom{3. Person} (feminine), & \textbf{eli}, they \\
\phantom{3. Person} (neuter), & \textbf{oli}, they: the longer forms being used when distinction of gender is desirable.
\end{tabular}
\textbf{Il}, \textbf{el}, \textbf{ol} are abbreviations of \textbf{ilu}, \textbf{elu}, \textbf{olu} so that their regular plural forms are \textbf{ili}, \textbf{eli}, \textbf{oli}.

17. When it is necessary to use a personal pronoun in accusative form, \textbf{n} is added directly to the pronouns ending in a vowel, and to the unabbreviated forms of those which end with a consonant: \textbf{men}, me; \textbf{ilun}, him; \textbf{elun}, her; \textbf{lin}, them; etc.

18. \textbf{Lu} is the common root of \textbf{ilu}, \textbf{elu}, \textbf{olu}. Its plural is regularly \textbf{li}. Both are genderless pronouns. \textbf{Lu} might appear to be identical with \textbf{ol}, but it is more general than the latter. \textbf{Lu} is preferably used when referring to living beings the gender of which is irrelevant, and \textbf{ol} preferably when referring to inanimate objects, although it would not be incorrect to use \textbf{lu} also in the second case and \textbf{ol} also in the first case (II, 3, 282; IV, 435, dec. 513; VII, 382, note). 

19. The indefinite personal pronoun \textbf{on} is abbreviated from \textbf{onu}. It is a nominative form and occurs only as subject. 

20. The reflexive pronoun of the third person singular and plural is \textbf{su}, himself, herself, itself, one's self, themselves. It occurs only after prepositions and as direct object of a verb. Preceding the predicate and not preceded by the subject it would therefore not need to receive the accusative ending \textbf{-n}. But for the sake of clearness and of uniformity the \textbf{n} is better appended: \textbf{sun la egoisto amas, ne altru}, it is himself that the egotist loves, not another one. 

21. The personal pronouns are not infrequently accompanied by the supplement \textbf{ipsa}, self: \textbf{me ipsa}, I myself; \textbf{el ipsa}, she herself; \textbf{li ipsa}, they themselves.

\Centering (b) Possessive Pronoun. \\ \justifying

22. The possessive pronouns are formed by adding \textbf{-a} to the personal pronouns (to the unabbreviated forms in the 3rd pers. singular): \\
Singular (one possessor): \textbf{mea}, my; \textbf{tua}, thy, your; \textbf{vua}, your; \textbf{lua}, his, her, its; \textbf{ilua}, \textbf{elua}, \textbf{olua}, his, her, its, respectively, when it is desirable to distinguish the gender of the possessor. \\
Plural (two or more possessors): \textbf{nia}, our; \textbf{via}, your; \textbf{lia}, their; \textbf{ilia}, \textbf{elia}, \textbf{olia}, their, when it is desirable to distinguish the gender of the possessors. \\
The reflexive possessive pronoun for singular and plural is \textbf{sua}, his, her, its, their. It is used when the possessor is the subject and the possessed object is a complement to the same predicate (see § 98c). 

23. The possessive pronouns being adjectives are invariable, t.i., the same form is used for singular (one possessed object) as for plural (two or more possessed objects). When used substantively, t.i., referring to a noun which is understood but not expressed the plural is preferably indicated in the same way as with other adjectives (see § 12): \textbf{mea}, mine; \textbf{lua}, his, hers, its; \textbf{nia}, ours, etc.; pural: \textbf{le mea}, mine; \textbf{le lia}, theirs; \textbf{le sua}, his, hers, its; etc. The forms \textbf{mei}, \textbf{lui}, \textbf{nii}, \textbf{lii}, etc., may also be used (see § 30, 2).

\Centering (c) Demonstrative Pronoun (Adjective). \\ \justifying

24. The demonstrative pronouns are \textbf{ca} or \textbf{ica}, this, pointing to a near object, \textbf{ta} or \textbf{ita}, that, pointing to a remote object. As adjectives they have the same form in singular and plural: \textbf{ca libro}, this book; \textbf{ta navo}, that ship; \textbf{ca} (\textbf{ica}) \textbf{plumi}, these pens; \textbf{ta} (\textbf{ita}) \textbf{landi}, those countries. As substantives they change the \textbf{-a} into \textbf{-i} in plural: \textbf{ca} (\textbf{ica}), this one; \textbf{ci} (\textbf{ici}), these; \textbf{ta} (\textbf{ita}), that one; \textbf{ti} (\textbf{iti}), those.

When it is desirable to indicate the gender of the object pointed at, the forms \textbf{ilea}, \textbf{ilci}; \textbf{elca}, \textbf{elci}; \textbf{olca}, \textbf{olei}; \textbf{ilta}, \textbf{ilti}; \textbf{elta}, \textbf{elti}; \textbf{olta}, \textbf{olti} are used. 

25. The above demonstrative pronouns point to a definite object (thing or person). When, however, the content of a sentence or of an infinitive, i.e., a fact, is to be pointed at, the forms \textbf{co}, \textbf{ico}, this; \textbf{to}, \textbf{ito}, that, are used (see § 16 and note 6). 

26. The general demonstrative pronoun is \textbf{ta} (\textbf{ita}), \textbf{ti} (\textbf{iti}), \textbf{to}. It is used whenever the nearness or remoteness of the object pointed at is irrelevant, and especially in conjunction with relative pronouns: \textbf{ta qua}, he, she who; \textbf{to quo}, that which; \textbf{ti qui}, those who; etc. (6). 

\Centering (d) Relative and Interrogative Pronoun (Adjective). \\ \justifying

27. The relative pronoun is \textbf{qua}, who, which, that. \textbf{Qua} is used both adjectively and substantively. As adjective it has the same form in plural as in singular, but as substantive it has in plural the form \textbf{qui}, who. 

The relative pronoun referring to facts is \textbf{quo}, which (see §§ 16, 25): \textbf{vu repartis tro frue, quo ne plezis a ni}, you left too early, which did not please us. 

\textbf{Quo} is used also when referring to other pronouns ending in \textbf{-o}: \textbf{co quo}, \textbf{to quo}; \textbf{nulo quo}, nothing that; \textbf{ulo quo}, something that; \textbf{omno quo}, everything that, etc. (7). 

28. The interrogative pronouns have the same forms as the relative pronouns: \textbf{qua}, who, which, what? \textbf{quo}, what? \textbf{quala}, what kind of? 

\textbf{Qua} as adjective is unchangeable, as substantive however it has in plural the form \textbf{qui}. The interrogative pronoun \textbf{quo} refers to something indefinite. \textbf{Qua libron vu selektis?} which book did you select? \textbf{Quala libron vu selek tis?} what kind of a book did you select? \textbf{Qua membri ne pagis ankore?} which members did not pay yet? \textbf{Quala planti kreskas en ica regiono?} what kind of plants grow in this region? \textbf{Qua dicis to?} who said that? \textbf{Qua esas el?} who is she? \textbf{Qui esas li?} who are they? \textbf{Qua esas ibe?} who is there? \textbf{Qui esas ibe?} who is (are) there? (when the question concerns more than one). \textbf{Quon vu vidis hiere? Me vidis la muzeo}, what did you see yesterday? I saw the museum. \textbf{Quo ibe plezis maxime a vu? La madono da Rafael}, what did you like there most? The Madonna by Raffael. 

29. A relative and interrogative pronoun used substantively must assume the accusative ending \textbf{-n} as direct object of a verb. For it always precedes it and is not preceded by the subject: \textbf{la libro quan me lektas}, the book that I read; \textbf{la vicini quin me amas}, the neighbors whom I love; \textbf{quan vu renkontris}, whom did you meet? But \textbf{qua libron vu lektas}, which book do you read?, although \textbf{quan libron vu lektas} would also be permissible (see § 13). 

\Centering (e) Indefinite Pronoun (Adjective). \\ \justifying

30. The indefinite pronouns (determinative adjectives) are: \textbf{altra}, other, another; \textbf{cetera}, remaining, the other, the rest of; \textbf{ipsa}, self; \textbf{irga}, any whatever; \textbf{kelka}, some, a few; \textbf{multa}, much, many; \textbf{nula}, no, no one; \textbf{omna}, all, every; \textbf{plura}, several; \textbf{poka}, little (opposite of \textbf{multa}, dec. 165, III, 593); \textbf{quala}, what, what kind of; \textbf{quanta}, how much; \textbf{sama}, same; \textbf{singla}, every single; \textbf{tala}, such; \textbf{tanta}, so much; \textbf{ula}, some, a certain. 

Three cases are to be distinguished in using the preceding pronouns. 

1) The pronouns are adjectives proper or epithets, i.e., they accompany a substantive that is expressed, in which case they are treated like any other adjective, having the same form in singular and plural: \textbf{altra pomo}, another apple; \textbf{altra pomi}, other apples; \textbf{irga profeto}, any prophet; \textbf{irga profeti}, any prophets. 

2) The pronouns are quasi-substantives, i.e., the substantive they would accompany is not expressed, but understood, in which case they end in \textbf{-a} in singular and in \textbf{-i} in plural, no matter whether the substantive understood is a thing or a person: \textbf{ca plumo ne skribas bone, prenez altra}, this pen does not write well, take another one; \textbf{en ca foresto esas grandega arbori, de qui kelki havas periferio de quar metri}, in this forest there are immense trees some of which have a periphery of four meters; \textbf{me ne savas, qua maristo di ca navo salvis la dronanta puerulo, me nur savas, ke ne esis ica ma altra}, I do not know which sailor of this ship saved the drowning boy, I only know that it was not this one but another one; \textbf{en ca kombato kelka soldati ocidesis, la ceteri salvesis, ma preske omni vundesis}, in this battle some soldiers were killed, the others were saved and almost all were wounded (see end of § 23). 

3) The pronouns are substantives proper, i.e., they are not accompanied by a substantive nor is there a substantive that would be understood. 

a) the substantive proper denotes a person, in which case the singular ends in \textbf{-u}, the plural in \textbf{-i}: \textbf{altru}, another; \textbf{altri}, others; \textbf{nulu}, nobody; \textbf{nuli}, none; \textbf{omnu}, everybody; \textbf{omni}, all; etc. (8). 

b) the substantive proper denotes anything whatsoever determined by the indefinite pronoun (by the determinative), in which case the singular ends in \textbf{-o} and there is no plural because such substantives by their nature have no plural: \textbf{altro}, something else; \textbf{cetero}, something remaining, the rest; \textbf{irgo}, anything whatsoever; \textbf{nulo}, nothing; \textbf{multo}, much; \textbf{kelko}, some quantity of; \textbf{omno}, everything; \textbf{samo}, the same; \textbf{ulo}, something. 

31. Any pronoun may receive one of the prefixes \textbf{il}, \textbf{el}, \textbf{ol} to indicate gender (dec. 486, IV, 433). These prefixes, however, should be used only for the purpose of avoiding ambiguity: \textbf{la spozino di mea amikulo, elquan vu vidis en la teatro}, the wife of my friend whom you saw in the theater; \textbf{quan} would refer to \textbf{amikulo}, \textbf{elquan} refers to \textbf{spozino}.

\subsection*{V. ADVERB.}
\addcontentsline{toc}{subsection}{Adverb—§§ 32–34}
32. The adverbs are either original or derived. The original adverbs have no characteristic grammatical ending: \textbf{ankore}, still; \textbf{balde}, soon; \textbf{forsan}, perhaps; \textbf{ja}, already; \textbf{jus}, just; \textbf{kam}, as; \textbf{maxim}, most; \textbf{min}, less; \textbf{minim}, least; \textbf{nun}, now; \textbf{nur}, only; \textbf{olim}, once, once upon a time; \textbf{plu}, more; \textbf{preske}, almost; \textbf{tam}, so; \textbf{tre}, very; etc. The derived adverbs have the ending \textbf{-e}. They are obtained from adjectives, substantives, and verbs by changing their characteristic endings into \textbf{-e}, and from prepositions by adding an \textbf{-e}; \textbf{bone}, well; \textbf{facile}, easily; \textbf{quale}, as, how; \textbf{tante}, so, so much; \textbf{irge}, in any manner whatever; \textbf{nule}, in no manner; \textbf{cadie}, to-day; \textbf{nultempe}, never; \textbf{cakaze}, in this case; \textbf{jorne}, in day time; \textbf{nokte}, at night; \textbf{okazione}, occasionally; \textbf{pede}, on foot; \textbf{konseque}, consequently; \textbf{itere}, again; \textbf{prefere}, preferably; \textbf{antee}, before; \textbf{kontree}, in a contrary manner; \textbf{dope}, behind, etc.

33. The ending \textbf{-e} is not fully characteristic of an adverb. A word ending in \textbf{-e} may be an adverb, preposition, conjunction, or interjection. 

34. The comparison of the adverbs is accomplished in the same way as that of the adjectives (see § 15) by means of \textbf{tam—kam}, \textbf{plu}, \textbf{min}, \textbf{maxim}, \textbf{minim}: 

\begin{center}
\begin{tabular}{l l}
    \begin{tabular}{l l}
    facile, & \\
    tam facile kam, & \\
    plu & \hspace{-4.5em}\rdelim\}{4}{*}[\,\, facile] \\
    min \\
    maxim \\
    minim \\
    \end{tabular}
&
    \begin{tabular}{l l}
    easily, & \\
    tam facile kam, & \\
    more & \hspace{-4.5em}\rdelim\}{4}{*}[\,\, easily] \\
    less \\
    most \\
    least \\
    \end{tabular}
\end{tabular}
\end{center}

Phrases like: as soon as possible, as much as possible, as well as possible are best translated by: \textbf{maxim balde posible}, \textbf{maxim multe posible}, \textbf{maxim bone posible} (see end of § 15, § 108, and note 5).

\subsection*{VI. VERB.}
\addcontentsline{toc}{subsection}{Verb—§§ 35–40}
35. The verb has characteristic endings in all modes and tenses. The participial endings \textbf{anta}, \textbf{inta}, \textbf{onta}, \textbf{ata}, \textbf{ita}, \textbf{ota}, however, are not specific, as shown by words like the following: \textbf{diamanto}, diamond; \textbf{eleganta}, elegant; \textbf{hiacinto}, hyacinth; \textbf{instinto}, instinct; \textbf{diskonto}, discount; \textbf{horizonto}, horizon; \textbf{karitato}, charity; \textbf{kubito}, cubitus; \textbf{kanoto}, canoe; \textbf{karoto}, carrot; etc. 

\Centering Active Voice. \\ \justifying

36. The infinitive ends in the present tense in \textbf{-ar}, in the past tense in \textbf{-ir}, and in the future tense in \textbf{-or}: \textbf{laudar}, to praise; \textbf{laudir}, to have praised; \textbf{laudor}, not translatable directly. 

The participle is formed by affixing to the root of the verb in the present tense \textbf{-ant}, in the past tense \textbf{-int}, and in the future tense \textbf{-ont}. To the form thus obtained is added \textbf{-a}, \textbf{-e}, \textbf{-o}, according to whether the participle is used as adjective, adverb, or noun; \textbf{konocanta}, knowing; \textbf{ridante}, laughingly; \textbf{kreinto}, creator, one who has created; \textbf{vinkonto}, the future victor. 

37. Finite modes. There is no distinction of person and number in the finite modes, for they are sufficiently marked through the subject, and in the imperative through the context. 

\Centering (a) Simple Tenses. \\ \justifying 

The indicative of the present tense is formed by the ending \textbf{-as}: \textbf{me amas}, I love; \textbf{ni esas}, we are. 

The past tense (imperfect) is formed by the ending \textbf{-is}: \textbf{vu skribis}, you wrote; \textbf{li respondis}, they answered. 

The future tense is formed by the ending \textbf{-os}: \textbf{el venos}, she will come; \textbf{tu departos}, you will depart. 
The conditional is formed by the ending \textbf{-us}: \textbf{il pagus}, he would pay; \textbf{ni joyus}, we would rejoice. 

The imperative (subjunctive, optative) is formed by the ending \textbf{-ez}: \textbf{irez}, go; \textbf{il lektez}, he may read; \textbf{ni esperez}, let us hope. 

\Centering (b) Perfect (Compound) Tenses. \\ \justifying

The perfect (compound) tenses are formed either by combining the simple tenses of the verb \textbf{esar}, to be, with the participle of the past or by means of the suffix \textbf{-ab} to which is attached the ending indicating tense and mode. The second method is not admissible in constructing the perfect proper. 

Perfect: \textbf{il esas arivinta}, he has arrived. 

Pluperfect: \textbf{ni esis komprinta} or \textbf{ni komprabis}, we had bought. 

Future perfect: \textbf{vu esos vendinta} or \textbf{vu vendabos}, you will have sold. 

Conditional perfect: \textbf{li esus dicinta} or \textbf{li dicabus}, they would have said. (3rd paragraph of note 53.) 

\small Remark. While for the perfect proper the form \textbf{amabas} is not used at all (IV, 321, dec. 407; V, 721, dec. 784), in the other compound tenses the forms with \textbf{-ab} are being employed more and more to the curtailment of the composed forms. \normalsize

\Centering Passive Voice. \\ \justifying

38. The passive participle is formed by affixing to the root of the verb in the present tense \textbf{-at}, in the past tense \textbf{-it}, and in the future tense \textbf{-ot}: \textbf{adjuntata}, being added; \textbf{kaptite}, having been caught (in a manner of one caught); \textbf{kondamnoto}, one who will be condemned. 
The other modes and tenses are formed by combining the verb \textbf{esar} with the passive participle, the present participle being used for the simple tenses and the past participle for the compound tenses. 

Present: \textbf{me esas protektata}, I am protected. 

Imperfect: \textbf{el esis admirata}, she was admired. 

Future: \textbf{ol esos konstruktata}, it will be constructed.

Conditional: \textbf{li esus punisata}, they would be punished. 

Imperative: \textbf{esez benedikata}, be blessed. 

Pluperfect: \textbf{ni esis duktita}, we had been led. 

Future perfect: \textbf{vi esos instruktita}, you will have been instructed. 

Conditional perfect: \textbf{vu esus sendita}, you would have been sent. 

The perfect does not occur, but might be constructed after the fashion of the other compound tenses: \textbf{lu esas konvertita}, he has been converted. 

Infinitive of the present: \textbf{esar konvinkata}, to be convinced. 

Infinitive of the future: \textbf{esor pardonata} (pardoned), not translatable directly. 

Infinitive of the perfect (past) admits of two constructions: \textbf{esir ornata} and \textbf{esar ornita}, to have been adorned. The first form is preferable as it refers to a past fact, an event, while the second one denotes a present state. (3rd paragraph of note 53.)

39. There is also a synthetic construction of the passive voice which consists in affixing the verb \textbf{esar} to the root of the verb: \textbf{propozesar}, to be proposed; \textbf{lezesir}, to have been injured; \textbf{ni imitesas}, we are imitated; \textbf{vu trompesis}, you were deceived; \textbf{ol destruktesabis}, it had been destroyed; \textbf{lu atakesabus}, he would have been attacked; etc., etc. 

40. Every polysyllabic word ending in \textbf{-ar}, \textbf{-ir}, \textbf{-or}, \textbf{-as}, \textbf{-is}, \textbf{-os}, \textbf{-us}, and \textbf{-ez} is a verb in the infinitive, present, past, etc. A similar inversion does not hold good for the participial endings (see §35). 

\subsection*{VII. NUMERAL.}
\addcontentsline{toc}{subsection}{Numeral—§§ 41–47}
41. Cardinal numbers: The simple numbers are: \textbf{zero}, 0; \textbf{un}, 1; \textbf{du}, 2; \textbf{tri}, 3; \textbf{quar}, 4; \textbf{kin}\footnotemark[1], 5; \textbf{sis}, 6; \textbf{sep}, 7; \textbf{ok}, 8; \textbf{non}, 9; \textbf{dek}, 10; \textbf{cent}, 100; mil, 1000; \textbf{milion}, million; \textbf{miliard}, thousand millions; \textbf{bilion}, 1,000,000,000,000; \textbf{trilion}; \textbf{quadrilion}; \textbf{quintilion}; \textbf{sextilion}; \textbf{septilion}; \textbf{oktilion}; \textbf{nonilion}; \textbf{decilion}.\footnotemark[2] 
\footnotetext[1]{About the deriviation of kin see Progr. I, p. 218.}
\footnotetext[2]{The numbers \textbf{milion}, \textbf{miliard}, \textbf{bilion}, etc., are to be considered as abbreviations of \textbf{miliono}, \textbf{miliardo}, \textbf{biliono}, etc., and have therefore the accent on the last syllable.}

The multiples of ten, hundred, thousand, million, etc., are formed in the manner of multiplication, and this by transforming the multiplicator into an adjective which is combined with the multiplicand into one word: \textbf{duadek}, 20; \textbf{triadek}, 30; \textbf{quaracent}, 400; \textbf{kinamil}, 5,000; \textbf{sisadekamil}, 60,000; \textbf{centamil}, 100,000; \textbf{triacentamil}, 300,000; \textbf{okamilion}, 8 millions; \textbf{milamilion}, 1,000 millions; \textbf{nonacentamilamilion}, 900,000 millions (VII, 37-41) (9). 

The numbers between these are formed by addition accomplished by means of the particle \textbf{e}, and: \textbf{dek e un}, 11; \textbf{dek e du}, 12; \textbf{duadek e tri}, 23; \textbf{triacent e okadek e sis}, 386; \textbf{duamil e kinacent e okadek e non}, 2,589 (10). 

\small Remark. A multiplicator of thousand, million, etc., that consists of more than one word is not combined with the multiplicand into one word: \textbf{dek e dua mil}, 12,000; \textbf{sepadek e kina mil}, 75,000; \textbf{triacent e quaradek e kina mil}, 345,000; \textbf{sisacent e sepadek e oka mil}, \textbf{nonacent e duadek e un}, 678,921; \textbf{duamil e quaracent e kinadek e sepa milion}, \textbf{sisacent e okadek e nona mil}, \textbf{triacent e duadek e un}, 2,457,689,321 (see § 130, and notes 10 and 57). \normalsize

42. By the ending \textbf{-o} numeral nouns are obtained which may be used also in the plural: \textbf{duo}, a pair; \textbf{tri dui}, three pairs; \textbf{dekeduo} (one word, see Progr. VII, p. 39, 3), a dozen; \textbf{quar dekedui}, four dozens; \textbf{duadeko}, a score; \textbf{sisadeko}, three score; \textbf{cento}, a hundred; \textbf{milo}, a thousand; etc. 

By the ending \textbf{-a} numeral adjectives, and by the ending \textbf{-e} numeral adverbs are constructed: \textbf{kanto una e sama}, one and the same song; \textbf{promeno dua}, a promenade by two; \textbf{pafo tria}, a shooting by three (by three persons at once); \textbf{li kantis trie}, they sang three together. 

From the number \textbf{un} the pronoun \textbf{unu} (plural: \textbf{uni}) is formed: \textbf{la unu—la altru} or \textbf{unu—altru}, one—the other; \textbf{la uni—la altri} or \textbf{uni—altri}, ones—the others. 

43. The ordinal numbers are formed by the suffix \textbf{-esm} and the grammatical endings \textbf{-a}, \textbf{-e}, \textbf{-o} furnishing an adjective, adverb, or substantive: \textbf{unesma}, first; \textbf{duesme}, secondly; \textbf{triesmo}, a third one. 

In the composed numbers the suffix is appended to the last component, and the particle \textbf{e} (\textbf{ed}) accomplishing the addition (see note 10) is connected with the numbers by hyphens (see de Beaufront, Exerc., 3rd edit., p. 14): 2,674th, \textbf{duamil-e-sisacent-e-sepadek-e-quaresma}. 

Analogous to the formation of the ordinal numbers is that of the expression \textbf{quantesma}, which of a series: \textbf{quantesma dio di la monato}, which day of the month? 

44. The fractional numbers are formed by the suffix \textbf{-im} and the grammatical endings \textbf{-a}, \textbf{-e}, \textbf{-o}, as the case may be: \textbf{duima litro}, half a litre; \textbf{kinima parto}, a fifth part; \textbf{milima parto}, a thousandth part; \textbf{triime}, thirdwise, by thirds; \textbf{duimo}, a half; \textbf{triimo}, a third; \textbf{quarimo}, a fourth; \textbf{tri duadek-e kinimi}, three twenty-fifths. 

A fraction, especially where the single numbers are large, may also be expressed by the preposition \textbf{sur} instead of the fractional suffix: 12/100, \textbf{dek e du centimi} or \textbf{dek e du sur cent}; 10/2,000, \textbf{dek duamilimi} or \textbf{dek sur duamil}. 

45. The multiplicative numbers are formed by the suffix \textbf{-opl} and the grammatical endings \textbf{-a}, \textbf{-e}, \textbf{-o}; \textbf{duopla parto}, a double part; \textbf{triople}, in a threefold manner; \textbf{dekoplo}, the tenfold. 

46. The numbers connected with the word time assume the suffix \textbf{-foy}. Generally they are employed adverbially, but sometimes also adjectively: \textbf{dufoye}, twice; \textbf{trifoye}, three times; \textbf{quarfoye}, four times; \textbf{kinfoya atako}, an attack undertaken five times; \textbf{trifoya eko}, an echo repeated three times. 

The word times in multiplication proper is translated with the suffix \textbf{-opl}, not with \textbf{-foy}: \textbf{triople kin esas dek e kin}, three times five are fifteen. The preposition \textbf{per}, by, may also be used for multiplication, especially with complicated numbers: \textbf{kinadek e quar per triacent e sisadek e sep}, 54 × 367. 

47. The distributive numbers are formed by the suffix \textbf{-op} and occur usually in adverbial form: \textbf{la soldati marchas triope}, the soldiers are marching three abreast. The corresponding interrogative adverb is \textbf{quantope}: \textbf{quantope mar chas la soldati}, how many abreast do the soldiers march? 

The adverb \textbf{pokope}, little by little, is to be mentioned here.

\subsection*{VIII. PREPOSITION.}
\addcontentsline{toc}{subsection}{Preposition—§ 48}
48. The prepositions have no characteristic ending. They are original words from which other parts of speech may be derived: \textbf{apud}, at, near; \textbf{apuda strado}, a nearby street; \textbf{apudesar}, to be near; \textbf{ante}, before; \textbf{antea}, previous; \textbf{kontre}, against; \textbf{kontrea}, contrary; \textbf{kun}, with; \textbf{kune}, together; etc. 

Frequently occurring prepositions are: \textbf{ad} (elided to \textbf{a} when euphony permits)\footnotemark[1], to; \textbf{ante}, before (of time); \textbf{avan}, before (of place); \textbf{che}, at, in the house of; \textbf{cirkum}, around; \textbf{da}, by, through (with the passive voice); \textbf{de}, of, from; \textbf{di}, of (marking the genitive); \textbf{dum}, during; \textbf{ek}, from, out of; \textbf{en}, in; \textbf{inter}, between; \textbf{malgre}, in spite of; \textbf{per}, by means of; \textbf{pri}, about, concerning; \textbf{pro}, because of; \textbf{por}, for; \textbf{segun}, according to; \textbf{sen}, without; \textbf{sub}, under; \textbf{sur}, on; \textbf{super}, over; \textbf{til}, till, until; \textbf{tra}, through; \textbf{trans}, on the other side of; \textbf{ultre}, besides; \textbf{vice}, instead of; etc. 

\textbf{Ye} is a preposition with no fixed meaning, employed when no other preposition fits the case.

\subsection*{IX. CONJUNCTION.}
\addcontentsline{toc}{subsection}{Conjunction—§§ 49–51}
49. The conjunctions have no characteristic ending. Some of them are original words, others are adverbs used conjunctionally. Some conjunctions consist of one part, others of two parts the first of which is an adverb or a preposition. 

50. Coordinate conjunctions are: \textbf{do}, therefore, then; \textbf{ed} (or \textbf{e})\footnotemark[1], and; \textbf{ma}, but; \textbf{nam}, for; \textbf{nek}, nor; \textbf{nek \ldots nek}, neither \ldots nor; \textbf{od} (or \textbf{o})\footnotemark[1], or; \textbf{od \ldots od}, either \ldots or; or, now, but; \textbf{sive \ldots sive}, be it that \ldots be it that; \textbf{tamen}, however; \textbf{vel}, or; \textbf{vel \ldots vel}, either \ldots or; \textbf{ya}, certainly, indeed. 

The coordinate conjunction \textbf{kad} (or \textbf{ka})\footnotemark[1] introduces a direct question: \textbf{kad vu savas}, do you know? \textbf{Kad ne}, is that not so? 
\footnotetext[1]{About \textbf{a}, \textbf{ad}; \textbf{e}, \textbf{ed}; \textbf{o}, \textbf{od}; \textbf{ka}, \textbf{kad} see Progr. II, pp. 15, 169, 579. The forms with \textbf{d} are used before a vowel and the \textbf{d} of \textbf{ad} is never to be elided in compositions. \label{elision}}

51. Subordinate conjunctions are: \textbf{kad} (or \textbf{ka}), whether; \textbf{kande}, when; ke, that; \textbf{quale}, as; \textbf{quankam}, although; \textbf{quante}, as much as; \textbf{se}, if; \textbf{ube}, where; etc. 

Subordinate conjunctions formed from adverbs are: \textbf{kaze ke}, in case that; \textbf{kondicione ke}, under the condition that; \textbf{omnafoye ke}, every time that; same \textbf{kam}, in the same manner as; \textbf{tale ke}, so that; \textbf{tam longe kam}, so long as; \textbf{tam ofte kam}, as often as; \textbf{tante ke}, so much that; \textbf{tante longe ke}, so long that; \textbf{tante ofte ke}, so often that; \textbf{time ke}, for fear that; \textbf{tante plu \ldots ke}, so much more as; \textbf{quante plu \ldots tante plu}, the more \ldots the more; \textbf{segun quante}, as far as; etc. 

Most of the prepositions are changed into subordinate conjunctions through the addition of \textbf{ke}: \textbf{per ke}, through that; \textbf{por ke}, in order that; \textbf{pro ke}, because; \textbf{segun ke}, according to whether; \textbf{depos ke}, since; etc.

\subsection*{X. INTERJECTION.}
\addcontentsline{toc}{subsection}{Interjection—§ 52}
52. The interjections are partly international sounds as \textbf{hura!} = hurrah! ha!, he!, ho!, \textbf{ve!} = alas!; \textbf{nu} = well, well then; \textbf{fi!} = fy, faugh! Partly they are words of other parts of speech as \textbf{ya}, well, very well; \textbf{yen!} = here!, \textbf{vere!}, \textbf{certe!} = verily!; \textbf{brave!} = bravo!; \textbf{bone!} = well!; \textbf{adbase!} = down with (II, 163); \textbf{adavan!} = go ahead; \textbf{fore!} = away!; \textbf{dope!} = back!; \textbf{haltez!} = stop!; \textbf{silencez!} = hush!; \textbf{helpo!} = help!; \textbf{shamo!} = shame!; \textbf{adio!} = good-bye! 
