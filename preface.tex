\lettrine{T}{o} enter upon the study of an international language in an intelligent manner some knowledge of the history of the attempts to create an international language is indispensable. A few of the most important historical data will therefore be briefly presented here. For a more thorough information reference is made to the excellent work on the subject by Dr.~L.~Couturat and Dr.~L.~Leau: Histoire de la Langue Universelle, a volume of nearly 600 pages treating historically and critically all artificial languages devised until about 1903. 

The history of artificial international languages dates back nearly 300 years. During this period about 250 idioms have been constructed. They reveal a gradual evolution from crude linguistic devices, hardly deserving the name language, in the beginning, to more or less perfect languages in recent times, the later systems, as a rule, surpassing the earlier ones in quality. 

From the beginning of this period to our times the problem of an artificial international language has occupied, to a greater or lesser degree, the minds of some of the greatest thinkers and scholars. Among them are to be mentioned Bacon, Descartes, Leibnitz, Voltaire, Alex.~v.~Humboldt, J.~v.~Grimm, A.~M.~Lamartine, Victor Hugo, Ernest Naville, W.~Förster, H.~Schuchardt, W.~Ostwald, Otto Jespersen, etc. 

Of all artificial languages Volapük and Esperanto have gained the widest publicity. It is impossible to establish which of the two had the larger number of adherents, for the reports in this respect are entirely unreliable. 

Volapük was devised by the German priest Schleyer of Konstanz, Baden, and published in 1880. It spread rapidly in all civilized countries and flourished until 1889. Then it began to lose ground and declined as rapidly as it had risen and was soon forgotten. Intrinsic defects of the language, which became apparent at the congress of Volapükists in Paris in 1889, and failure of its advocates to come to an understanding as to the way to amend them furnished the causes for its downfall. 

Esperanto was devised by the Russian physician Dr.~L.~Zamenhof of Warsaw while Volapük was still at its height and published in 1887. Its progress was very slow until the Frenchman Marquis L. de Beaufront, an eminent philologist, began to take interest in it about 1890. He became the second father of Esperanto by creating a real language out of the crude sketch of its inventor, and to him more than to anybody else is due the fact that Esperanto became known in every civilized country gaining everywhere great numbers of adherents many of whom still swear faithful allegiance to it. 

Esperanto has attained a high degree of perfection and in many respects it excels all artificial languages preceding it and those which were devised as a result of the disintegration of Volapük. Because of its good features competent students of the problem of an international language praised and advocated it. They were well aware that it contained also substantial defects and pointed them out. But they recognized that these imperfections could be remedied and hoped that the necessary reforms would be introduced soon enough to prevent the language from deteriorating through its defects to such an extent as to be beyond correction. 

The proper opportunity for starting the reforms of Esperanto came in 1907. In October of this year the Committee of the International Delegation for the Adoption of an International Auxiliary Language convened in Paris for the purpose which had brought the Delegation into existence at the Paris exposition in 1900. This purpose was to procure to the civilized nations an auxiliary international language either by endorsing one of the existing artificial languages or by constructing a new one if none of the existing ones were found adequate. The Committee composed of scholars of international reputation examined many artificial languages and rejected them all, including Esperanto in its original form. It adopted, however, Esperanto in principle under reserve of certain modifications to be introduced by the Permanent Commission of the Delegation in the sense defined by the report of the secretaries of the Committee and by the project of Ido which had also been presented for examination. Besides, the Committee recommended an understanding with the Linguistic Committee for Esperanto. When subsequently this understanding was sought, the majority of the Esperanto Committee, or at least that number of its members which was given out as its majority, refused absolutely to recognize the necessity of reforming Esperanto, and all negotiations in this respect had to be broken off. The Permanent Commission of the Delegation thereupon went its own way beginning a systematic further development and perfection of the project of Ido. 

The Esperanto Committee also went its own way. But its failure to agree with the Permanent Commission of the Delegation caused Esperanto to lose ground everywhere. The deterioration of the language which had already commenced before continued now unchecked and is still continuing so that the language has become so overcharged with absurdities as to be entirely unfit for the role of an international language. 

It is evident that the history of Volapuk is repeating itself in Esperanto, though at a slower pace owing to the relative superiority of the latter over the former. The primary cause for the retrogression of Esperanto lies in substantial philological defects which have called forth sharp criticism from the most competent sources. The foremost of these linguistic defects is a faulty system of derivation or rather lack of a proper system of derivation. For while other imperfections, as peculiar letters not contained in any other language, cacophony, too much inflection, can not render the language more imperfect than it was in its original form, faulty derivation increases the faults in leading to absurd word formations which by-and-by become an integral part of the language. This defect of Esperanto has been most clearly expounded by Dr.~L.~Couturat in his lucid monograph: Etude sur la Dérivation en Esperanto. 

These linguistic defects, however, could and would have been remedied but for the chief cause of Esperanto's decline. It consists in the most influential leaders of Esperanto having endeavored and succeeded to make Esperanto the medium of a sort of new religion called Esperantism or Homaranism and proclaiming that a common language of the nations in general and of Esperanto in particular would bring about universal brotherhood of men. As the medium of a religion Esperanto became inviolable, was not to be subjected to criticism, much less to change or reform. The fallacy of this new religion has been pointed out by the author (Progreso, I, p. 83). To still further strengthen the inviolability of the medium of the new religion the ridiculous assertion was put forward that Esperanto was the living language of a living people---vivanta lingvo de vi vanta popolo. Here every argument had to stop, for a living language must indeed not be touched. 

Entirely different was the policy of the Permanent Commission of the Delegation which undertook to develop the project of Ido. It founded in its behalf an official organ, Progreso, and an academy and invited criticism from all sides to be either published in this magazine or presented directly to the academy which was then to introduce the change in the language necessary to correct the imperfect feature criticised. 

Ido is the pseudonym under which the foremost Esperantist Marquis L. de Beaufort presented to the Committee of the Delegation for examination several documents, a complete grammar, Grammaire Complète, an elementary grammar, and a short dictionary. They completely outlined a new language devoid of the defects of Esperanto while preserving its good features. It is actually this language that the Committee adopted after rejecting all others examined. But the decision to this effect was evidently drafted with the view to gain the adherence to the new language of the great number of those who favored original Esperanto. The decision reads: “Le Comité a décidé d’adopter en principe l’Esperanto \ldots sous la resérve de certaines modifications \ldots à exécuter \ldots dans le sens défini \ldots par le projet de Ido.” Esperanto, accordingly, was to be modified in conformity with this project, not inversely. Any other modern artificial language modified in this sense would have given the same result. Only a few more modifications would have been required. For the modern artificial languages resemble each other a good deal. The view, therefore, that the new language represents simplified Esperanto, has to be taken cum grano salis. At the present day after further deterioration of Esperanto on one side, and considerable further improvement of the new language through its academy on the other side, the difference between the two is very great, so that that view cannot at all be maintained any longer. 

During the negotiations aiming at an understanding with the Esperanto Committee and for quite some time after they had been broken off, the new language was officially known under the name International Language of the Delegation. For about two years the name Ilo, constructed from the letters I and L of the term International Language, was favored by many and frequently used by writers. Several other names were proposed. But finally the pseudonym Ido of Mr. de Beaufront was adopted by the Committee of the Union of Idists (II, 288) and found general acceptance as the name of the new language. 

Already from the beginning Ido was far superior to any artificial language ever devised. Through the painstaking strictly scientific work of the academy during the past seven years and through its liberal policy of inviting and considering every criticism, important changes and additions, notably in the vocabulary, have been made whereby Ido has attained a still higher degree of perfection. Recently the academy has decreed a period of stability of ten years during which no further changes are to be made. It may safely be stated that only very few will be required after the lapse of this period. The construction of Ido is thus completed and authors of text books are now enabled to offer to students work of lasting value.

Ido is therefore to be defined as the language laid down in the “Grammaire Complète par Ido” and modified and amplified by the decisions of the Ido academy. This text book conforms strictly to this definition, setting forth only the rules, and all the rules, sanctioned by these two authorities. The official organ Progreso has been freely consulted and numerous references to it have been given for explaining and illustrating grammatical rules and forms of good style. 

The object of this work is not only to enable the average scholar to acquire a good working knowledge of the language, but also to acquaint the accomplished student with the principles to be observed in the construction of an artificial language, to point out to him niceties of grammar and stylistic that are requisite to proficiency in the international language, and to help him over many linguistic difficulties. All grammars of Esperanto and Ido are incomplete in this respect. The serious student of Ido had therefore to resort frequently to the official magazine for enlightenment as shown by the numerous linguistic questions (“linguala questioni”) treated extensively in almost every number of the seven volumes of Progreso and representing its best feature. To supply this want has been a most important aim of this work. \begin{flushright}New York, January 1914.\end{flushright}

The first draft of this book was read and approved in the spring of 1914 by the foremost authority on the problem of an international language, Dr.~L.~Couturat. He characterized the book as very carefully and conscientiously composed (“la libro esas tre sorgoze e konciencoze kompozita”). This warranted the belief that the work did not remain far away from its aim. Its publication was to take place in the fall of 1914, but was prevented by the outbreak of the world war. The manuscript was locked away to be taken up again at a time more propitious for a work intended to promote international intercourse. When this time approached not long ago, the work was resumed. A good deal of material was added to bring the book still nearer its aim. Most of the material of importance to the accomplished student and teacher, which in the first draft was scattered here and there in the form of notes, was gathered into a separate part, the fourth one. The average scholar needs to pay but little attention to this part. 
\RaggedLeft New York, June 1919. \justifying
